% Options for packages loaded elsewhere
\PassOptionsToPackage{unicode}{hyperref}
\PassOptionsToPackage{hyphens}{url}
%
\documentclass[
]{book}
\usepackage{lmodern}
\usepackage{amssymb,amsmath}
\usepackage{ifxetex,ifluatex}
\ifnum 0\ifxetex 1\fi\ifluatex 1\fi=0 % if pdftex
  \usepackage[T1]{fontenc}
  \usepackage[utf8]{inputenc}
  \usepackage{textcomp} % provide euro and other symbols
\else % if luatex or xetex
  \usepackage{unicode-math}
  \defaultfontfeatures{Scale=MatchLowercase}
  \defaultfontfeatures[\rmfamily]{Ligatures=TeX,Scale=1}
\fi
% Use upquote if available, for straight quotes in verbatim environments
\IfFileExists{upquote.sty}{\usepackage{upquote}}{}
\IfFileExists{microtype.sty}{% use microtype if available
  \usepackage[]{microtype}
  \UseMicrotypeSet[protrusion]{basicmath} % disable protrusion for tt fonts
}{}
\makeatletter
\@ifundefined{KOMAClassName}{% if non-KOMA class
  \IfFileExists{parskip.sty}{%
    \usepackage{parskip}
  }{% else
    \setlength{\parindent}{0pt}
    \setlength{\parskip}{6pt plus 2pt minus 1pt}}
}{% if KOMA class
  \KOMAoptions{parskip=half}}
\makeatother
\usepackage{xcolor}
\IfFileExists{xurl.sty}{\usepackage{xurl}}{} % add URL line breaks if available
\IfFileExists{bookmark.sty}{\usepackage{bookmark}}{\usepackage{hyperref}}
\hypersetup{
  pdftitle={Methoden en Statistiek 1},
  hidelinks,
  pdfcreator={LaTeX via pandoc}}
\urlstyle{same} % disable monospaced font for URLs
\usepackage{longtable,booktabs}
% Correct order of tables after \paragraph or \subparagraph
\usepackage{etoolbox}
\makeatletter
\patchcmd\longtable{\par}{\if@noskipsec\mbox{}\fi\par}{}{}
\makeatother
% Allow footnotes in longtable head/foot
\IfFileExists{footnotehyper.sty}{\usepackage{footnotehyper}}{\usepackage{footnote}}
\makesavenoteenv{longtable}
\usepackage{graphicx}
\makeatletter
\def\maxwidth{\ifdim\Gin@nat@width>\linewidth\linewidth\else\Gin@nat@width\fi}
\def\maxheight{\ifdim\Gin@nat@height>\textheight\textheight\else\Gin@nat@height\fi}
\makeatother
% Scale images if necessary, so that they will not overflow the page
% margins by default, and it is still possible to overwrite the defaults
% using explicit options in \includegraphics[width, height, ...]{}
\setkeys{Gin}{width=\maxwidth,height=\maxheight,keepaspectratio}
% Set default figure placement to htbp
\makeatletter
\def\fps@figure{htbp}
\makeatother
\setlength{\emergencystretch}{3em} % prevent overfull lines
\providecommand{\tightlist}{%
  \setlength{\itemsep}{0pt}\setlength{\parskip}{0pt}}
\setcounter{secnumdepth}{5}
\usepackage{booktabs}
\usepackage{amsthm}
\makeatletter
\def\thm@space@setup{%
  \thm@preskip=8pt plus 2pt minus 4pt
  \thm@postskip=\thm@preskip
}
\makeatother
\usepackage[]{natbib}
\bibliographystyle{apalike}

\title{Methoden en Statistiek 1}
\author{true}
\date{8 juli 2020}

\begin{document}
\maketitle

{
\setcounter{tocdepth}{1}
\tableofcontents
}
\hypertarget{voorwoord}{%
\chapter*{Voorwoord}\label{voorwoord}}
\addcontentsline{toc}{chapter}{Voorwoord}

Data spelen een steeds belangrijker rol, ook in de geesteswetenschappen.
De beschikbaarheid van digitale gegevens (o.a. tekst, spraak, video, en gedragsregistraties) leidt tot nieuwe onderzoeksvragen, die vooral met kwantitatieve methoden beantwoord worden.
Dit boek biedt onderzoekers en studenten een overzicht en inleiding van de belangrijkste kwantitatieve methoden en statistische technieken in de geesteswetenschappen. Het boek geeft de lezer een stevig methodologisch fundament voor kwantitatief onderzoek, en biedt een inleiding in de meest gebruikte statistische technieken om gegevens te beschrijven en om hypothesen te toetsen. Daarmee is de lezer ook in staat om kwantitatief onderzoek kritisch te beoordelen.

Dit tekstboek wordt gebruikt als leesstof bij de cursus \emph{Methoden en Statistiek 1} aan de Universiteit Utrecht. Het boek is tevens bruikbaar voor zelfstudie op inleidend niveau, voor iedereen die meer wil weten over methoden en statistiek.

De hoofdtekst is gevrijwaard van wiskundige afleidingen en formules, die voor geesteswetenschappers immers weinig bruikbaar zijn. De uitleg is vooral conceptueel, en rijk aan voorbeelden van geesteswetenschappelijk onderzoek. Waar nodig worden formules aangeboden in een aparte paragraaf.

Dit boek bevat ook aanwijzingen over hoe de besproken statistische analyses en visualisaties uitgevoerd kunnen worden in twee veelgebruikte programma's, nl. SPSS (versie 22 en later) en R (versie 3.0 en later). Ook deze aanwijzingen staan los van de hoofdtekst, in afzonderlijke paragrafen.

Graag willen we onze mede-docenten danken voor de vele discussies en voorbeelden die op enige wijze verwerkt zijn in dit tekstboek. Onze studenten danken we voor hun nieuwsgierigheid en nauwkeurigheid die geleid heeft tot deze versie van dit tekstboek.

Ook betonen wij grote dank aan
Willemijn Heeren, Gerrit Bloothooft, Marijn Struiksma,
Margot van den Berg,
Els Rose,
Tobias Quené
en Kirsten Schutter
voor hun adviezen, data, en/of commentaar bij eerdere versies.

Utrecht, december 2016 - juli 2020

Hugo Quené, \url{https://www.hugoquene.nl}

Huub van den Bergh

\begin{center}\rule{0.5\linewidth}{0.5pt}\end{center}

\hypertarget{notatie}{%
\section*{Notatie}\label{notatie}}
\addcontentsline{toc}{section}{Notatie}

In aansluiting op het internationale gebruik en op de conventies van Engelstalige tijdschriften gebruiken we de punt als decimaalteken; we schrijven dus \(\frac{3}{2}=1.5\). Hierbij is een waarschuwing op zijn plaats: het decimale symbool kan verschillen tussen computers, en zelfs tussen programma's op dezelfde computer. Controleer dus welk decimaal symbool gebruikt wordt door (elk programma op) jouw computer.

\hypertarget{licentie}{%
\section*{Licentie}\label{licentie}}
\addcontentsline{toc}{section}{Licentie}

This document is licensed under the \emph{GNU GPL 3} license (for details see
\url{https://www.gnu.org/licenses/gpl-3.0.en.html}). It was created with the \texttt{bookdown} package \citep{R-bookdown} in \href{https://www.rstudio.com}{Rstudio}.

\hypertarget{over-de-auteurs}{%
\section*{Over de auteurs}\label{over-de-auteurs}}
\addcontentsline{toc}{section}{Over de auteurs}

Beide auteurs zijn verbonden aan de Faculteit Geesteswetenschappen van de Universiteit Utrecht. HQ is hoogleraar Kwantitatieve Methoden van Empirisch Onderzoek in de Geesteswetenschappen, en geeft daarnaast leiding aan het Centre for Digital Humanities. HvdB is hoogleraar Didactiek en Toetsing van het Taalvaardigheidsonderwijs, en is daarnaast vaksectievoorzitter Nederlands bij het College voor Toetsen en Examens (CvTE).

\hypertarget{part-deel-i-methodologie}{%
\part*{Deel I: Methodologie}\label{part-deel-i-methodologie}}
\addcontentsline{toc}{part}{Deel I: Methodologie}

\hypertarget{ch:inleiding}{%
\chapter{Inleiding}\label{ch:inleiding}}

In dit tekstboek worden de grondbeginselen, methoden en technieken van
empirisch wetenschappelijk onderzoek besproken, zowel in algemene zin
als toegespitst op het brede domein van taal en communicatie. We zullen ons bezighouden met vragen als: Wat is een goede onderzoeksvraag? Welke methode is de beste om de onderzoeksvraag te beantwoorden? Hoe kunnen onderzoekers zinnige en valide conclusies trekken uit (statistische analyses van) hun gegevens? In dit tekstboek beperken we ons tot de belangrijkste grondbeginselen, en tot de belangrijkste methoden en technieken. In dit eerste hoofdstuk zullen we een overzicht geven van verschillende typen en vormen van wetenschappelijk onderzoek. In het vervolg van dit tekstboek geven we de meeste aandacht aan methoden van wetenschappelijk onderzoek waarbij empirische observaties uitgedrukt worden in de vorm van getallen (kwantitatief), die geanalyseerd worden met behulp van statistische technieken.

\hypertarget{sec:wetenschappelijk-onderzoek}{%
\section{Wetenschappelijk onderzoek}\label{sec:wetenschappelijk-onderzoek}}

Om te beginnen moeten we een vraag stellen die terugslaat op de
allereerste zin hierboven: wat is eigenlijk wetenschappelijk onderzoek?
Wat is het verschil tussen wetenschappelijk en niet-wetenschappelijk
onderzoek (bijv. door onderzoeksjournalisten)? Onderzoek dat een
wetenschapper uitvoert, hoeft nog geen wetenschappelijk onderzoek te
zijn. Evenmin is journalistiek onderzoek per definitie
onwetenschappelijk omdat het door een journalist wordt uitgevoerd. In
dit tekstboek hanteren we de volgende definitie \citep[p.14]{KL00}:

\begin{quote}
``Scientific
research is systematic, controlled, empirical, amoral, public, and
critical investigation of natural phenomena. It is guided by theory and hypotheses about the presumed relations among such phenomena.''
\end{quote}

Wetenschappelijk onderzoek is systematisch en gecontroleerd.
Wetenschappelijk onderzoek is zodanig ontworpen dat we geloof kunnen
hechten aan de conclusies, omdat die conclusies goed onderbouwd zijn.
Het onderzoek kan door anderen herhaald worden, met (hopelijk) dezelfde
resultaten. Deze eis van repliceerbaarheid maakt ook dat
wetenschappelijk onderzoek zeer nauwgezet wordt ontworpen en uitgevoerd
(zie Hoofdstukken \ref{ch:integriteit} en \ref{ch:ontwerp}). De sterkste vorm van controle is die van een
wetenschappelijk experiment; we besteden daarom in dit tekstboek veel
aandacht aan experimenteel onderzoek (§\ref{sec:experimenteel-onderzoek}). Mogelijke alternatieve
verklaringen voor het onderzochte verschijnsel worden één voor één
onderzocht en zo mogelijk uitgesloten, zodat tenslotte slechts één
verklaring overblijft \citep{KL00}. Die verklaring vormt dan onze
wetenschappelijk onderbouwde conclusie of theorie over het onderzochte
verschijnsel.

Ook wordt in de definitie gesteld dat wetenschappelijk onderzoek
\emph{empirisch} van aard is. De conclusies die de onderzoeker trekt moeten
uiteindelijk gebaseerd zijn op (systematische en gecontroleerde)
waarnemingen of observaties van een verschijnsel in de werkelijkheid --- bijvoorbeeld op de waargenomen inhoud van een tekst, of op het
waargenomen gedrag van een proefpersoon. Als die waarneming ontbreekt,
dan kunnen de eventuele conclusies niet logisch verbonden worden met de werkelijkheid, waardoor ze geen wetenschappelijke waarde hebben.
Vertrouwelijke gegevens uit een onbekende bron, of inzichten verkregen
in een droom of in een mystieke beleving, zijn niet empirisch
onderbouwd, en kunnen dus niet de basis vormen van een wetenschappelijke theorie.

\hypertarget{sec:theorie}{%
\subsection{Theorie}\label{sec:theorie}}

Het doel van wetenschappelijk
onderzoek is te komen tot een theorie over een deel van de
werkelijkheid. Die theorie is te zien als een coherente en consistente
verzameling van ``justified true beliefs'' \citep{Mort03}. In deze
overtuigingen, en in de theorie, wordt geabstraheerd van de complexe
werkelijkheid van de natuurlijke verschijnselen, naar een abstract
mentaal \emph{construct}, dat uit zijn aard niet rechtstreeks waarneembaar
is. Voorbeelden van dergelijke constructen zijn: leesvaardigheid,
intelligentie, activatie-niveau, verstaanbaarheid, omvang van iemands
actieve woordenschat, schoenmaat, woon-werk-afstand, introvertheid, etc.

Een onderzoeker definieert in een theorie niet alleen verschillende
constructen, maar ook specificeert hij de \emph{verbanden} of relaties tussen
deze constructen. Pas wanneer zowel de constructen gedefinieerd zijn als
de relaties tussen de constructen gespecificeerd zijn, kan een
onderzoeker komen tot een systematische verklaring van het onderzochte
verschijnsel. Deze verklaring of theorie kan weer de basis zijn van een
\emph{voorspelling} over het onderzochte verschijnsel: het aantal gesproken
talen op de wereld zal verminderen in de 21e eeuw; teksten zonder
voegwoorden zullen moeilijker te begrijpen zijn dan teksten met
voegwoorden; kinderen die tweetalig opgroeien zullen niet slechter
presteren op school dan eentalige kinderen.

Wetenschappelijk onderzoek is er in vele verschillende typen en vormen,
die op verschillende manieren ingedeeld kunnen worden. In de volgende sectie \ref{sec:paradigmata} bespreken we een indeling op basis van
paradigma, de manier waarop de onderzoeker tegen de werkelijkheid
aankijkt. Onderzoek kan ook ingedeeld worden op een continuüm van
`zuiver theoretisch' naar `toegepast'. Een derde manier om onderzoek in
te delen is gericht op het type onderzoek, bijvoorbeeld
instrumentatieonderzoek (§\ref{sec:instrumentatie-onderzoek}), beschrijvend onderzoek (§\ref{sec:beschrijvend-onderzoek}),
en experimenteel onderzoek (§\ref{sec:experimenteel-onderzoek}).

\hypertarget{sec:paradigmata}{%
\section{Paradigmata}\label{sec:paradigmata}}

Eén criterium om typen onderzoek te onderscheiden is op basis van het
gebruikte paradigma, de manier waarop de onderzoeker tegen de
werkelijkheid aankijkt. In dit tekstboek besteden we nagenoeg alleen
aandacht aan het empirisch-analytisch paradigma, omdat dit het meest
uitgewerkte en meest invloedrijke paradigma is. Heden ten dage kan deze
benadering opgevat worden als `de' standaardopvatting, waar andere
paradigma's zich min of meer tegen afzetten.

Binnen het \emph{empirisch-analytische paradigma} onderscheiden we twee
varianten: het positivisme en het kritisch-rationalisme. Beide
stromingen hebben gemeen dat er aangenomen wordt dat er wetmatigheden
zijn die `ontdekt' kunnen worden: verschijnselen kunnen beschreven en
verklaard worden in abstracte termen (constructen). Het verschil tussen
beide stromingen binnen de empirisch-analytische traditie is gelegen in
de pretentie van de uitspraken die gedaan worden. Volgens de
positivisten is het mogelijk om uitspraken te doen vanuit feitelijke
waarnemingen naar een theorie. Op basis van de observaties kunnen we
generaliseren naar een algemeen geldende regel, door middel van
inductie. (De vogels die ik zie, die hoor ik ook fluiten, dus alle
vogels fluiten.)

De tweede stroming is het kritisch-rationalisme. De aanhangers van deze
stroming keren zich tegen bovengenoemde inducties: al hoor ik talloze
vogels ook fluiten, dan nog kan ik geen zekerheid verkrijgen over de
veronderstelde algemene regel. Maar we kunnen het wel omkeren, en
proberen aan te tonen dat de veronderstelde algemene regel of hypothese \emph{niet} juist is. Hoe werkt dat? Op basis van de algemeen geldende regel
kunnen we voorspellingen afleiden voor specifieke observaties, door
middel van deductie. (Als alle vogels fluiten, dan moet het zo zijn dat
alle vogels in mijn steekproef fluiten.) Als niet alle vogels in mijn
steekproef fluiten, dan is de algemene regel blijkbaar onjuist. Dit
wordt het falsificatie-principe genoemd; we bespreken dat uitgebreider
in sectie \ref{sec:falsificatie}.

Ook aan het kritisch-rationalisme kleven echter tenminste twee bezwaren.
Met het falsificatieprincipe kunnen waarnemingen (empirische feiten,
observaties, onderzoeksresultaten) gebruikt worden om theoretische
uitspraken te doen (met betrekking tot hypothesen). Strikt genomen moet
een veronderstelde algemene regel meteen verworpen worden na één
geslaagde falsificatie (een van de vogels in mijn steekproef fluit
niet): als theorie en observatie niet overeenstemmen, dan faalt de
theorie, volgens de kritisch-rationalisten. Maar om te komen tot een
observatie moet een onderzoeker vele keuzes maken (bijv.: hoe maak ik
een goede steekproef, wat is een vogel, hoe bepaal ik of een vogel
fluit?), die de geldigheid van de observaties onzeker kunnen maken. Er
kan dus ook iets mis zijn met de waarnemingen zelf (horen), of met de
operationalisaties van de gebruikte constructen (vogels, fluiten).

Een tweede probleem is dat er in de praktijk eigenlijk zeer weinig
theorieën zijn die werkelijk iets uitsluiten. Wanneer er discrepanties
waargenomen worden tussen theorie en observaties, dan wordt de theorie
bijgeschaafd, zodat de nieuwe observaties toch weer binnen de theorie
passen. Theorieën worden dan ook zelden volledig verworpen.

Een tweede paradigma is de kritische benadering. Het \emph{kritische paradigma}
onderscheidt zich van andere paradigmata in de nadruk op
maatschappelijke bepaaldheden; `de' werkelijkheid bestaat niet, ons
beeld ervan is een voorlopige, door maatschappelijke oorzaken bepaalde
werkelijkheid. Inzicht in de maatschappelijke verhoudingen heeft zelf
dus ook invloed op die werkelijkheid. Onze wetenschapsopvatting zoals
verwoord in bovengenoemde definities van onderzoek en theorie wordt in
het kritische paradigma dan ook afgewezen. Kritische onderzoekers menen
dat onderzoeksprocessen niet los gezien kunnen worden van de
maatschappelijke context waarin het onderzoek is verricht. Deze laatste
visie wordt overigens overgenomen door steeds meer onderzoekers, ook
door hen die andere paradigmata aanhangen.

\hypertarget{sec:instrumentatie-onderzoek}{%
\section{Instrumentatie-onderzoek}\label{sec:instrumentatie-onderzoek}}

Onderzoek is, zoals gezegd, een gesystematiseerde en gecontroleerde
wijze om empirische gegevens te verzamelen en te interpreteren.
Onderzoekers streven naar inzicht in natuurlijke verschijnselen, en in
de wijze waarop (de constructen van) die verschijnselen met elkaar
samenhangen. Een voorwaarde hiervoor is dat de onderzoeker deze
verschijnselen daadwerkelijk kan meten, d.i. uitdrukken in een
observatie (bij voorkeur in de vorm van een getal).
Instrumentatieonderzoek is voornamelijk gericht op de constructie van
instrumenten of methoden om verschijnselen, gedrag, vaardigheden,
attitudes, etc. meetbaar te maken. De ontwikkeling van goede
meetinstrumenten is bepaald geen sinecure: het is ambachtelijk handwerk,
waarbij de constructeur vele valkuilen moet zien te vermijden. Het
meetbaar maken van verschijnselen, van gedrag of van constructen noemen
we de \emph{operationalisatie}. Een concrete leestoets is bijvoorbeeld op te
vatten als een operationalisatie van het abstracte construct
`leesvaardigheid'.

We kunnen een nuttig onderscheid maken tussen het abstracte theoretische
construct en het gemeten construct, ofwel een onderscheid tussen: het
begrip-zoals-bedoeld en het begrip-zoals-bepaald. Het is uiteraard de
bedoeling dat het begrip-zoals-bepaald (de toets, de vragenlijst, de
observatie) het begrip-zoals-bedoeld (het theoretische construct) zo
goed mogelijk benadert. Indien het theoretische construct goed wordt
benaderd, dan spreken we van een adequate of valide meting.

Bij operationalisatie van een begrip-zoals-bedoeld moeten talloze keuzen
ge-maakt worden. Zo moet het CITO (Centraal instituut voor
toetsontwikkeling) elk jaar tekstbegripstoetsen construeren om de
leesvaardigheid van eindexamenkandidaten te meten. Daarvoor moet
allereerst een tekst gekozen of geredigeerd worden. Deze tekst mag niet
te moeilijk, maar ook niet te makkelijk zijn voor de doelgroep. Voorts
mag het onderwerp van de tekst niet al te bekend zijn, omdat anders de
bij sommige leerlingen aanwezige algemene kennis kan interfereren met de
meningen en standpunten die in de tekst naar voren gebracht worden.
Vervolgens moeten de vragen zó ontworpen worden dat de verschillende
passages in de tekst aan bod komen. Ook moeten de vragen zó samengesteld
zijn dat het theoretische construct `leesvaardigheid' adequaat
geoperationaliseerd wordt. Tot slot moet ook nog rekening gehouden
worden met de examens uit voorafgaande jaren; het nieuwe examen mag
immers niet al te veel afwijken van oude examens.

Een construct moet dus op de juiste wijze geoperationaliseerd zijn, om
observaties te verkrijgen die niet alleen valide zijn (een goede
benadering van het abstracte construct, zie Hoofdstuk \ref{ch:validiteit})
maar die ook betrouwbaar zijn (ongeveer
gelijke observaties bij herhaalde meting, zie
Hoofdstuk \ref{ch:betrouwbaarheid}).
In ieder onderzoek zijn de validiteit
en de betrouwbaarheid van een meting cruciaal; we besteden dan ook twee hoofdstukken aan deze begrippen.
Maar in instrumentatieonderzoek zijn
deze begrippen zelfs essentieel, omdat dit type onderzoek juist beoogt
om valide en betrouwbare instrumenten te leveren, die een goede
operationatisatie zijn van het abstracte construct-zoals-bedoeld.

\hypertarget{sec:beschrijvend-onderzoek}{%
\section{Beschrijvend onderzoek}\label{sec:beschrijvend-onderzoek}}

Met beschrijvend onderzoek bedoelen we onderzoek dat voornamelijk
gericht is op de beschrijving van een bepaald natuurlijk verschijnsel in
de werkelijkheid. De onderzoeker richt zich dus vooral op een
\emph{beschrijving} van het verschijnsel: het huidige vaardigheidsniveau, het
verloop van een proces of een discussie, de wijze waarop de lessen
Nederlands in het voortgezet onderwijs vorm worden gegeven, de politieke
voorkeur van stemmers vlak voor verkiezingen, de samenhang tussen het
aantal uren zelfstudie en het eindcijfer dat een student behaalt, etc.
Kortom, ook de onderwerpen voor beschrijvend onderzoek kunnen zeer
divers zijn.

\begin{center}\rule{0.5\linewidth}{0.5pt}\end{center}

\begin{quote}
\emph{Voorbeeld 1.1}: \citep{DTE13} hebben opnames van conversaties gekozen of gemaakt in 10 talen. Uit die opgenomen conversaties zijn woorden genomen waarmee een luisteraar om
``open verduidelijking'' vraagt: woordjes als \emph{hè} (Nederlands), \emph{huh}
(Engels), \emph{ã?} (Siwu). Van deze woorden is de klankvorm en het
toonhoogteverloop vastgesteld, met akoestische metingen en met
fonetische transcripties door experts. Een conclusie van dit
beschrijvende onderzoek luidt dat deze tussenvoegsels in de
verschillende talen veel meer op elkaar lijken (in klankvorm en
toonhoogteverloop) dan op grond van toeval te verwachten is.
\end{quote}

\begin{center}\rule{0.5\linewidth}{0.5pt}\end{center}

Dit voorbeeld illustreert dat beschrijvend onderzoek niet ophoudt als de
gegevens (klankvormen, toonhoogteverloop) beschreven zijn. Vaak zijn
\emph{verbanden} tussen de verzamelde gegevens ook zeer interessant (zie §\ref{sec:wetenschappelijk-onderzoek}).
Zo wordt in opiniepeilingen naar het stemgedrag bij verkiezingen vaak
een verband gelegd tussen het gepeilde stemgedrag enerzijds, en
leeftijd, geslacht en opleidingsniveau van de respondent anderzijds. En
evenzo wordt in onderwijskundig onderzoek een verband gelegd tussen
aantal uren studietijd enerzijds, en studiesucces van de respondent
anderzijds. Dit type van beschrijvend onderzoek, waarbij een correlatie
wordt vastgesteld tussen mogelijke oorzaken en mogelijke gevolgen, wordt
ook aangeduid als \emph{correlationeel onderzoek}.

Het essentiële verschil tussen beschrijvend en experimenteel onderzoek
is gelegen in de vraag naar oorzaak en gevolg. Op basis van beschrijvend
onderzoek kan een causaal verband tussen oorzaak en gevolg \emph{niet} goed
vastgesteld worden. Uit beschrijvend onderzoek zou kunnen blijken dat er
een samenhang is tussen een bepaald soort voeding en een langere
levensduur. Is het voedingspatroon dan ook de oorzaak van de langere
levensduur? Dat hoeft bepaald niet het geval te zijn: het is ook
mogelijk dat dat soort voeding vooral genuttigd wordt door mensen die
relatief hoog opgeleid en welvarend zijn, en door deze \emph{andere} factoren
ook relatief langer in leven zijn\footnote{Het is zelfs mogelijk dat het onderzochte voedingspatroon de
  oorzaak is van een relatief \emph{kortere} levensduur, maar dat dit
  negatieve effect gemaskeerd wordt door de sterkere, positieve
  effecten van opleidingsniveau en welvaartsniveau op de levensduur.}. Om vast te kunnen stellen of er
een causaal verband is, moeten we experimenteel onderzoek opzetten en
uitvoeren.

\hypertarget{sec:experimenteel-onderzoek}{%
\section{Experimenteel onderzoek}\label{sec:experimenteel-onderzoek}}

Experimenteel onderzoek wordt gekenmerkt doordat de onderzoeker een
bepaald aspect van de onderzoeksomstandigheden systematisch varieert
\citep{SCC02}. Het effect van deze manipulatie staat dan centraal in het
onderzoek. Een onderzoeker vermoedt bijvoorbeeld dat een bepaalde nieuwe
lesmethode zal resulteren in betere prestatie van de leerlingen dan de
huidige lesmethode. De onderzoeker wil deze hypothese toetsen door
middel van een experimenteel onderzoek. Hij manipuleert het type
onderwijs: sommige klassen of groepen krijgen les volgens de nieuwe
experimentele lesmethode en andere klassen of groepen krijgen les op de
traditionele wijze. Het effect van de nieuwe lesmethode wordt
geëvalueerd door de prestaties van de twee soorten schoolklassen te
vergelijken, na de `behandeling' met de oude vs.~nieuwe lesmethode.

Experimenteel onderzoek heeft als voordeel dat we de
onderzoeksresultaten doorgaans mogen interpreteren als het gevolg van de
experimentele manipulatie. Omdat de onderzoeker het onderzoek
systematisch controleert en slechts één aspect (i.c. de lesmethode)
varieert, kunnen eventuele verschillen tussen de prestaties van de twee
categorieën alleen toegeschreven worden aan het veranderde kenmerk, i.c.
aan de lesmethode. Dit veranderde kenmerk moet dan logischerwijs wel de
oorzaak zijn van de geobserveerde verschillen. Experimenteel onderzoek
is dus gericht op de evaluatie van causale verbanden.

Deze redenering vereist wel dat proefpersonen (of schoolklassen, in
bovenstaand voorbeeld) volgens het toeval, aselect (Eng. `at random'),
worden toegewezen aan de experimentele condities (i.c. de oude of nieuwe
lesmethode). Deze aselecte toewijzing (Eng. `random assignment') is de
beste methode om eventuele niet-relevante verschillen tussen de
behandelcondities uit te sluiten. Een dergelijk experiment met aselecte
toewijzing van proefpersonen aan condities wordt een \emph{gerandomiseerd
experiment} genoemd (Eng. `randomized experiment', `true experiment', ).
Om bij ons voorbeeld te blijven: als de onderzoeker de oude lesmethode
zou inzetten bij jongens, en de nieuwe lesmethode bij meisjes, dan is
een eventueel verschil in prestaties niet meer uitsluitend toe te
schrijven aan het gemanipuleerde kenmerk (de lesmethode), maar ook aan
een niet-gemanipuleerd maar wel relevant kenmerk, hier het geslacht van
de leerlingen. Zo'n mogelijk verstorend kenmerk wordt een \emph{storende
variabele} (Eng. `confound') genoemd. In
Hoofdstuk \ref{ch:ontwerp} bespreken we hoe we deze storende variabelen
kunnen neutraliseren, door random toewijzing van proefpersonen (of
schoolklassen) aan de experimentele condities, in combinatie met andere maatregelen.

Er is ook experimenteel onderzoek waarbij een bepaald aspect (zoals
lesmethode) wel systematisch varieert, maar waarbij de proefpersonen of
schoolklassen \emph{niet aselect} zijn toegewezen aan de experimentele
condities; dit wordt \emph{quasi-experimenteel} onderzoek genoemd \citep{SCC02}.
In het bovenstaande voorbeeld is daarvan sprake als de gebruikte
lesmethode onderzocht wordt, met gegevens van schoolklassen waarvan niet
de onderzoeker maar de docent bepaald heeft of de oude of nieuwe
lesmethode gebruikt wordt. Als de nieuwe lesmethode betere prestaties
zou opleveren, dan weten we \emph{niet} met zekerheid dat het verschil in
prestaties toe te schrijven is aan de lesmethode. Ook het enthousiasme
of de werkstijl van de docent kan een storende variabele zijn geweest in dit quasi-experiment. In dit tekstboek zullen we verschillende
voorbeelden van quasi-experimenteel onderzoek tegenkomen.

Binnen het type van experimenteel onderzoek kunnen we een verdere
verdeling aanbrengen, tussen laboratoriumonderzoek en veldonderzoek. In
beide typen experimenteel onderzoek wordt een aspect van de
werkelijkheid gemanipuleerd. Het verschil tussen beide typen onderzoek
is gelegen in de mate waarin de onderzoeker in staat is om allerlei
storende aspecten van de werkelijkheid onder controle te houden. In
laboratoriumonderzoek kan de onderzoeker zeer exact bepalen onder welke
omgevingscondities de observaties worden gedaan, en kan de onderzoeker
dus ook vele mogelijke storende variabelen onder controle houden (denk
aan verlichting, temperatuur, omgevingslawaai, etc.). In veldonderzoek
is dit niet het geval. De onderzoeker is `in het vrije veld' niet in
staat om alle (mogelijk relevante) aspecten van de werkelijkheid
volledig onder controle te houden.

\begin{center}\rule{0.5\linewidth}{0.5pt}\end{center}

\begin{quote}
\emph{Voorbeeld 1.2}: Margot van den Berg
onderzocht samen met collega's van de Universiteit van Ghana en de
Universiteit van Lomé hoe meertalige sprekers hun talen gebruiken als
zij eigenschappen zoals kleur, grootte en waarde moeten benoemen door
middel van een zogenaamde \emph{Director-Matcher task} \citep{BAEYT2017}. In deze
taak gaf de ene onderzoeksdeelnemer (de directeur) aanwijzingen aan een
ander (de uitvoerder) om een reeks voorwerpen in een bepaalde volgorde
neer te zetten. Zo konden in een kort tijdsbestek veel voorkomens van
eigenschapswoorden worden verzameld (`Zet de gele auto naast de rode
auto maar boven de kleine slipper'). De gesprekken werden opgenomen,
uitgeschreven en vervolgens onderzocht op taalkeuze, moment van
taalwisseling en type grammaticale constructie. Bij dergelijk veldwerk
kunnen echter allerlei niet-gecontroleerde aspecten in de omgeving van
invloed zijn op de geluidsopnames, en daarmee op de gegevens: ``kakelende kippen, een buurman die z'n motor aan het repareren is en 'm om de haverklap moet starten terwijl je een gesprek aan het opnemen bent,
keiharde regen op het aluminium dak van het gebouw waar de interviews
plaats vinden.'' (Margot van den Berg, pers.comm.)
\end{quote}

\begin{center}\rule{0.5\linewidth}{0.5pt}\end{center}

\begin{quote}
\emph{Voorbeeld 1.3}: Bij het luisteren naar gesproken zinnen kunnen we uit de oogbewegingen van een proefpersoon afleiden, hoe die gesproken zinnen worden verwerkt. In een zgn. `visual world'-taak krijgen luisteraars een zin te horen (bijv. ``Bert zegt dat het konijn is gegroeid''), terwijl ze kijken naar meerdere afbeeldingen op het scherm (meestal 4, bijv. een schelp, pauw, zaag, en wortel). Luisteraars blijken vooral te kijken naar de afbeelding die geassocieerd is aan het woord dat ze op dat moment mentaal verwerken: als ze het woord \emph{konijn} verwerken, dan kijken ze naar de wortel (exacter gezegd: ze kijken vaker en langer naar de wortel dan naar de andere afbeeldingen). Met een zgn. `eye tracker' kan worden vastgesteld naar welke positie van het scherm de proefpersoon kijkt (door observatie van de pupillen). De onderzoeker kan zo dus observeren welk woord op welk moment mentaal verwerkt wordt \citep{KMR12}. Dergelijk onderzoek kan het beste uitgevoerd worden in een laboratorium, met controle over achtergrondgeluiden, verlichting, en positie van de ogen t.o.v. computerscherm.
\end{quote}

\begin{center}\rule{0.5\linewidth}{0.5pt}\end{center}

Laboratoriumonderzoek en veldonderzoek hebben beide voordelen en
nadelen. Het grote voordeel van laboratoriumonderzoek is natuurlijk de
mate waarin de onderzoeker allerlei externe zaken onder controle kan
houden. In een laboratorium zal het experiment niet vaak verstoord
worden door een startende motor of door een regenbui. Dit voordeel van
laboratoriumonderzoek is echter ook een belangrijk nadeel, nl. dat het
onderzoek plaatsvindt in een min of meer kunstmatige omgeving. Het is
dan nog maar de vraag in hoeverre resultaten die onder kunstmatige
omstandigheden verkregen zijn, ook zullen gelden in het leven van
alledag buiten het laboratorium. Dit laatste is dan ook een punt in het
voordeel van veldonderzoek: het onderzoek wordt verricht onder
natuurlijke omstandigheden. Het nadeel van veldonderzoek is dan weer dat
er in het veld kan van alles gebeuren wat de onderzoeksresultaten
beïnvloedt, maar waar de onderzoeker geen controle over kan houden (zie het bovenstaande
voorbeeld). De keuze die een onderzoeker maakt tussen
beide typen experimenteel onderzoek wordt uiteraard sterk bepaald door
de vraagstelling van het onderzoek. Sommige vraagstellingen laten zich
beter in laboratoriumsituaties onderzoeken, terwijl andere beter in
veldsituaties onderzocht kunnen worden (zoals bovenstaande voorbeelden
illustreren).

\hypertarget{vooruitblik}{%
\section{Vooruitblik}\label{vooruitblik}}

Dit tekstboek bestaat uit drie delen. Deel I (hoofdstukken 1 tot en met 7) van dit tekstboek behandelt
methoden van onderzoek, en geeft een toelichting bij allerlei termen en
begrippen die van belang zijn bij het ontwerpen en opzetten van goed
wetenschappelijk onderzoek.

In deel II (hoofdstukken 8 tot en met 12) van het tekstboek behandelen we de beschrijvende statistiek
(Eng. `descriptive statistics') en in deel III (hoofdstukken 13 tot en met 17) behandelen we de
elementaire technieken uit de toetsende statistiek (Eng. `inferential
statistics'). Met deze laatste twee delen streven we drie doelen na.

Allereerst
willen we dat je in staat bent om artikelen en andere verslagen waarin
statistische verwerkings- en toetsingstechnieken zijn gebruikt, kritisch
te beoordelen.
Ten tweede willen we dat je dat je de noodzakelijke
kennis en inzicht hebt in de belangrijkste statistische procedures.
Ten derde willen we met deze statistische delen bereiken dat je in staat
bent om zelfstandig statistische bewerkingen uit te voeren voor je eigen
onderzoek, bijvoorbeeld voor je stage of eindwerkstuk.

Deze drie doelen zijn geordend in volgorde van belangrijkheid. Wij menen
dat een adequate en kritische interpretatie van statistische resultaten
en de conclusies die daaraan verbonden kunnen worden van groot belang is
voor alle studenten. Om die reden besteden we in dit tekstboek dan ook
relatief veel aandacht (in deel I) aan de `filosofie' of methodologie achter de besproken statistische technieken en analyses. Ook geven we
aan hoe je de besproken statistische analyses zelf kunt uitvoeren in
SPSS (een populair pakket voor statistische analyses) en in R (een wat
moeilijker, maar ook krachtiger en veelzijdiger pakket, met stijgende
populariteit). Beide statistische pakketten zijn geïnstalleerd in de computerleerzalen van de Faculteit Geesteswetenschappen. SPSS is beschikbaar via SurfSpot.nl voor een sterk gereduceerde prijs. R is vrijelijk beschikbaar via www.R-project.org. Meer achtergrond over het gebruik van R is te vinden via \url{https://hugoquene.github.io/emlar2020/}.

\hypertarget{ch:onderzoek}{%
\chapter{Hypothese-toetsend onderzoek}\label{ch:onderzoek}}

\hypertarget{inleiding}{%
\section{Inleiding}\label{inleiding}}

Veel empirisch onderzoek heeft tot doel om verbanden vast te stellen
tussen (vermeende) oorzaken en hun (vermeende) gevolgen. De onderzoeker
wil weten of de ene variabele van invloed is op de andere. Het onderzoek
toetst de hypothese dat er een verband is tussen de vermeende oorzaak en
het vermeende gevolg (zie Tabel \ref{tab:oorzaakgevolg}).
De beste methode om zo'n causaal verband vast
te stellen, en dus om de hypothese te toetsen, is het experiment. Een
goed opgezet en goed uitgevoerd experiment is de `gouden standaard' in
veel wetenschappelijke disciplines, omdat het goede waarborgen biedt
voor de validiteit van de conclusies (zie
Hoofdstuk \ref{ch:validiteit}.
Anders gezegd: de uitkomsten van een goed
experiment vormen de sterkst mogelijke evidentie voor een verband tussen
de onderzochte variabelen. Zoals besproken in
Hoofdstuk \ref{ch:inleiding} zijn er ook vele andere vormen van onderzoek,
en kunnen hypotheses ook op andere wijze en volgens andere paradigmata
onderzocht worden, maar we beperken ons hier tot experimenteel
onderzoek.

\begin{longtable}[]{@{}lll@{}}
\caption{\label{tab:oorzaakgevolg} Mogelijke oorzaken en mogelijke gevolgen.}\tabularnewline
\toprule
onderwerp & vermeende oorzaak & vermeend gevolg\tabularnewline
\midrule
\endfirsthead
\toprule
onderwerp & vermeende oorzaak & vermeend gevolg\tabularnewline
\midrule
\endhead
handel & buitentemperatuur & aantal verkochte ijsjes\tabularnewline
zorg & type behandeling & mate van herstel\tabularnewline
onderwijs & lesmethode & prestatie in toets\tabularnewline
taal & beginleeftijd van onderwijs & mate van taalbeheersing\tabularnewline
onderwijs & klassegrootte & schoolprestatie algemeen\tabularnewline
zorg & temperatuur & hoogte van malaria-gebieden\tabularnewline
taal & leeftijd & spreeksnelheid\tabularnewline
zorg & ligtijd voedsel op grond & mate van bacteriële besmetting\tabularnewline
\bottomrule
\end{longtable}

In experimenteel onderzoek wordt het effect onderzocht van een door de
onderzoeker gemanipuleerde variabele op een andere variabele. In de
inleiding is al een voorbeeld gegeven van een experimenteel onderzoek.
Een nieuwe lesmethode werd beproefd door leerlingen te verdelen over
twee groepen. De ene groep kreeg les volgens een nieuwe methode, terwijl
de andere groep het gebruikelijke onderwijs genoot. De onderzoeker
hoopte en verwachtte dat zijn nieuwe lesmethode een gunstig effect zou
hebben, d.w.z. dat het zou leiden tot betere prestaties.

In hypothese-toetsend onderzoek wordt nagegaan of de onderzochte
variabelen inderdaad met elkaar samenhangen op de verwachte wijze. In
deze definitie staan twee termen centraal: `variabelen' en `op de
verwachte wijze'. Voordat we nader ingaan op experimenteel onderzoek
zullen we deze termen nader beschouwen.

\hypertarget{sec:variabelen}{%
\section{Variabelen}\label{sec:variabelen}}

Wat is een variabele? Grofweg is een variabele een eigenschap van
objecten of personen die kan variëren, en die dus verschillende waarden
kan aannemen. Laten we twee eigenschappen van personen bekijken: het
aantal broers en zussen, en het geslacht van de moeder van die persoon.
De eerste eigenschap kan variëren tussen personen, en is dus een
variabele (tussen personen). De tweede eigenschap kan niet variëren: als
er een moeder is, dan is die altijd en per definitie van het vrouwelijke
geslacht. De tweede eigenschap is dus niet een variabele, maar een
constante eigenschap.

In onze wereld bestaat bijna alles in een variabele hoeveelheid of
hoedanigheid of mate. Ook een eigenschap die lastig te definiëren is,
zoals de populariteit van een persoon in een groep, kan een variabele
vormen. We kunnen immers personen in een groep rangschikken van meer tot
minder populair. Voorbeelden van variabelen zijn er te over:

\begin{itemize}
\item
  van \emph{personen}: hun lengte, hun gewicht, schoenmaat, spreeksnelheid,
  aantal broers en zussen, aantal kinderen, politieke voorkeur,
  inkomen, geslacht, populariteit in een groep, enz.
\item
  van \emph{teksten}: het totaal aantal woorden (`tokens'), aantal
  verschillende woorden (`types'), aantal spelfouten, aantal zinnen,
  aantal leestekens, enz.
\item
  van \emph{woorden}: de gebruiksfrequentie, aantal lettergrepen, aantal
  klanken, grammaticale woordsoort, enz.
\item
  van \emph{objecten} zoals auto's, telefoons, enz.: het gewicht, aantal
  componenten, energieverbruik, kostprijs, enz.
\item
  van \emph{organisaties}: het aantal werknemers, postcode, omzet, aantal
  burgers of klanten of patiënten of leerlingen, aantal operaties of
  diploma's of transacties, rechtsvorm, enz.
\end{itemize}

\hypertarget{sec:onafhankelijkeafhankelijkevariabelen}{%
\section{Onafhankelijke en afhankelijke variabelen}\label{sec:onafhankelijkeafhankelijkevariabelen}}

In hypothese-toetsend onderzoek kennen we twee soorten variabelen: de
afhankelijke en de onafhankelijke variabele. De \emph{onafhankelijke
variabele} is dat wat het veronderstelde effect teweeg moet brengen. De
onafhankelijke variabele is het aspect dat in een onderzoek door de
onderzoeker gemanipuleerd wordt. In het voorbeeld waar een experiment
uitgevoerd wordt om het effect van een nieuwe lesmethode te evalueren,
vormt die lesmethode de onafhankelijke variabele. Wanneer de prestaties
van de leerlingen die de nieuwe lesmethode gevolgd hebben vergeleken
worden met de prestaties van leerlingen die alleen traditioneel
schrijfonderwijs gevolgd hebben, dan neemt de onafhankelijke variabele
twee waarden aan. Deze twee waarden (ook wel \emph{niveau's} genoemd) van de
onafhankelijke variabele kunnen we in dit voorbeeld benoemen als
``experimenteel'' en ``controle'', of als ``nieuw'' en ``oud''. We zouden de
waarden van de onafhankelijke variabele ook kunnen uitdrukken als een
getal, 1 resp. 0. Deze getallen hebben geen numerieke betekenis (we
zouden de waarden ook 17 resp. 23 kunnen noemen), maar worden hier enkel
gebruikt als willekeurige etiketten om verschillende groepen te
onderscheiden. De gemanipuleerde variabele wordt `onafhankelijk' genoemd
omdat de gekozen (gemanipuleerde) waarden van deze variabele in een
onderzoek niet afhankelijk zijn van iets anders: de onderzoeker is
onafhankelijk in zijn of haar keuze van de gekozen waarden. Een
onafhankelijke variabele wordt ook wel \emph{factor} of soms \emph{predictor}
genoemd.

Het tweede type variabele is de \emph{afhankelijke variabele}. De
afhankelijke variabele is de variabele waarvoor we het veronderstelde
effect verwachten. De onafhankelijke variabele veroorzaakt dus
mogelijkerwijs een effect op de afhankelijke variabele, of: men
veronderstelt dat de waarde van de afhankelijke variabele afhankelijk is
van de waarde van de onafhankelijke variabele --- vandaar hun
benamingen. De afhankelijke variabele is dus datgene wat we meten of
observeren. Een geobserveerde waarde van de afhankelijke variabele wordt
ook wel \emph{responsie} of \emph{score} genoemd; ook de afhankelijke variabele
zelf wordt vaak zo aangeduid. In het voorbeeld waar een experiment
uitgevoerd wordt om het effect van een nieuwe lesmethode op de
prestaties van leerlingen te evalueren, vormen die prestaties van de
leerlingen de afhankelijke variabele. Andere voorbeelden zijn de
spreeksnelheid, of de score op een vragenlijst, of het aantal malen dat
een product verkocht wordt (zie Tabel \ref{tab:oorzaakgevolg}).
Kortom, in principe kan elke variabele
als afhankelijke variabele gebruikt worden. Het is voornamelijk de
vraagstelling die bepaalt welke afhankelijke variabele gekozen wordt, en
hoe deze gemeten wordt.

De onafhankelijke en afhankelijke variabelen dienen we overigens
nadrukkelijk \emph{niet} te interpreteren als `oorzaak' resp. `gevolg'. Het
doel van het onderzoek is immers om overtuigend aan te tonen dàt er een
(causaal) verband bestaat tussen de onafhankelijke en de afhankelijke
variabele. In
Hoofdstuk \ref{ch:validiteit} zullen we echter zien hoe complex dat is.

De onderzoeker varieert de onafhankelijke variabele en observeert of dit
resulteert in verschillen in de afhankelijke variabele. Als de waarden
van de afhankelijke variabele verschillen voor en na de manipulatie van
de onafhankelijke variabele, dan nemen we aan dat dit een gevolg is van
de manipulatie van de onafhankelijke variabele. Er is sprake van een
relatie tussen beide variabelen. Als de waarde van de afhankelijke
variabele niet verschilt onder invloed van de waarden van de
onafhankelijke variabele, dan is er geen verband tussen beide
variabelen.

\begin{center}\rule{0.5\linewidth}{0.5pt}\end{center}

\begin{quote}
\emph{Voorbeeld 2.1}: \citep{QSF12} onderzochten of een
glimlach of frons invloed heeft op hoe luisteraars gesproken woorden
verwerken. De woorden werden door de computer uitgesproken
(gesynthetiseerd) in verschillende fonetische varianten, en wel op zo'n manier dat die woorden klonken alsof ze neutraal, of met een glimlach,
of met een frons waren uitgesproken. Luisteraars moesten de woorden zo
snel mogelijk classificeren als `positief' danwel `negatief' (qua
betekenis). In dit onderzoek vormt de fonetische variant (neutraal,
glimlach, frons) de onafhankelijke variabele, en de snelheid waarmee de
luisteraars oordelen vormt de afhankelijke variabele.
\end{quote}

\begin{center}\rule{0.5\linewidth}{0.5pt}\end{center}

\hypertarget{sec:falsificatie}{%
\section{Falsificatie en nul-hypothese}\label{sec:falsificatie}}

Het doel van wetenschappelijk onderzoek is om te komen tot een coherente
verzameling van ``justified true beliefs'' \citep{Mort03}. Een
wetenschappelijke overtuiging moet dus deugdelijk onderbouwd en
gerechtvaardigd zijn (en coherent met andere overtuigingen). Hoe komen
we tot zo'n goede onderbouwing en rechtvaardiging? Daarvoor moeten we
eerst terug naar het zgn. inductie-probleem van \citep{Hume1739}. Hume constateerde dat
het logisch onmogelijk is om een bewering te generaliseren van een
aantal specifieke gevallen (de waarnemingen in een onderzoek) naar een
algemene regel (alle mogelijke waarnemingen in het universum).

Het probleem met deze generalisatie of inductie zullen we illustreren
met de overtuiging `alle zwanen zijn wit'. Als ik 10 zwanen heb gezien
die allemaal wit zijn, dan zou ik dat kunnen beschouwen als een
onderbouwing voor deze overtuiging. Deze generalisatie zou echter ook
onterecht kunnen zijn: misschien bestaan er ook niet-witte zwanen, al
heb ik die niet gezien. Meer algemeen: de inductie van specifieke
waarnemingen naar een generalisatie houdt altijd een risico in, en kan
niet gedaan worden ``met behoud van waarheid''. Er zit dus altijd een
logische `sprong' in, waardoor de generalisatie niet zonder risico is.
Een regel die wel opgaat voor alle waargenomen specifieke gevallen
(`alle zwanen zijn wit') hoeft daarmee nog niet een algemene regel te
zijn. Hetzelfde inductie-probleem blijft bestaan als ik 100 of 1000
witte zwanen heb gezien. Maar wat als ik één zwarte zwaan heb gezien?
Dan weet ik meteen, met zekerheid, dat de overtuiging dat alle zwanen
wit zijn, niet waar is. Dit principe gebruiken we ook in
wetenschappelijk onderzoek.

Laten we terugkeren naar ons eerdere voorbeeld waarin we hebben
verondersteld dat een nieuwe lesmethode beter is dan een oude
lesmethode; deze overtuiging noemen we H1. Laten we deze redenering nu
eens omdraaien, en ons baseren op de complementaire overtuiging\footnote{Twee beweringen zijn complementair als ze elkaar wederzijds
  uitsluiten, zoals H1 en H0 in dit voorbeeld.} dat
de nieuwe methode \emph{niet} beter is dan de oude; deze overtuiging noemen
we de nul-hypothese of H0. Deze overtuiging H0 `alle methoden hebben
gelijk effect' is analoog aan de overtuiging `alle zwanen zijn wit' uit
het voorbeeld in de vorige alinea. Hoe moeten we nu toetsen of de
overtuiging of hypothese H0 waar is? Laten we daarvoor een
representatieve steekproef van leerlingen trekken (zie
Hoofdstuk \ref{ch:steekproeftrekking}), en laten we de leerlingen volgens
het toeval toewijzen aan de nieuwe of oude lesmethode (waarden van
onafhankelijke variabele); we observeren vervolgens alle prestaties
(afhankelijke variabele) van alle deelnemende leerlingen, volgens
hetzelfde protocol voor alle gevallen. Vooralsnog veronderstellen we dat
H0 waar is. We verwachten dus ook geen verschil tussen de prestaties van
de verschillende groepen leerlingen. Als de leerlingen van de nieuwe
methode desalniettemin veel beter blijken te presteren dan de leerlingen
van de oude methode, dan vormt dat waargenomen verschil de figuurlijke
zwarte zwaan: het gevonden verschil (dat in tegenspraak is met H0) maakt
het onwaarschijnlijk dat H0 waar is (\emph{mits} het onderzoek valide was;
meer daarover in het volgende hoofdstuk). Omdat H0 en H1 elkaar
uitsluiten, is het dan dus ook erg waarschijnlijk dat H1 wèl waar is. En
omdat we onze onderbouwing baseerden op H0 en niet op H1, kunnen
sceptici ons niet van partijdigheid beschuldigen: we probeerden immers
juist aan te tonen dat er géén verschil was tussen de prestaties van de
leerlingen uit de twee groepen.

Deze methode wordt \emph{falsificatie} genoemd, omdat we kennis verwerven
door hypothesen te verwerpen (falsifiëren) en niet door hypothesen te
aanvaarden (verifiëren). Deze methodologie is ontwikkeld door de
wetenschapsfilosoof Karl Popper \citep{Popp35, Popp59, Popp63}. De
falsificatie-methode heeft interessante overeenkomsten met de
evolutietheorie. Door variatie tussen de individuen kunnen sommigen zich
succesvol voortplanten, terwijl veel anderen voortijdig sterven en/of
zich niet voortplanten. Op analoge wijze kunnen sommige tentatieve
beweringen niet weerlegd worden, en kunnen deze dus `overleven' en `zich
voortplanten', terwijl veel andere beweringen weerlegd worden en dus
`sterven'. In de woorden van \citep[p.51]{Popp63}:

\begin{quote}
" \ldots{} to explain (the world) \ldots{} as far as possible, with the help of laws and explanatory theories \ldots there is no more rational procedure than the method of trial and error --- of conjecture and refutation: of boldly proposing theories; of trying our best to show that these are erroneous; and of accepting them tentatively if our critical efforts are unsuccessful."
\end{quote}

Een goede wetenschappelijke bewering of theorie dient dus falsifieerbaar
of weerlegbaar of toetsbaar te zijn \citep{Popp63}, d.w.z. het moet mogelijk
zijn om de onjuistheid van die bewering of theorie aan te tonen. De
wetenschappelijke onderbouwing en daarmee de plausibiliteit van een
toetsbare bewering neemt toe, naarmate die bewering vaker en onder meer
wisselende omstandigheden bestand is gebleken tegen falsificatie. `Het
klimaat wordt warmer' is een goed voorbeeld van een bewering die steeds
beter bestand blijkt te zijn tegen falsificatie, en die daarmee steeds
sterker wordt.

\begin{center}\rule{0.5\linewidth}{0.5pt}\end{center}

\begin{quote}
\emph{Voorbeeld 2.2:} `Alle zwanen zijn wit' en
`de gemiddelde temperatuur van de aarde stijgt sinds 1900'
zijn falsifieerbare, en daarom wetenschappelijk bruikbare
beweringen. Maar hoe zit dat met de volgende beweringen?\\
a. Goud lost op in water.\\
b. Zout lost op in water.\\
c.~Vrouwen praten meer dan mannen.\\
d.~De muziek van Coldplay is beter dan die van U2.\\
e. De muziek van Coldplay verkoopt beter dan die van U2.\\
f.~Als een patiënt een duiding van de psychoanalyticus afwijst, dan is dat het gevolg van weerstand omdat de duiding van de
psychoanalyticus juist is.\\
g. De stijging van de gemiddelde temperatuur van de aarde is het gevolg van menselijke activiteiten.
\end{quote}

\begin{center}\rule{0.5\linewidth}{0.5pt}\end{center}

\hypertarget{sec:empirischecyclus}{%
\section{De empirische cyclus}\label{sec:empirischecyclus}}

In het voorafgaande hebben we op een vrij
globale manier kennis gemaakt met experimenteel onderzoek. In deze
paragraaf beschrijven we het verloop van experimenteel onderzoek meer
systematisch. Er zijn in de loop der tijd verschillende schema's
opgesteld waarin onderzoek in fasen beschreven wordt. De bekendste van
deze schema's is waarschijnlijk wel de empirische cyclus van \citep{Groot61}.

In de empirische cyclus worden vijf onderzoeksfasen onderscheiden: de
observatiefase, de inductiefase, de deductiefase, de toetsingsfase en de
evaluatiefase. In de laatste fase worden tekortkomingen en alternatieve
interpretaties geformuleerd. Dit leidt weer tot nieuw onderzoek, waarin
opnieuw de serie fasen kan worden doorlopen (vandaar `cyclus'). Deze
vijf onderzoeksfasen zullen wij één voor één behandelen.

\hypertarget{observatie}{%
\subsection{observatie}\label{observatie}}

In deze fase construeert de onderzoeker een probleem. Dat wil zeggen dat
de onderzoeker een idee vorm over de mogelijke relaties tussen
verschillende (theoretische) concepten of constructen. Deze
veronderstellingen worden later uitgewerkt tot meer algemene hypothesen.
Veronderstellingen kunnen op duizenden manieren tot stand komen --- maar
vereisen altijd nieuwsgierigheid van de onderzoeker. De onderzoeker kan
een vreemd fenomeen opmerken dat verklaard moet worden, bv het fenomeen
dat het vermogen om absolute toonhoogte te horen (``absoluut gehoor'')
veel vaker voorkomt bij Chinezen dan bij Amerikanen \citep{Deut06}. Ook het
systematisch doorzoeken van wetenschappelijke publicaties kan leiden tot
veronderstellingen. Soms blijkt dan dat de resultaten van verschillende
onderzoeken elkaar tegenspreken, of dat er een duidelijke lacune zit in
onze kennis.

Veronderstellingen kunnen ook gebaseerd zijn op case-studies:
onderzoeken waarbij één of enkele gevallen intensief bestudeerd en
extensief beschreven worden. Zo ontwikkelde Piaget zijn theorie over de
verstandelijke ontwikkeling van kinderen op basis van observaties van
zijn eigen kinderen in de tijd dat hij werkloos was. Deze observaties
vormden later, toen Piaget zijn eigen laboratorium had, aanleiding voor
vele experimenten op basis waarvan hij zijn theoretische inzichten kon
verdiepen en verifiëren.

Het is belangrijk om te beseffen dat puur onbevangen, objectieve
waarneming niet mogelijk is. Waarnemingen zijn altijd min of meer
theorie-geladen of kennis-geladen. Als we niet weten waarop we moeten
letten, kunnen we ook niet goed waarnemen. Zo kunnen wolken-experts veel
meer typen van bewolking onderscheiden en interpreteren dan leken.
Voordat er observaties gedaan worden en feiten worden geanalyseerd, is
het dus verstandig om eerst een expliciet theoretisch kader aan te
brengen, ook al is dit nog rudimentair.

Een onderzoeker komt tot veronderstellingen naar aanleiding van
opmerkelijke verschijnselen, case-studies, literatuurstudie, e.d. Er
zijn echter geen methodologische richtlijnen over hoe dit proces zou
moeten verlopen: het is een creatief proces.

\hypertarget{inductie}{%
\subsection{inductie}\label{inductie}}

In de inductiefase wordt de in de observatiefase geopperde
veronderstelling gegeneraliseerd. Op grond van specifieke observaties
wordt nu een hypothese geopperd waarvan de onderzoeker vermoedt dat die
algemeen geldig is. (\textbf{Inductie} is de logische stap waarbij een algemene
bewering of hypothese wordt afgeleid uit specifieke gevallen: mijn
kinderen (hebben) leren praten \(\rightarrow\) alle kinderen (kunnen)
leren praten.)

Zo kan een onderzoeker uit de observatie dat de vrouwen in zijn/haar
omgeving meer praten dan de mannen (meer minuten per etmaal, en meer
woorden per etmaal), een algemene hypothese afleiden: H1: vrouwen praten
meer dan mannen (zie Voorbeeld 2.2); deze hypothese kan nader ingeperkt
worden in tijd en plaats.

De hypothese moet tevens een duidelijk omschreven empirische inhoud
hebben, d.w.z. het type of de klasse van observaties moet goed
omschreven zijn. Gaat het over alle vrouwen en mannen? Of alleen
sprekers van het Nederlands? En hoe zit het met meertalige sprekers? En
met kinderen die hun taal nog aan het leren zijn? Die duidelijk
omschreven inhoud is nodig om de hypothese te kunnen toetsen (zie subsectie Toetsing hieronder, en zie Hoofdstuk \ref{ch:toetsing}).

Tenslotte moet een hypothese ook logisch coherent zijn, d.w.z. de
hypothese moet aansluiten bij andere theorieën of hypothesen. Als een
hypothese niet logisch coherent is, dan kan zij per definitie niet
eenduidig aan de empirie gerelateerd worden, en is zij dus niet goed
toetsbaar. Hieruit volgt dat een hypothese niet multi-interpretabel mag
zijn: een hypothese moet op zichzelf één en niet meer dan één uitkomst
van een experiment voorspellen.

In het algemeen worden drie typen hypothesen onderscheiden \citep{Groot61}:

\begin{itemize}
\item
  Universeel-deterministische hypothesen.\\
  Deze hebben als algemene vorm: \emph{alle A's zijn B's}. Bijvoorbeeld: alle
  zwanen zijn wit, alle (volwassen) mensen kunnen spreken. Als een
  onderzoeker voor één A kan aantonen dat deze niet B is, dan is de
  hypothese in beginsel gefalsifieerd. Een universeel deterministische
  hypothese kan nooit geverifieerd worden; een onderzoeker kan alleen
  een uitspraak doen over de gevallen die hij geobserveerd, dan wel
  gemeten heeft. Bij een oneindige verzameling, zoals: alle vogels, of
  alle mensen, of alle kachels, kan dit tot problemen leiden. De
  onderzoeker weet niet of er misschien één enkel geval bestaat waarin
  geldt: A is niet B; er is één vogel die niet kan vliegen, et cetera.
  Over deze andere gevallen kan dus geen uitspraak gedaan worden,
  waardoor de universele geldigheid van de hypothese nooit volledig
  `bewezen' kan worden.
\item
  Deterministische existentiehypothesen.\\
  Deze hebben als algemene vorm: \emph{er is tenminste één A die B is}.
  Bijvoorbeeld: er is tenminste één zwaan die wit is, er is tenminste
  één mens die kan praten, er is tenminste één kachel die warmte
  geeft. Als een onderzoeker kan aantonen dat er één A is die B is,
  dan is de hypothese geverifieerd. Deterministische
  existentiehypothesen kunnen echter nooit gefalsifieerd worden.
  Daarvoor zou het nodig zijn om van een oneindige verzameling alle
  eenheden of individuen te onderzoeken op het al dan niet B zijn, en
  dat is door de oneindigheid van de verzameling nu juist uitgesloten.
  Hieruit blijkt tegelijk dat dit type hypothesen geen algemene
  uitspraken doen, en dat het wetenschappelijk belang ervan niet zo
  duidelijk is. Je kunt het ook zo zeggen: voor elk specifiek geval A
  doet een dergelijke hypothese helemaal geen duidelijke voorspelling;
  een gegeven A zou de gezochte B kunnen zijn, maar dat hoeft helemaal
  niet. In deze zin voldoet een deterministische existentiehypothese
  dan ook niet aan ons criterium van falsificatie.
\item
  Probabilistische hypothesen.\\
  Deze hebben als algemene vorm: \emph{er zijn relatief meer A's die B zijn,
  dan niet-A's die B zijn}.
  In de gedragswetenschappen (inclusief taal
  en communicatie) is dit verreweg het meest voorkomende type
  hypothese.\\
  Bijvoorbeeld: er zijn relatief meer vrouwen die veelpratend zijn dan
  mannen die veelpratend zijn. Of: er zijn relatief meer
  hoog-presterende leerlingen bij de nieuwe methode dan bij de oude
  methode. Of: versprekingen treden relatief vaker op bij het begin
  dan bij het einde van een woord. Daarmee wordt nog niet aangegeven
  dat alle vrouwen meer praten dan alle mannen, en evenmin wordt
  aangegeven dat alle leerlingen met de nieuwe methode beter presteren
  dan alle leerlingen van de oude methode.
\end{itemize}

\hypertarget{deductie}{%
\subsection{deductie}\label{deductie}}

In deze fase van de empirische cyclus worden specifieke voorspellingen
afgeleid uit de algemeen geformuleerde hypothese die is opgezet in de
inductiefase. (\textbf{Deductie} is de logische stap waarbij een specifieke
bewering of voorspelling wordt afgeleid uit een meer algemene bewering:
alle kinderen leren praten \(\rightarrow\) mijn kinderen (zullen) leren
praten.)

Laten we veronderstellen (H1) dat ``vrouwen meer praten dan mannen''. Uit
deze hypothese doen we in deze fase specifieke voorspellingen voor
specifieke steekproeven. Wanneer we bijvoorbeeld 40 vrouwelijke en 40
mannelijke docenten Nederlands zouden interviewen, zonder
tijdsbeperking, dan luidt de voorspelling op grond van deze H1 dat de
vrouwelijke docenten in deze steekproef meer zullen zeggen dan de
mannelijke docenten in de steekproef (en dus ook, dat ze een groter
aantal lettergrepen zullen spreken in het interview).

Zoals hierboven uitgelegd (§\ref{sec:falsificatie}), wordt in het meeste wetenschappelijk
onderzoek echter niet de H1 getoetst, maar de logische tegenhanger
daarvan, die met H0 wordt aangeduid.
Voor de toetsing (in de volgende fase van de empirische
cyclus) is het dus gebruikelijk om voorspellingen te toetsen die zijn
afgeleid uit de H0 (!), bijvoorbeeld ``vrouwen en mannen produceren \emph{even veel} lettergrepen in een vergelijkbaar interview''.

In de praktijk worden de termen `hypothese' en `voorspelling' vaak door
elkaar gebruikt, en spreken we vaak over het toetsen van hypothesen.
Volgens bovenstaande terminologie toetsen we echter niet de hypothesen, maar leiden we uit de hypothesen voorspellingen af (via deductie), en toetsen we daarna die voorspellingen aan de data.

\hypertarget{toetsing}{%
\subsection{toetsing}\label{toetsing}}

In deze fase verzamelen we empirische observaties en vergelijken we die
met de uitgewerkte voorspellingen ``onder H0'', d.w.z. de voorspellingen
als H0 waar zou zijn. In Hoofdstuk \ref{ch:toetsing} zullen we nader ingaan op deze toetsing. Hier introduceren we alleen het algemene principe om nulhypotheses te toetsen. (Naast het hier beschreven conventionele ``frequentistische'' principe kunnen we ook hypotheses toetsen of vergelijken op een nieuwere ``Bayesiaanse'' wijze; we bespreken die in §\ref{sec:bayesiaans}).

Als de observaties buitengewoon onwaarschijnlijk zijn onder H0, dan zijn
er twee logische mogelijkheden. (i) De observaties deugen niet, we
hebben fout geobserveerd. Maar als de onderzoeker zijn werk goed
gecontroleerd heeft en zichzelf serieus neemt, dan is dat niet
waarschijnlijk. (ii) De voorspelling was onjuist, H0 is wellicht niet
juist, en moet dus verworpen worden, ten gunste van H1.

In ons voorbeeld hierboven (in de voorgaande subsectie over deductie) hebben we uit H0 (!) de voorspelling afgeleid
dat in een steekproef van 40 mannelijke en 40 vrouwelijke docenten, de
leden van de twee groepen even veel lettergrepen gebruiken in een
gestandaardiseerd interview. We vinden echter dat de mannen meer lettergrepen gebruiken (gemiddeld 4210 lettergrepen) dan de vrouwen (gemiddeld 3926 lettergrepen) \citep[p.1112]{Quene08}.
Hoe
waarschijnlijk is dit verschil als de observaties kloppen, en als H0
waar zou zijn? Die kans is zodanig klein dat de onderzoeker H0 verwerpt
(zie optie (ii) hierboven), en concludeert dat vrouwen en mannen \emph{niet even veel} praten, althans in dit onderzoek.

In het bovenstaande voorbeeld worden in de toetsingsfase twee groepen
vergeleken, hier mannen en vrouwen. Eén van die twee groepen is vaak een
neutrale groep of controle-groep, zoals we al zagen in het eerdere
voorbeeld van de nieuwe en oude lesmethode. Waarom maken onderzoekers
vaak gebruik van zo'n controle-groep? Stel je eens voor dat we alleen de
nieuwe-methode-groep zouden onderzoeken. In de toetsingsfase meten we de
prestaties van de leerlingen, en die is ruim voldoende: gemiddeld een 7.
Is de nieuwe methode dan een succes? Misschien niet: als de leerlingen
volgens de oude methode een 8 zouden behalen, dan zou de nieuwe methode
eigenlijk slechter zijn, en zouden we de nieuwe methode beter niet
kunnen invoeren. Om daar een zinnige conclusie over te kunnen trekken,
is het essentieel om de nieuwe en oude methoden onderling te
vergelijken. Vandaar dat in veel onderzoek een neutrale conditie,
nul-conditie, controle-groep, placebo-behandeling, o.i.d, is opgenomen.

Hoe kunnen we nu de kans bepalen op de gevonden observaties, als H0 waar
zou zijn? Dat is vaak wat complex, maar we illustreren het hier met een
eenvoudig voorbeeld. We gooien kop of munt met een munt. We
veronderstellen (H0): de munt is eerlijk, de kans op kop is \(1/2\) per
worp. We gooien \(10\times\) met dezelfde munt, en wonderbaarlijk genoeg
observeren we alle \(10\times\) een kop als uitkomst. De kans dat dit
gebeurt, als H0 waar is, is \(P = (1/2)^{10} = 1/1024\). Als H0 waar zou
zijn is deze uitkomst uiterst onwaarschijnlijk (al is de uitkomst niet
onmogelijk, want \(P > 0\)), en daarom verwerpen we H0. We concluderen dus
dat de munt waarschijnlijk niet eerlijk is.

Dit roept een belangrijk punt op: wanneer is een uitkomst zò
onwaarschijnlijk dat we H0 verwerpen? Welk criterium hanteren we voor de
kans op de gevonden observaties als H0 waar zou zijn? Dit is de vraag
naar het significantieniveau, d.w.z. het kansniveau waarbij we besluiten
de H0 te verwerpen. Dit wordt aangeduid met symbool \(\alpha\). Als in een
onderzoek een significantieniveau gehanteerd wordt van \(\alpha = 0.05\),
dan wordt de H0 verworpen als de kans om deze resultaten te vinden als
H0 waar is\footnote{Vollediger: Als de kans om deze resultaten te vinden, of
  resultaten die nog meer verschillen van de door H0 voorspelde
  resultaten, kleiner is dan 5\%, dan wordt H0 verworpen.}, kleiner is dan 5\%. De uitkomst is dan zo
onwaarschijnlijk, dat we ervoor kiezen om H0 te verwerpen (optie (ii)
hierboven), d.w.z. we concluderen dat H0 waarschijnlijk \emph{niet} waar is.

Als we H0 aldus verwerpen, dan lopen we wel een kleine kans dat we
eigenlijk met optie (i) te maken hebben: H0 is waar, maar de observaties
wijken \emph{toevallig} sterk af van de voorspelling op basis van H0, en H0
wordt dan ten onrechte verworpen. Dit wordt een Type-I-fout genoemd.
Deze fout is vergelijkbaar met de onterechte veroordeling van een
onschuldig persoon, of met de onjuiste classificatie van een onschuldig
email-bericht als `spam'. Meestal wordt \(\alpha = .05\) gebruikt, maar
ook andere significantie-niveau's zijn mogelijk en soms verstandiger.

Merk op dat de significantie betrekking heeft op de kans om de gevonden
extreme gegevens (of meer extreme gegevens) te vinden, indien H0 waar
is: \[\textrm{significantie} = P(\textrm{data}|\textrm{H0})\] De
significantie is dus \emph{niet} de kans dat H0 waar is als je deze gegevens
gevonden hebt, \(P(\textrm{H0}|\textrm{data})\), hoewel we deze denkfout
vaak tegenkomen.

Bij iedere vorm van toetsing is er ook een kans op de omgekeerde fout,
nl. dat we H0 ten onrechte niet verwerpen. Dat wordt een Type-II-fout
genoemd: H0 is eigenlijk niet waar (dus H1 is waar) maar H0 wordt
desalniettemin niet verworpen. Deze fout is vergelijkbaar met de
onterechte vrijspraak van een schuldig persoon, of met het onterecht
goedkeuren van een \emph{spam} email-bericht (zie
Tabel \ref{tab:H0H1uitkomsten}).

\begin{longtable}[]{@{}lll@{}}
\caption{\label{tab:H0H1uitkomsten} Mogelijke uitkomsten van beslissingsprocedure.}\tabularnewline
\toprule
werkelijkheid & &\tabularnewline
\midrule
\endfirsthead
\toprule
werkelijkheid & &\tabularnewline
\midrule
\endhead
& \textbf{H0 verworpen} & \textbf{H0 niet verworpen}\tabularnewline
H0 is waar (H1 onwaar) & Type-I-fout (\(\alpha\)) & correct\tabularnewline
H0 is onwaar (H1 waar) & correct & Type-II-fout (\(\beta\))\tabularnewline
& \textbf{verdachte veroordeeld} & \textbf{verdachte vrijgesproken}\tabularnewline
verdachte is onschuldig (H0) & Type-I-fout & correct\tabularnewline
verdachte is schuldig & correct & Type-II-fout\tabularnewline
& \textbf{bericht weggegooid} & \textbf{bericht doorgestuurd}\tabularnewline
bericht is OK (H0) & Type-I-fout & correct\tabularnewline
bericht is spam & correct & Type-II-fout\tabularnewline
\bottomrule
\end{longtable}

Als we het significantieniveau hoger instellen, bv. \(\alpha = .20\), dan
is de kans om de H0 te verwerpen dus ook veel groter. In de
toetsingsfase verwerpen we immers H0 al indien de kans op deze gegevens
(of meer extreme gegevens) kleiner is dan 20\%. Een uitkomst van
\(8\times\) kop in 10 worpen is dan al voldoende om H0 te verwerpen (d.i.
om de munt als onzuiver te beoordelen). Er zijn dus meer uitkomsten
mogelijk waarbij we H0 zullen verwerpen. Dat hogere significantieniveau
houdt dus een groter risico in op een Type-I-fout, en tegelijk een
kleiner risico op een Type-II-fout. De afweging tussen de twee typen
fouten hangt af van de precieze omstandigheden van het onderzoek, en van
de consequenties van de twee typen van fouten. Welke fout is ernstiger:
een goed bericht weggooien, of een \emph{spam}-bericht doorsturen? De kans op
een Type-I-fout (significantieniveau) heeft een onderzoeker in eigen
hand. De kans op een Type-II-fout is afhankelijk van drie factoren, en
is lastig te bepalen. We zullen dat nader bespreken in
Hoofdstuk \ref{ch:power}.

\hypertarget{evaluatie}{%
\subsection{evaluatie}\label{evaluatie}}

Aan het einde van het onderzoek moet de onderzoeker de
onderzoeksresultaten evalueren: wat is het nu allemaal waard? Het draait
hier niet slechts om de vraag of de onderzoeksresultaten al dan niet ten
gunste van de getoetste theorie uitgevallen zijn. Het gaat om een
kritische beschouwing van de wijze waarop de data zijn verzameld, de
denkstappen, de operationalisatie, de mogelijke alternatieve
verklaringen, alsmede de consequenties van de resultaten. De resultaten
moeten in een bredere context geplaatst en besproken worden. Wellicht
leiden de conclusies ook tot aanbevelingen, bijvoorbeeld voor klinische
toepassingen of voor de onderwijspraktijk. Dit is ook het moment om
suggesties voor ander of vervolgonderzoek te doen.

In deze fase gaat het primair om de interpretatie van de resultaten,
waarbij de onderzoeker als interpretator een belangrijke en persoonlijke
rol speelt. Verschillende onderzoekers kunnen dezelfde uitkomsten geheel
anders interpreteren. En soms zijn de resultaten in tegenspraak met wat
was voorspeld of gewenst.

\hypertarget{sec:keuzemomenten}{%
\section{Keuzemomenten}\label{sec:keuzemomenten}}

Onderzoek bestaat uit een reeks van keuze-momenten: van de inspirerende
observaties in de eerste fase, via de operationele beslissingen in de
uitvoering van het onderzoek, tot de interpretatie van de resultaten in
de laatste fase. Zelden zal een onderzoeker in staat zijn om altijd de
beste keuze te maken, maar hij of zij moet er voor waken dat ergens een
slechte keuze gemaakt zou worden. Het hele onderzoek is net zo sterk als
de zwakste schakel: de waarde van het hele onderzoek hangt af van de
slechtste keuze in de reeks van keuzes. Ter illustratie geven we een
beeld van de keuzes die een onderzoeker moet maken tijdens de gehele
empirische cyclus.

De eerste keuze die gemaakt moet worden betreft de probleemstelling.
Relevante vragen die de onderzoeker op dit moment moet beantwoorden
zijn: hoe herken ik een bepaalde onderzoeksvraag, is onderzoek hier het
aangewezen middel, is dit idee onderzoekbaar? De beantwoording van
dergelijke vragen is van allerlei factoren afhankelijk, zoals mens- en
maatschappijvisie, wensen van de opdrachtgever, financiële en praktische
mogelijkheden, enz.

De onderzoeksvraag moet wel te beantwoorden zijn met de beschikbare
methoden en middelen. Maar binnen die beperking kan de onderzoeksvraag
elk aspect van de werkelijkheid betreffen, ongeacht of dit aspect nu
irrelevant of belangrijk wordt geacht. Er zijn vele voorbeelden van
onderzoek dat aanvankelijk werd afgedaan als irrelevant, maar dat
desondanks wel degelijk van wetenschappelijke waarde bleek te zijn,
bijvoorbeeld een studie over de vraag ``is `Huh?' a universal word?''
\citep{DTE13} (Voorbeeld 1.1).
Ook bleken ideeën die eerst als onjuist werden afgedaan later toch te kloppen met de werkelijkheid. Zo beweerde Galilei, zogenaamd `ten onrechte', dat de aarde om de zon draaide. Kortom, onderzoeksvragen moeten niet te snel verworpen worden omdat zij `nutteloos', een `open deur', `irrelevant' of `triviaal' zouden zijn.

Als de onderzoeker besluit om verder te gaan met het onderzoek, dan is
de volgende stap doorgaans literatuurstudie. In de meeste handboeken
wordt aanbevolen veel te lezen, maar hoe wordt de literatuur verzameld?
Uiteraard moet de relevante onderzoeksliteratuur over het probleemgebied
doorgenomen worden. Gelukkig bestaan er tegenwoordig allerlei
hulpmiddelen om relevante wetenschappelijke publicaties te vinden. Het
is raadzaam om daarvoor de aanwijzingen en zgn. ``libguides'' te
bestuderen die de Universiteitsbibliotheek aanbiedt (zie
\url{http://www.uu.nl/bibliotheek}, en \url{http://libguides.library.uu.nl}).
Ook de gids van \citep{Sand11} bevelen we ten zeerste aan: de gids bevat vele uiterst
nuttige aanwijzingen over het opsporen van relevante
onderzoeksliteratuur.

In de fase daarna doemen de eerste methodologische problemen op. De
onderzoeker moet namelijk de probleemstelling exacter formuleren. Een
belangrijke afweging die hier gemaakt dient te worden is of de
probleemstelling wel onderzoekbaar is (§\ref{sec:falsificatie}).
Een vraag als ``wat is het effect van
beginleeftijd van leren op de taalvaardigheid in een vreemde taal?'' is
bijvoorbeeld niet zonder meer onderzoekbaar. Deze vraag moet nader
gespecificeerd worden. Cruciale concepten moeten ge(her)definieerd
worden: wat is de beginleeftijd van het leren van een vreemde taal? Wat is taalvaardigheid? Wat is een effect? En wat is eigenlijk een vreemde taal? Hoe definieer ik de populatie?
De onderzoeker wordt geconfronteerd
met allerlei vragen over definities en operationalisatie: Worden
begrippen theoretisch of empirisch of pragmatisch gedefinieerd? Welke
instrumenten worden gebruikt om de verschillende constructen te meten?
Maar ook: hoe ingewikkeld moet het onderzoek worden? Kan het hele
onderzoek dan wel tot een goed einde worden gebracht? Op welke wijze
moeten de gegevens verzameld worden? Kunnen de gewenste gegevens wel
verzameld worden, of zullen respondenten dergelijke vragen nooit
(kunnen) beantwoorden? Is de voorgestelde manipulatie ethisch
verantwoord? Wat is de afstand tussen het theoretische construct en de
wijze waarop dat zal worden gemeten? Wanneer in deze fase iets fout
gaat, dan heeft dat direct weerslag op de rest van het onderzoek.

Als er met succes een probleemstelling is geformuleerd en
geoperationaliseerd, dan volgt een nadere literatuurverkenning. Dit
tweede literatuuronderzoek is veel meer toegespitst op de inmiddels
uitgewerkte onderzoeksvraag dan de eerder genoemde, brede
literatuurverkenning. Op grond van eerdere publicaties kan de
onderzoeker zijn of haar oorspronkelijke probleemstelling heroverwegen.
Niet alleen moet de literatuur nu doorgenomen worden met het oog op
inhoudelijk theoretische overwegingen, maar ook moet aandacht worden
besteed aan voorbeelden van operationalisering van de kernbegrippen.
Zijn deze begrippen wel goed geoperationaliseerd, en als er
verschillende manieren van operationalisering zijn, wat is dan de ratio
achter deze verschillen? En, kunnen de kernbegrippen zo
geoperationaliseerd worden dat de afstand tussen het
begrip-zoals-bedoeld en het begrip-zoals-bepaald (nog) kleiner is
(§\ref{sec:instrumentatie-onderzoek})? De aanwijzingen hierboven
voor het zoeken in wetenschappelijke literatuur zijn hier wederom van
nut. De onderzoeker dient zich vervolgens (nogmaals) te beraden op het
nut van het onderzoek. Afhankelijk van de probleemstelling moeten vragen
gesteld worden als: draagt het onderzoek bij aan de kennis op een
bepaald gebied, worden door het onderzoek oplossingen gecreëerd voor
ervaren knelpunten of problemen, of draagt het onderzoek bij aan te
creëren oplossingen? Voldoet de vraagstelling nog aan het
oorspronkelijke probleem (of de oorspronkelijke vraagstelling) van de
opdrachtgevers? Zijn er voldoende (technische, financiële, praktische)
mogelijkheden om het onderzoek uit te voeren?

In de volgende stap moet worden gespecificeerd hoe de gegevens worden
verzameld. Dit is een essentiële stap die van invloed is op de rest van
het onderzoek; we wijden er daarom een apart hoofdstuk aan
(Hoofdstuk \ref{ch:steekproeftrekking}).
Waaruit bestaat de populatie: uit
taalgebruikers? leerlingen? tweetalige babies? versprekingen van
medeklinkers? zinnen? En hoe moet je een representatieve steekproef of
steekproeven trekken uit deze populatie(s)? Hoe groot moet die
steekproef dan zijn? Ook moet er in deze fase gekozen worden voor een
analysemethode. Het is zelfs aan te bevelen om in deze fase al een
analyseplan te ontwerpen. Welke analyses zullen worden uitgevoerd, welke exploraties van de gegevens worden voorzien?

Met al deze keuzes zijn de voorbereidingen nog niet afgerond. Ook de
instrumenten moeten worden gekozen: welke apparatuur,
opname-gereedschap, vragenlijsten, enz., worden gebruikt om waarnemingen
mee te doen? Bestaan er al geschikte instrumenten? Zo ja, zijn deze dan
makkelijk toegankelijk en mogen zij gebruikt worden? Zo nee, dan moeten
instrumenten ontwikkeld worden
(§\ref{sec:instrumentatie-onderzoek}).
Maar in dat geval neemt de
onderzoeker ook de taak op zich om deze instrumenten eerst te beproeven,
om na te gaan of de gegevens die met deze instrumenten verkregen worden,
voldoen aan de kwaliteitseisen die de onderzoeker zich gesteld heeft, of
die in het algemeen aan de instrumenten in wetenschappelijk onderzoek
gesteld mogen worden (in termen van betrouwbaarheid en validiteit, zie
Hoofdstukken \ref{ch:validiteit} en \ref{ch:betrouwbaarheid}.

Pas nadat ook de instrumenten in gereedheid gebracht zijn begint het
eigenlijke empirische onderzoek: de gekozen gegevens van de gekozen
steekproef worden verzameld op de gekozen wijze met behulp van de
gekozen instrumenten. Ook hierbij zijn er allerlei, vaak praktische
problemen waar de onderzoeker tegenaan loopt. Een waar gebeurd
voorbeeld: drie dagen nadat een onderzoeker zijn vragenlijst verstuurd
had begon een poststaking die twee weken duurde. Helaas had de
onderzoeker de respondenten ook twee weken de tijd gegeven om te
reageren. Dus toen de poststaking voorbij was, was de inzendtermijn
verlopen. Wat moest hij toen? Bij gebrek aan alternatieven besloot de
onderzoeker alle 1020 respondenten telefonisch te benaderen met het
verzoek de vragenlijst alsnog in te vullen en te retourneren.

Voor de onderzoeker die zich de moeite getroost heeft van te voren een
analyseplan op te stellen breekt nu de tijd aan om te oogsten. Eindelijk
kunnen de geplande analyses ook uitgevoerd worden. Helaas blijkt de
werkelijkheid meestal veel weerbarstiger dan de onderzoeker van te voren
had bedacht. Proefpersonen geven onverwachte responsies, of houden zich
niet aan de instructies, veronderstelde verbanden blijken niet aanwezig,
en onverwachte (en ongewenste) verbanden blijken in sterke mate
aanwezig. In latere hoofdstukken zullen we dieper ingaan op
analysemethoden en problemen daarbij.

Tenslotte moet de onderzoeker ook rapporteren over het onderzoek. Zonder
(adequaat) onderzoeksverslag zijn de gegevens niet toegankelijk en had
het onderzoek net zo goed \emph{niet} uitgevoerd kunnen worden. Dit is een
essentiële stap, waarbij onder meer de vraag gesteld dient te worden of
het onderzoek op basis van de verslaglegging controleerbaar en
repliceerbaar is. Meestal wordt van onderzoeksactiviteiten verslag
gedaan in de vorm van een werkstuk, een onderzoeksrapport of een artikel in een wetenschappelijk tijdschrift. Soms wordt van een onderzoek ook
verslag gedaan in een meer populair tijdschrift, dat voor een bredere
doelgroep bedoeld is dan alleen collega-onderzoekers.

Tot zover een beknopt overzicht van de keuzen die onderzoekers moeten
maken tijdens hun onderzoek. Ieder empirisch onderzoek bestaat uit een
aaneenschakeling van problemen, keuzes en beslissingen. De belangrijkste
keuzes zijn al gemaakt voordat de onderzoeker begint met gegevens
verzamelen.

\hypertarget{ch:integriteit}{%
\chapter{Integriteit}\label{ch:integriteit}}

\hypertarget{sec:integriteit.inleiding}{%
\section{Inleiding}\label{sec:integriteit.inleiding}}

Wetenschappelijk onderzoek heeft de mensheid onmetelijk grote baten
opgeleverd, zoals betrouwbare computer-technologie, goede medische zorg,
en begrip van andere talen en culturen. Al deze verworvenheden zijn
gebaseerd op wetenschappelijk onderbouwde kennis. Onderzoekers
produceren kennis, en de vooruitgang en groei van kennis ontstaat omdat
onderzoekers voortbouwen op de ervaringen en inzichten van hun
voorgangers.

\begin{center}\rule{0.5\linewidth}{0.5pt}\end{center}

\begin{quote}
\emph{Voorbeeld 3.1}: Sir Isaac Newton schreef over zijn
wetenschappelijke werk: ``If I have seen further it is by standing on (the)
shoulders of Giants'' (in een brief aan Robert Hooke d.d. 5 Feb
1676\footnote{Een kopie van de brief is te lezen via
  \url{http://digitallibrary.hsp.org/index.php/Detail/Object/Show/object_id/9285};
  voor achtergrond-informatie zie
  \url{http://www.bbc.co.uk/worldservice/learningenglish/movingwords/shortlist/newton.shtml}.}). Dit beeld is te herleiden tot de middeleeuwse geleerde
Bernard de Chartres: ``\ldots nos esse quasi nanos gigantum umeris
insidentes'' (dat wij zijn als dwergen gezeten op de schouders van
reuzen) in vergelijking tot geleerden uit de Oudheid. Newton's
uitspraak is ook het motto van Google Scholar
(\url{scholar.google.com}), een zoekmachine voor
wetenschappelijke publicaties.
\end{quote}

\begin{center}\rule{0.5\linewidth}{0.5pt}\end{center}

In dit hoofdstuk bespreken we de ethische en morele aspecten van
wetenschappelijk onderzoek. Wetenschap is mensenwerk, en het vereist een
goed ontwikkeld beoordelingsvermogen van de onderzoekers. De
\emph{Nederlandse Gedragscode Wetenschappelijke Integriteit} \citep{VSNU18}
(\url{http://www.vsnu.nl/wetenschappelijke_integriteit}) beschrijft hoe
wetenschappelijke onderzoekers (en studenten) zich dienen te gedragen.
Volgens deze gedragscode dient wetenschappelijk onderzoek en onderwijs
gebaseerd te zijn op de volgende principes:

\begin{itemize}
\item
  eerlijkheid,
\item
  zorgvuldigheid,
\item
  transparantie,
\item
  onafhankelijkheid, en
\item
  verantwoordelijkheid
\end{itemize}

In de volgende paragrafen zullen we nagaan hoe we volgens deze principes
dienen te handelen bij de verschillende fasen van wetenschappelijk
onderzoek. Hoe moeten we op eerlijke, zorgvuldige, transparante,
onafhankelijke en verantwoordelijke wijze een onderzoek opzetten, de
gegevens verzamelen en verwerken, en verslag doen van het onderzoek? We
moeten daarover nadenken nog voor het onderzoek begint, en daarom
bespreken we deze onderwerpen aan het begin van deze syllabus, hoewel we
ook vooruit zullen wijzen naar termen en begrippen die worden uitgewerkt
in volgende hoofdstukken.

\hypertarget{sec:ontwerp}{%
\section{Ontwerp}\label{sec:ontwerp}}

Weliswaar levert wetenschappelijk onderzoek ons onmetelijk grote baten
op, maar daar staan ook aanzienlijke kosten tegenover. De directe kosten
zijn o.a. de inrichting en onderhoud van laboratoria, apparatuur en
technische ondersteuning, maar ook de loonkosten van de onderzoekers,
vergoedingen voor informanten en proefpersonen, reiskosten voor toegang
tot bibliotheken, archieven, informanten en proefpersonen, e.d. Deze
directe kosten worden doorgaans gefinancierd uit publieke middelen van
universiteiten en andere wetenschappelijke instellingen. Daarnaast zijn
er indirecte kosten, die voor een deel ten laste komen van de
informanten en proefpersonen: tijd en moeite die niet aan iets anders
besteed kan worden, verlies van privacy, en mogelijke andere risico's
die we nog niet kennen. Een vaak vergeten kostenpost is het verlies van
onbevangenheid: een proefpersoon die heeft meegedaan aan een experiment
leert daarvan, en reageert daarna misschien anders in een volgend
experiment (zie
§\ref{sec:internevaliditeit}, onder Geschiedenis). De resultaten
uit zo'n volgend experiment zijn daardoor minder goed generaliseerbaar
naar andere personen die een andere geschiedenis hebben, en \emph{niet}
eerder aan een onderzoek hebben meegedaan.

Gezien de grote kosten moet onderzoek zodanig zijn doordacht en
ontworpen, dat de verwachte baten redelijkerwijs opwegen tegen de
verwachte kosten \citep[Ch.3]{Rose08}. Als de kans op valide conclusies uit
een onderzoek erg klein is, dan is het beter om dat onderzoek \emph{niet} uit
te voeren, en zo de directe en indirecte kosten te besparen.

\begin{center}\rule{0.5\linewidth}{0.5pt}\end{center}

\begin{quote}
\emph{Voorbeeld 3.2}:
Stel dat we willen onderzoeken of tweetalige kinderen van 4 jaar oud een
cognitief voordeel hebben boven eentalige leeftijdsgenoten. Op grond van
eerder onderzoek verwachten we een verschil van tenminste 2 punten (op
een 10-punts-schaal) tussen beide groepen (met ``pooled standard
deviation'' \(s_p=4\), dus \(d=0.5\), zie
§ \ref{sec:ttoets-formules} en
§ \ref{sec:ttoets-effectgrootte}).
\end{quote}

\begin{quote}
We vergelijken twee groepen van
elk \(n=4\) kinderen. Zelfs als er inderdaad een verschil is van 2 punten
tussen de twee groepen (dus als de onderzoekshypothese waar is), dan nog
is er in dit onderzoek slechts 51\% kans om een significant verschil te
vinden: de power is slechts .51
(Hoofdstuk \ref{ch:power}),
omdat de twee groepen zo weinig proefpersonen
bevatten. De vierjarige kinderen en hun ouders kunnen beter andere
dingen doen (school, thuis, werk) dan meedoen aan dit onderzoek.
\end{quote}

\begin{quote}
Als er echter \(n=30\) kinderen in elk van de twee groepen zouden meedoen,
en als er inderdaad een verschil is van 2 punten tussen de twee groepen
(dus als de onderzoekshypothese waar is) dan zou de power .90 zijn. Met
grotere groepen hebben we dus een veel betere kans om onze
onderzoekshypothese te bevestigen. Dit uitgebreide ontwerp van het
onderzoek zal meer kosten (voor de onderzoekers en de kinderen en hun
ouders), maar levert vermoedelijk ook veel meer op: een valide conclusie
met grote maatschappelijke impact.
\end{quote}

\begin{center}\rule{0.5\linewidth}{0.5pt}\end{center}

Het ontwerp van een onderzoek (zie
Hoofdstuk \ref{ch:ontwerp}) moet zo efficiënt mogelijk zijn, en de
onderzoeker moet daarover al in een vroeg stadium nadenken. De
efficiëntie hangt ten eerste af van keuzes over hoe de onafhankelijke
variabelen worden gevarieerd. Is er een aparte groep proefpersonen voor
iedere conditie van de onafhankelijke variabele (condities zijn ``between
subjects'', zoals in voorbeeld 3.2 hierboven? Bij een between-subjects
ontwerp met twee groepen zijn er ca \(n=(5.6/d)^2\) nodig in elke groep \citep{Gelm07}
(zie §\ref{sec:ttoets-effectgrootte}). Of doen alle proefpersonen mee
aan alle condities (condities zijn ``within subjects'')? Bij een
within-subjects ontwerp met twee condities zijn er dan slechts
\(n=(2.8/d)^2\) proefpersonen nodig in elke conditie, en het onderzoek
heeft dan dus minder directe en indirecte kosten voor veel minder
proefpersonen. In het algemeen is het daarom beter om indien mogelijk,
onafhankelijke variabelen te variëren binnen proefpersonen, en niet
tussen proefpersonen. Toch is dat niet altijd mogelijk, ten eerste omdat
individuele kenmerken nu eenmaal alleen verschillen tussen proefpersonen
(denk aan: mannelijk/vrouwelijk geslacht, wel/niet meertalige jeugd,
wel/niet afasie, enz.). Ten tweede moeten we terdege rekening houden met
effecten van `transfer' tussen condities, die de validiteit bedreigen
(denk aan: ervaring, leren, vermoeidheid, rijping). We keren hierop
terug in §\ref{sec:validiteit}.

Meertaligheid en geslacht zijn kenmerken die alleen tussen personen
kunnen variëren. Maar andere condities kunnen ook variëren binnen
personen, bijvoorbeeld de dag waarop een cognitieve meting wordt
afgenomen. Stel dat we een verschil verwachten van \(D=2\) punten tussen
cognitieve metingen afgenomen op maandag of op vrijdag (met \(s=4\) en
\(d=0.5\), zie voorbeeld 3.2. Als we de dag van de meting variëren
tussen proefpersonen, en dus aparte groepen maken voor de
maandag-kinderen en de vrijdag-kinderen, dan zijn er \(n=(5.6/0.5)^2=126\)
kinderen nodig in iedere groep, dus \(N=252\) kinderen in totaal. Als we
de dag van de meting echter variëren binnen proefpersonen, en iedere
proefpersoon dus observeren zowel op maandag als op vrijdag, dan zijn er
in totaal slechts \(N=(2.8/0.5)^2=32\) kinderen nodig. Met het
within-subjects ontwerp hoeven we dus veel minder kinderen lastig te
vallen met onze cognitieve meting. Wel moeten we terdege rekening houden
met leereffecten tussen de eerste en de tweede meting, en daarvoor
gepaste maatregelen treffen. We kunnen bijvoorbeeld niet meer dezelfde
vragenlijsten afnemen in beide condities.

De efficiëntie van een onderzoek hangt ook af van de afhankelijke
variabele, en met name van het meetniveau
(Hoofdstuk \ref{ch:meetniveau}), de nauwkeurigheid, en de betrouwbaarheid
van de observaties
(Hoofdstuk \ref{ch:betrouwbaarheid}). Hoe lager het meetniveau, des te lager
ook de efficiëntie van het onderzoek. En hoe lager de nauwkeurigheid,
des te lager ook de efficiëntie van het onderzoek, en des te meer
proefpersonen en observaties zijn er nodig om valide conclusies te
kunnen trekken.

Stel dat we een verschil willen onderzoeken tussen twee condities binnen
proefpersonen, en stel dat het verschil in werkelijkheid 2 punten
bedraagt (met \(s_D=4\) en \(d=0.5\), zie
voorbeeld 3.2). We kijken nu echter niet naar de richting en
de grootte van het verschil, maar alleen naar de \emph{richting} van het
verschil tussen de twee observaties per proefpersoon: heeft die
proefpersoon een positief of een negatief verschil tussen de eerste en
de tweede conditie? Deze binomiale afhankelijke variabele bevat minder
informatie dan de oorspronkelijke puntenscore (nl. alleen de richting,
en niet de grootte van het verschil), en het onderzoek is daardoor dus
minder efficiënt. We hebben daarom in dit specifieke voorbeeld niet 34
maar tenminste 59 proefpersonen nodig.

Onderzoekers zijn dus verantwoordelijk om de kosten en baten van hun
onderzoek zorgvuldig en eerlijk af te wegen en te beoordelen, en zij
dienen te beschikken over voldoende methodologische bagage om een goed
onderzoeksontwerp (design) te kiezen gezien het tijdsbestek, de mogelijk
beschikbare proefpersonen, de meetinstrumenten, enz.

\hypertarget{proefpersonen-en-informanten}{%
\section{Proefpersonen en informanten}\label{proefpersonen-en-informanten}}

Wetenschappelijk onderzoek is mensenwerk: onderzoekers zijn ook mensen.
Op het gebied van de geesteswetenschappen bestuderen die onderzoekers
weer het gedrag en de geestelijke producten van (andere) mensen.
Daarvoor gelden wetten, regels, richtlijnen en gedragscodes waaraan
onderzoekers (en studenten!) zich dienen te houden, vanuit de eerder
genoemde principes van zorgvuldigheid en verantwoordelijkheid. Het
onderzoek zelf, en de verzamelde gegevens, mogen geen schade of groot
verlies van privacy opleveren voor de deelnemers.

Voor geesteswetenschappelijk onderzoek zijn twee wetten relevant:

\begin{itemize}
\item
  Algemene Verordening Gegevensbescherming (AVG),\\
  zie
  \url{https://autoriteitpersoonsgegevens.nl/nl/onderwerpen/avg-europese-privacywetgeving}
\item
  Wet Medisch-wetenschappelijk Onderzoek met mensen (WMO),\\
  zie \url{http://www.wetten.nl}
\end{itemize}

Het is verplicht om proefpersonen (of hun wettelijke vertegenwoordigers)
te vragen om expliciete ``informed consent''. Dat houdt in dat de
proefpersonen eerlijk geïnformeerd worden over het onderzoek, over de
baten en kosten daarvan, en over hun beloning, en dat zij daarna (d.i.
``informed'') expliciet toestemmen in hun deelname (``consent'').
Voorbeelden van informed consent (informatiebrieven en
toestemmingsverklaringen) zijn te vinden op de website van de Facultaire
Ethische Toetsingscommissie (FETC, hieronder nader besproken), via
\url{https://fetc-gw.wp.hum.uu.nl/}.

Alle gegevens waaruit een individuele persoon te herleiden is, worden
beschouwd als ``persoonsgegevens'', en deze persoonsgegevens mogen alleen
worden verzameld en verwerkt conform de AVG. Het is raadzaam om de
onderzoeksgegevens zo snel mogelijk los te koppelen van de
persoonsgegevens, d.w.z. dat je de gegevens anonimiseert. De koppeling
tussen persoonsgegevens en en onderzoeksgegevens (bijv. een lijst met
namen van proefpersonen en hun bijbehorende anonieme persoonlijke code)
is zelf weer vertrouwelijke informatie die je zorgvuldig moet bewaren en
opslaan. Bewaar de persoonsgegevens niet langer dan nodig. De
onderzoeksgegevens mag je alleen gebruiken voor het (wetenschappelijke)
doel waarmee ze zijn verzameld. Zorg ook dat de proefpersonen niet
herkenbaar zijn (gebruik anonieme codes) in verslagen en publicaties
over het onderzoek.

Foto's en opnames van personen (audio, video, fysiologische gegevens,
EEG) vallen onder het zgn. portretrecht. Foto's en andere
identificerende opnames worden dus als portretten beschouwd. Bij
publicatie kan de afgebeelde/weergegeven persoon zich beroepen op het
portretrecht, en een schadevergoeding eisen voor het letsel dat hem of
haar door die publicatie wordt aangedaan. Als je een herkenbare opname
zou willen publiceren, dan moet je dus vooraf expliciete toestemming
daarvoor vragen van de opgenomen persoon of zijn wettelijke
vertegenwoordiger (zie het bovengenoemde voorbeeld van ``informed
consent''). Dat geldt ook als je een fragment van zo'n opname laat zien
of horen tijdens een conferentie of op een website.

In de wet WMO is vastgelegd dat onderzoek met mensen eerst moet worden
goedgekeurd door een speciale commissie; voor de Faculteit
Geesteswetenschappen van de Universiteit Utrecht is dat de
Medisch-Ethische Toetsingscommissie die valt onder het Universitair
Medisch Centrum Utrecht (METC). Die commissie weegt af of de mogelijke
baten van het onderzoek redelijkerwijs opwegen tegen de kosten en
mogelijke schade voor de proefpersonen.

Het meeste onderzoek op het gebied van talen en communicatie bij de
Universiteit Utrecht is vrijgesteld van de tijdrovende toetsing door de
METC, maar moet wel verplicht worden voorgelegd aan de \textbf{Facultaire
Ethische Toetsingscommissie} (FETC), en wel aan de kamer Linguïstiek
daarvan. Dat geldt echter niet voor onderzoek door studenten! Op de website
van de FETC is meer informatie te vinden:
\url{https://fetc-gw.wp.hum.uu.nl/}. Overleg bij twijfel altijd met je begeleider of docent.
Ethische toetsing is ook verplicht voor
studenten en onderzoekers uit andere domeinen (literatuur, geschiedenis,
media \& cultuur) die van plan zijn onderzoek te verrichten met mensen.

\hypertarget{gegevens}{%
\section{Gegevens}\label{gegevens}}

De verzamelde data of gegevens vormen de onderbouwing voor de conclusies
uit wetenschappelijk onderzoek. Die gegevens zijn daarmee van essentieel
belang: zonder gegevens geen valide conclusies. Zoals we hierboven zagen
(§\ref{sec:ontwerp})
zijn die gegevens bovendien zeer kostbaar (in tijd, geld, privacy, enz).
We moeten er dus uiterst zorgvuldig mee omgaan. We moeten anderen kunnen
overtuigen van de validiteit van onze conclusies op basis van die
gegevens, en we moeten de onderliggende gegevens desgevraagd kunnen
delen met andere onderzoekers.

Die zorgvuldigheid vereist dus in ieder geval dat we zo snel mogelijk
voldoende reservekopieën maken, en die bewaren op verschillende veilige
plaatsen. Bedenk eens wat er zou gebeuren als een brand of overstroming
je werkplek of woning zou vernietigen, of als tijdens je scriptie-project je laptop wordt gestolen (waar gebeurd!). Heb je dan goede en
recente kopieën van de gegevens elders opgeslagen? Voor kopieën en
`backups' kun je goed gebruik maken van een afdoende beveiligde ``cloud
service''\footnote{Medewerkers van de UU kunnen SurfDrive
  (\url{https://www.surfdrive.nl}) gebruiken om gegevens veilig en
  makkelijk te bewaren op een beveiligde netwerk-schijf.}.

De zorgvuldigheid vereist ook dat we goed bijhouden wat de gegevens
voorstellen, en hoe ze zijn verzameld. Gegevens zonder bijbehorende
beschrijving zijn nagenoeg waardeloos voor wetenschappelijk onderzoek.
Charles Darwin noteerde nauwkeurig welke vogel op welk van de
Galapagos-eilanden welke vorm van snavel had, en deze observaties
vormden later (deel van) de onderbouwing van zijn evolutie-theorie. Houd
dus een logboek bij (op papier of digitaal) waarin je alle stappen van
je onderzoek beschrijft, en eventueel motiveert. Noteer ook merk en type
en instellingen van de gebruikte apparatuur, en noteer versie-nummer en
instellingen van de gebruikte software. Houd bij welke bewerkingen je op
de gegevens hebt toegepast, en waarom, en in welk bestand welke gegevens
zijn opgeslagen.

Als je werkt met geditigaliseerde data (bv in Excel of SPSS of R), houd
dan ook zorgvuldig bij welke variabelen in welke kolom is opgeslagen, in
welke eenheden, en met welke codes.

\begin{center}\rule{0.5\linewidth}{0.5pt}\end{center}

\begin{quote}
\emph{Voorbeeld 3.3}: Het
bestand \url{http://tinyurl.com/nj4pjaq} bevat gegevens van 80 sprekers van
het Nederlands, ten dele ontleend aan het Corpus Gesproken Nederlands
(CGN). De eerste regel bevat de namen van de variabelen. Iedere volgende
regel correspondeert met één spreker. De gegevens op iedere regel zijn
gescheiden door spaties. De eerste kolom bevat de anonieme
identificatie-code van de spreker volgens het CGN. In de vijfde kolom
staat de regio van herkomst van de spreker gecodeerd, als één
letterteken, met de codes \texttt{W} Randstad, \texttt{M} Midden-Nederland, \texttt{N}
Noord-Nederland, \texttt{S} Zuid-Nederland) \citep{Quene08}. Door de zorgvuldige
annotatie zijn deze gegevens nog goed bruikbaar, ook al zijn ze ruim 20
jaar geleden verzameld door collega-onderzoekers.
\end{quote}

\begin{center}\rule{0.5\linewidth}{0.5pt}\end{center}

Gegevens blijven het intellectuele eigendom van degene die ze heeft
verzameld. Gebruik van andermans data zonder bronvermelding kan
beschouwd worden als diefstal, of als plagiaat.

Fraude met gegevens (gegevens fabriceren of verzinnen, in plaats van
observeren) is uiteraard strijdig met meerdere principes uit de
bovengenoemde gedragscode \citep{VSNU18}. Fraude schaadt het wederzijds
vertrouwen waarop wetenschap is gebaseerd. Het misleidt andere
onderzoekers die voortbouwen op de fictieve resultaten, en
onderzoeksgeld voor die frauduleuze onderzoekslijn wordt weggezogen uit
ander, niet frauduleus onderzoek --- kortom, een wetenschappelijke
doodzonde. Als je wilt overleggen over vragen of dilemma's hierover,
neem dan contact op met prof.dr. Josine Blok, vertrouwenspersoon
wetenschappelijke integriteit van de Faculteit Geesteswetenschappen
(\texttt{j.h.blok@uu.nl}).

\hypertarget{teksten}{%
\section{Teksten}\label{teksten}}

Wetenschappelijk onderzoek wordt pas echt nuttig, als de resultaten
ervan verspreid worden. Onderzoek dat niet wordt gerapporteerd, zou net
zo goed \emph{niet} kunnen zijn uitgevoerd, en de kosten van dat onderzoek
zijn dan feitelijk tevergeefs geweest. Een belangrijk deel van het
wetenschappelijk werk bestaat daarom uit verslaglegging ervan.
Publicaties (en octrooien) vormen een zeer belangrijk deel van de
``output'' van wetenschappelijk onderzoek. Onderzoekers worden gemeten
naar het aantal publicaties, en naar de ``impact'' daarvan (het aantal
malen dat die publicaties weer geciteerd worden door anderen die erop
voortbouwen). Mede gezien de grote belangen dienen we dus zorgvuldig om
te gaan met teksten van anderen en van onszelf.

De onderzoekers die betrokken zijn bij een onderzoek, moeten met elkaar
overleggen wie de auteurs van het verslag of van de publicatie zullen
zijn, en in welke volgorde. Mede-auteurs van een wetenschappelijk
verslag moeten voldoen aan drie voorwaarden \citep[Ch.10]{ORI12}. Ten eerste
moeten zij een substantiële wetenschappelijke bijdrage hebben geleverd
aan één of meer fasen in het onderzoek: het oorspronkelijke idee
bedenken, het onderzoek opzetten en ontwerpen, de gegevens verzamelen,
en de gegevens analyseren en interpreteren. Ten tweede moeten ze hebben
meegewerkt aan het verslag, als schrijver en/of als commentator. Ten
derde moeten ze instemmen met de definitieve tekst van het verslag
(meestal impliciet, soms expliciet), en tevens instemmen met hun
mede-auteurschap daarvan. De auteurs doen er goed aan om af te spreken
in welke volgorde hun namen vermeld worden. Meestal correspondeert die
volgorde met het afnemend belang en de afnemende omvang van de
respectievelijke bijdragen van de auteurs. Als de eindverantwoordelijke
hoofd-onderzoeker tevens mede-auteur is, dan wordt deze vaak als laatste
genoemd.

\begin{center}\rule{0.5\linewidth}{0.5pt}\end{center}

\begin{quote}
\emph{Voorbeeld 3.4}:
Student-assistent A heeft geholpen bij het verzamelen van de gegevens,
maar deze assistent heeft geen andere bijdragen geleverd, en weet niet
goed waar het onderzoek eigenlijk over gaat. A hoeft geen mede-auteur te
worden van het verslag, maar de auteurs dienen de bijdrage van A wel te
beschrijven en te erkennen in hun verslag.
\end{quote}

\begin{quote}
Student B heeft één van de delen van een onderzoeksproject uitgevoerd
onder begeleiding van onderzoeker C. Deze begeleider C heeft het hele
project bedacht, maar B heeft literatuur verzameld, een deelonderzoek
opgezet en uitgevoerd, data verzameld, geanalyseerd en geïnterpreteerd,
en daarvan verslag gedaan in een werkstuk. Student B en begeleider C
zijn daarom beiden mede-auteurs van een publicatie over B's deel van het
onderzoeksproject. Zij spreken af in welke volgorde de auteurs genoemd
worden. Omdat student B het belangrijkste was voor dit werk, en C de
eindverantwoordelijke was, spreken zij af dat B de eerste auteur wordt
en C de tweede en laatste.
\end{quote}

\begin{center}\rule{0.5\linewidth}{0.5pt}\end{center}

Onderzoekers bouwen voort op het werk van hun voorgangers (zie voorbeeld 3.1).
Dat kan ook gelden voor hun
redeneringen, en zelfs hun teksten, maar daarbij moeten we dan altijd
correct verwijzen naar de juiste bron, d.w.z. naar het werk van die
voorgangers. Anders is immers niet meer te onderscheiden wie
verantwoordelijk is voor welke gedachte of tekstfragment. Plagiaat is
``het overnemen van stukken, gedachten, redeneringen van anderen en deze
laten doorgaan voor eigen werk'' (\emph{Van Dale}, 12e druk). Ook deze vorm
van fraude is een wetenschappelijke doodzonde, waar krachtige sancties
op kunnen volgen. De Faculteit Geesteswetenschappen van de UU zegt
daarover het volgende:

\begin{quote}
"Van plagiaat is sprake bij het in een scriptie of ander werkstuk
gegevens of tekstgedeelten van anderen overnemen zonder
bronvermelding. Onder plagiaat valt onder meer:

\begin{itemize}
\item
  het knippen en plakken van tekst van digitale bronnen zoals
  encyclopedieën en digitale tijdschriften zonder aanhalingstekens
  en verwijzing;
\item
  het knippen en plakken van teksten van het internet zonder
  aanhalingstekens en verwijzing;
\item
  het overnemen van gedrukt materiaal zoals boeken, tijdschriften en
  encyclopedieën zonder aanhalingstekens en verwijzing;
\item
  het opnemen van een vertaling van bovengenoemde teksten zonder
  aanhalingstekens en verwijzing;
\item
  het parafraseren van bovengenoemde teksten zonder (deugdelijke)
  verwijzing: parafrasen moeten als zodanig gemarkeerd zijn (door de
  tekst uitdrukkelijk te verbinden met de oorspronkelijke auteur in
  tekst of noot), zodat niet de indruk wordt gewekt dat het gaat om
  eigen gedachtegoed van de student;
\item
  het overnemen van beeld-, geluids- of testmateriaal van anderen
  zonder verwijzing en zodoende laten doorgaan voor eigen werk;
\item
  het zonder bronvermelding opnieuw inleveren van eerder door de
  student gemaakt eigen werk en dit laten doorgaan voor in het kader
  van de cursus vervaardigd oorspronkelijk werk, tenzij dit in de
  cursus of door de docent uitdrukkelijk is toegestaan;
\item
  het overnemen van werk van andere studenten en dit laten doorgaan
  voor eigen werk. Indien dit gebeurt met toestemming van de andere
  student is de laatste medeplichtig aan plagiaat;
\item
  ook wanneer in een gezamenlijk werkstuk door een van de auteurs
  plagiaat wordt gepleegd, zijn de andere auteurs medeplichtig aan
  plagiaat, indien zij hadden kunnen of moeten weten dat de ander
  plagiaat pleegde;
\item
  het indienen van werkstukken die verworven zijn van een
  commerciële instelling (zoals een internetsite met uittreksels of
  papers) of die tegen betaling door iemand anders zijn geschreven."
\end{itemize}

\url{http://students.uu.nl/praktische-zaken/regelingen-en-procedures/fraude-en-plagiaat}
\end{quote}

Bij plagiaat van eigen werk worden de teksten of gedachten niet
overgenomen van anderen maar van één van de auteurs. Over dit
zelf-plagiaat wordt verschillend gedacht; het is echter raadzaam om in
voorkomende gevallen wel de bron te vermelden, vanuit de principes van
zorgvuldigheid, betrouwbaarheid, controleerbaarheid, en
verantwoordelijkheid.

Een verwijzing of citatie of referentie is een verkorte bronvermelding
in de tekst; in dit boek ben je er al vele tegengekomen. Aan het
einde van de tekst volgt dan de volledige
lijst van bronnen, meestal aangeduid als bronvermeldingen, geraadpleegde
bronnen, referenties, literatuur, of bibliografie (``boekbeschrijving'').
Een foutieve bronvermelding kan worden beschouwd als plagiaat \citep{UBVU15}
omdat de lezer niet verwezen wordt naar de juiste bron. Onderzoekers
dienen hun bronnen daarom op correcte wijze te vermelden. Daarvoor
bestaan verschillende conventies, afhankelijk van het vakgebied. Meestal
zal een docent aangeven volgens welke stijl of conventie je je bronnen
moet vermelden. In dit boek volgen we zoveel mogelijk de stijl van de \citep{APA10}, die
gebruikelijk is in de sociale wetenschappen en een deel van de
geesteswetenschappen.

De regels voor bronvermelding zijn soms ingewikkeld. Bovendien moeten de
auteurs zorgen dat de citaties in de tekst overeenkomen met de lijst van
referenties. Deze taken kunnen beter worden bijgehouden door een
zgn.``reference manager'', een programma dat referenties of
bronvermeldingen verzamelt en op de juiste wijze invoegt in de tekst.
Een overzicht van zulke programma's is te vinden via
\url{https://en.wikipedia.org/wiki/Comparison_of_reference_management_software}.
Voor dit tekstboek is gebruik gemaakt van Zotero,
gecombineerd met BibTeX.

\hypertarget{ch:Meetniveau}{%
\chapter{Meetniveau}\label{ch:Meetniveau}}

\hypertarget{inleiding-1}{%
\section{Inleiding}\label{inleiding-1}}

In Hoofdstuk \ref{ch:onderzoek} maakten we al kennis met variabelen:
eigenschappen die verschillende waarden kunnen aannemen. De waarde van
een variabele is dus een aanduiding van een eigenschap, of kwaliteit, of
hoedanigheid, van een object of persoon. Als het gaat om een
afhankelijke variabele, dan wordt die waarde ook aangeduid als \emph{score}
of \emph{responsie}, vaak aangeduid met symbool \(Y\). De wijze waarop een
kenmerk wordt uitgedrukt in een (gemeten) waarde, noemen we het
\emph{meetniveau} van de variabele; het meetniveau is dus een eigenschap of
kenmerk van de variabele zelf! We onderscheiden vier meetniveau's, in
toenemende niveau's van informativiteit: nominaal, ordinaal, interval,
ratio. Bij de eerste twee meetniveau's worden alleen discrete
categorieën onderscheiden, zonder of met ordening. Bij de laatste twee
meetniveau's worden getalswaarden gebruikt, zonder of met nulpunt. We
zullen de meetniveau's hieronder nader bespreken. Inzicht in het
meetniveau van een variabele is van belang voor de interpretatie van de
scores op een variabele en --- zoals we later zullen zien --- voor de
keuze van de juiste statistische toets om een onderzoeksvraag te
beantwoorden.

\hypertarget{sec:nominaal}{%
\section{Nominaal}\label{sec:nominaal}}

We spreken van een nominale variabele (of nominaal meetniveau) als een
kenmerk gecategoriseerd wordt in afzonderlijke (discrete) categorieën,
waarbij er \emph{niet} een ordening is tussen de categorieën. Bekende
voorbeelden zijn o.a. de nationaliteit van een proefpersoon, het merk
van een auto, de kleur van iemands ogen, de smaak van een bak schepijs,
je woonsituatie (bij gezin van herkomst, op kamers, zelfstandig,
samenwonend, anders), enz. De scores kunnen alleen gebruikt worden om de
categorieën te onderscheiden (de uitspraak ``vanille is anders dan
aardbei'' is wel zinnig). We kunnen wel tellen hoe vaak iedere categorie
voorkomt, maar er is geen interpreteerbare rangorde (de uitspraak ``vanille
is groter dan aardbei'' is onzinnig), en we kunnen ook niet rekenen met
de gemeten waarden van een nominale variabele. We kunnen dus wel de
meest voorkomende nationaliteit vaststellen, maar we kunnen niet de
gemiddelde nationaliteit uitrekenen.

\hypertarget{sec:ordinaal}{%
\section{Ordinaal}\label{sec:ordinaal}}

Er is sprake van een ordinale variabele (of van een ordinaal meetniveau)
als een kenmerk gecategoriseerd wordt in afzonderlijke categorieën,
waarbij er \emph{wel} een rangorde is tussen de categorieën. Bij een ordinale
variabele weten we echter niets over de afstand tussen de verschillende
categorieën. Bekende voorbeelden zijn o.a. schooltype (VMBO, HAVO, VWO,
\ldots), antwoord op een schaalvraag (\emph{mee eens, neutraal, niet mee eens}),
positie op een ranglijst, volgorde van afvallen bij een talentenjacht,
kledingmaat (XS, S, M, L, XL, \ldots), of militaire rang (soldaat, majoor,
generaal, \ldots). Ook hier kunnen we wel tellen hoe vaak iedere categorie
voorkomt, en we kunnen ook de rangorde zinnig interpreteren (wie als
laatste afvalt presteert beter dan wie als eerste afvalt, maat L is
groter dan M, een generaal is de baas van een majoor). We kunnen echter
niet rekenen met de gemeten waarden van een ordinale variabele. We
kunnen wel de meest verkochte kledingmaat vaststellen, maar we kunnen
niet de gemiddelde verkochte kledingmaat uitrekenen\footnote{Als de helft van de respondenten \emph{mee eens} antwoordt, en de
  andere helft \emph{niet mee eens}, dan kunnen we niet zinnig concluderen
  dat de responsies gemiddeld \emph{neutraal} zouden zijn.}.

\hypertarget{sec:interval}{%
\section{Interval}\label{sec:interval}}

Er is sprake van een interval-variabele (of van een interval-meetniveau)
als een kenmerk uitgedrukt wordt in een getal op een continue schaal,
waarbij deze schaal \emph{niet} een nulpunt heeft. Door de schaal weten we
bij een interval-variabele ook wat de afstanden of intervallen zijn
tussen de verschillende waarden. Bekende voorbeelden zijn o.a.
temperatuur in graden Celcius (het nulpunt is arbitrair), of jaartal
(idem). We kunnen tellen hoe vaak iedere categorie voorkomt, we kunnen
de rangorde zinnig interpreteren (het jaar 1999 in onze gregoriaanse
kalender ging vooraf aan het jaar 2000), en we kunnen ook de intervallen
zinnig interpreteren (van 1918 tot 1939 is net zo lang als van 1989 tot
2010). We kunnen wel rekenen met de waarden van een interval-variabele,
maar de enige zinnige bewerkingen zijn optellen en aftrekken. Daarmee
kunnen we wel een gemiddelde berekenen, bijv. het gemiddelde jaar waarin de
personen in een steekproef hun eerste mobiele telefoon begonnen te gebruiken.

\hypertarget{sec:ratio}{%
\section{Ratio}\label{sec:ratio}}

Het vierde en hoogste meetniveau is het ratio-niveau. Er is sprake van
een ratio-variabele (of van een ratio-meetniveau) als een kenmerk
uitgedrukt wordt in een getal op een continue schaal, waarbij deze
schaal \emph{wel} een nulpunt heeft. Door de schaal weten we bij een
ratio-variabele wat de afstanden of intervallen zijn tussen de
verschillende waarden. Bovendien weten we door het nulpunt wat de
verhoudingen of ratio's zijn tussen de verschillende waarden. Bekende
voorbeelden zijn o.a. temperatuur in graden Kelvin (vanaf het absolute
nulpunt), de responsietijd\footnote{Het nulpunt is het moment van de gebeurtenis waarop de
  proefpersoon moet reageren.} in duizendsten van een seconde (ms), je
lengte in cm, je leeftijd in jaren, het aantal gemaakte fouten in een
toets, enz. Bij een ratio-variabele kunnen we tellen hoe vaak iedere
categorie voorkomt, we kunnen de rangorde zinnig interpreteren (iemand
van 180 cm is langer dan iemand van 179 cm), we kunnen intervallen
zinnig interpreteren (de toename in leeftijd van 12 naar 18 is tweemaal
zo groot als de toename van 9 naar 12), en we kunnen ook verhoudingen
tussen de waarden zelf zinnig interpreteren (een leeftijd van 24 is
\emph{tweemaal} zo oud als een leeftijd van 12). We kunnen rekenen met de
waarden van een interval-variabele, en daarbij kunnen we niet alleen
optellen en aftrekken maar ook delen en vermenigvuldigen. Ook hier is
het mogelijk om een gemiddelde te berekenen, bijv. de gemiddelde leeftijd
waarop de personen in een steekproef hun eerste mobiele telefoon begonnen te gebruiken.

\hypertarget{sec:ordeningvanmeetniveaus}{%
\section{Ordening van meetniveaus}\label{sec:ordeningvanmeetniveaus}}

De meetniveaus zijn hierboven besproken in toenemende informativiteit of
sterkte. Een nominale variabele bevat het minste informatie en geldt als
het laagste meetniveau, en een ratio-variabele bevat het meeste
informatie en geldt als het hoogste meetniveau.

Het is altijd mogelijk om gegevens gemeten op een hoger meetniveau te
interpreteren alsof ze op een lager niveau zijn gemeten. Als we
bijvoorbeeld het maandinkomen van de personen in een steekproef hebben
gemeten op ratio-niveau (in €), dan kunnen we daar probleemloos een
ordinale variabele van maken (\emph{minder dan modaal, van modaal tot
tweemaal modaal, meer dan tweemaal modaal}). We gooien daarbij
informatie weg: de oorspronkelijke meting in € bevat meer informatie dan
de daaruit afgeleide classificatie in drie geordende categorieën.

Natuurlijk is het omgekeerde niet mogelijk: een variabele van een laag
meetniveau kunnen we niet interpreteren op een hoger niveau. We zouden
dan informatie achteraf moeten toevoegen die we niet hebben verzameld
bij de oorspronkelijke meting van die variabele. Het is dus zaak om de
relevante variabelen te meten of te observeren op het juiste meetniveau.
Stel je
voor dat we de lichaamslengte van volwassen mannen en vrouwen willen
vergelijken. Als we de lichaamslengte meten op ordinaal meetniveau (met
drie categorieën \emph{kort}, \emph{middelmatig} en \emph{lang} gelijkelijk
gedefinieerd voor alle personen), dan kunnen we dus niet de gemiddelde
lichaamslengte uitrekenen, en we kunnen ook niet een statistische toets
gebruiken die refereert aan het gemiddelde van de lichaamslengte. Dat
hoeft geen probleem te zijn, maar het is wel goed om vooraf te
doordenken wat de consequenties zijn van de keuze voor een bepaald
meetniveau.

\hypertarget{ch:validiteit}{%
\chapter{Validiteit}\label{ch:validiteit}}

\hypertarget{inleiding-2}{%
\section{Inleiding}\label{inleiding-2}}

Experimenteel onderzoek heeft tot doel om hypotheses te toetsen. Ook in
ander, niet-experimenteel onderzoek kunnen hypotheses worden getoetst,
maar we beperken ons hier voor de helderheid tot experimenteel
onderzoek, d.w.z. onderzoek waarin het experiment als methode wordt
gebruikt. In experimenteel onderzoek wordt getracht causale verbanden
aannemelijk te maken. Als de resultaten van een experimenteel onderzoek
de onderzoekshypothese bevestigen (d.w.z. de nulhypothese wordt
verworpen), dan is het aannemelijk dat een verandering in de
onafhankelijke variabele de oorzaak (Latijn: \emph{causa}) is voor een
verandering of \emph{effect} in de afhankelijke variabele. Zo kunnen we na
experimenteel onderzoek met enige zekerheid concluderen, bijvoorbeeld,
dat een verschil in behandeling na een herseninfarct de oorzaak is, of
een belangrijke oorzaak is, van een verschil in taalvaardigheid van een
patiënt zoals geobserveerd 6 maanden na een herseninfarct. Het
experiment heeft aannemelijk gemaakt dat er een causaal of oorzakelijk
verband is tussen de behandelingsmethode (onafhankelijke variabele) en
de resulterende taalvaardigheid (afhankelijke variabele).

\hypertarget{sec:causaliteit}{%
\section{Causaliteit}\label{sec:causaliteit}}

Een causaal of oorzakelijk verband tussen twee variabelen is iets anders
dan een `gewoon' verband of samenhang tussen twee variabelen. Als twee
verschijnselen met elkaar samenhangen, hoeft het ene niet de oorzaak van
het andere te zijn. Een eerste voorbeeld zien we bij de samenhang tussen
de lengte van personen en hun gewicht: lange mensen zijn over het
algemeen zwaarder dan korte mensen (en omgekeerd: korte mensen zijn over
het algemeen lichter dan lange mensen). Is er nu sprake van een causale
relatie tussen lengte en gewicht? Wordt het ene kenmerk (deels)
veroorzaakt door het andere? Nee, in dit voorbeeld is er wel samenhang
maar geen causaal verband tussen de kenmerken: zowel lengte als gewicht
worden ``veroorzaakt'' door andere variabelen, o.a. genetische
eigenschappen en voedingspatronen. Een tweede voorbeeld is de samenhang
tussen motivatie en taalvaardigheid van iemand die een vreemde taal
leert: hoog gemotiveerde studenten leren een nieuwe vreemde taal beter
en vlotter dan laag gemotiveerde studenten, maar ook hier is niet
duidelijk wat de oorzaak en wat het gevolg is.

Een causaal verband is een speciale vorm van samenhang. Een causaal
verband is een verband tussen twee twee verschijnselen of kenmerken,
waarbij bovendien voldaan moet zijn aan een aantal extra voorwaarden
\citep{SCC02}. Ten eerste moet de oorzaak aan het gevolg vooraf gaan (na
behandeling treedt verbetering op). Ten tweede moet het gevolg niet
optreden als de oorzaak niet aanwezig was (zonder behandeling geen
verbetering). Bovendien moet het gevolg --- althans in theorie ---
altijd optreden als de oorzaak aanwezig is (behandeling resulteert
altijd in verbetering). Ten derde kunnen we geen andere plausibele
verklaring vinden voor het optreden van het gevolg, behalve de mogelijke
oorzaak. Als we het causale mechanisme kennen (we snappen waarom
behandeling de oorzaak is van verbetering), dan zijn we beter in staat
om mogelijke andere plausibele verklaringen uit te sluiten. Helaas is
dat bij de gedragswetenschappen, inclusief de taalwetenschap, echter
zelden het geval. We constateren wel dat een behandeling resulteert in
verbetering, maar de theorie die oorzaak (behandeling) en gevolg
(verbetering) verbindt is zelden compleet en vertoont belangrijke
lacunes. Dat betekent dat we goede voorzorgen moeten treffen in onze
onderzoeksmethoden, teneinde mogelijke alternatieve plausibele
verklaringen voor de gevonden effecten uit te sluiten.

\hypertarget{sec:validiteit}{%
\section{Validiteit}\label{sec:validiteit}}

Een bewering of conclusie is \emph{valide} als de bewering \emph{waar} (true) en
\emph{gerechtvaardigd} (justified) is. Een ware uitspraak correspondeert met
de werkelijkheid: de bewering \emph{ieder kind leert ten minste een taal} is
waar, omdat de bewering de werkelijkheid goed weergeeft. Een
gerechtvaardigde bewering ontleent geldigheid aan de empirische
evidentie waarop die bewering is gebaseerd: ieder kind dat wij hebben
geobserveerd, of dat anderen hebben geobserveerd, leert een taal of
heeft een taal geleerd (behalve bijzondere gevallen voor wie een aparte
verklaring nodig is). De rechtvaardiging van een bewering is sterker
naarmate de methode van (directe of indirecte) observatie sterker is en
meer zekerheid biedt. Dit houdt ook in dat de validiteit van een
bewering niet een categoriale eigenschap is (wel/niet valide) maar een
gradueel kenmerk: een bewering kan meer of minder valide zijn.

Aan de validiteit van een bewering kunnen drie verschillende aspecten
worden onderscheiden.

\begin{enumerate}
\def\labelenumi{\arabic{enumi}.}
\item
  In hoeverre zijn de conclusies over de relaties tussen de
  afhankelijke en de onafhankelijke variabele geldig? Deze vraag heeft
  betrekking op de \emph{interne validiteit}.
\item
  In hoeverre zijn de uitwerkingen, operationaliseringen, van de
  afhankelijke en onafhankelijke variabele adequaat? Deze vraag heeft
  betrekking op de \emph{constructvaliditeit}.
\item
  In hoeverre kunnen de conclusies gegeneraliseerd worden naar andere
  proefpersonen, stimuli, condities, situaties, observaties? Deze
  vraag heeft betrekking op de \emph{externe validiteit}.
\end{enumerate}

Deze drie vormen van validiteit zullen wij in de navolgende paragrafen
toelichten.

\hypertarget{sec:internevaliditeit}{%
\section{Interne validiteit}\label{sec:internevaliditeit}}

Het is vanzelfsprekend de bedoeling om in een experimenteel onderzoek
zoveel mogelijk alternatieve verklaringen voor de onderzoeksresultaten
uit te sluiten. Er moet immers aangetoond worden dat er een causaal
verband is tussen twee variabelen X en Y, en daarbij moeten storende
factoren zoveel mogelijk onder controle gehouden worden. Laten we eens
kijken naar voorbeeld 5.1 hieronder.

\begin{center}\rule{0.5\linewidth}{0.5pt}\end{center}

\begin{quote}
\emph{Voorbeeld 5.1}: \citep{Verh04} onderzochten o.a. de hypothese dat
ouderen (boven de 45 jaar) langzamer spreken dan jongeren (onder de 40
jaar). Om dat te onderzoeken werd spraak opgenomen van 160 sprekers,
gelijk verdeeld over de twee leeftijdsgroepen, in een interview van
ongeveer 15 minuten. Na fonetische analyse van de articulatiesnelheid
blijkt dat de ``jongeren'' relatief snel spreken met 4.78 lettergrepen per
seconde, en de ``ouderen'' aanzienlijk langzamer, met 4.52 lettergrepen
per seconde \citep[p.302]{Verh04}. We concluderen dat de hogere leeftijd de
\emph{oorzaak} is van het lagere spreektempo bij de oudere sprekers --- maar
is die conclusie terecht?
\end{quote}

\begin{center}\rule{0.5\linewidth}{0.5pt}\end{center}

Deze vraag naar de rechtvaardiging van de conclusie is een vraag naar de
interne validiteit van het onderzoek. De interne validiteit heeft
betrekking op de relaties tussen gemeten of gemanipuleerde variabelen,
en is onafhankelijk van de (theoretische) constructen die de
verschillende variabelen representeren (vandaar de term `interne
validiteit'). Of, anders gezegd: de vraag naar de interne validiteit is
een vraag naar mogelijke alternatieve verklaringen voor de gevonden
onderzoeksresultaten. Veel van de mogelijke alternatieve verklaringen
kunnen worden ondervangen door de manier waarop de gegevens worden
verzameld. We bespreken hieronder de meest in het oog lopende
bedreigingen van de interne validiteit \citep{SCC02}.

\begin{enumerate}
\def\labelenumi{\arabic{enumi}.}
\tightlist
\item
  \textbf{Geschiedenis}
  is een bedreiging van de
  interne validiteit. Het begrip `geschiedenis' omvat o.a. gebeurtenissen
  die hebben plaatsgevonden tussen of tijdens een voormeting en een
  nameting; het gaat dan om gebeurtenissen die geen deel uitmaken van de
  experimentele manipulatie (de onafhankelijke variabele), maar die wel
  van invloed zouden kunnen zijn op de afhankelijke variabele. Een
  hittegolf, bijvoorbeeld, kan van invloed zijn op het gedrag van de
  proefpersonen tijdens een onderzoek.
\end{enumerate}

In een laboratorium wordt de `geschiedenis' onder controle gehouden door
de proefpersonen af te sluiten van invloeden van buitenaf (zoals een
hittegolf), of door afhankelijke variabelen te kiezen die nauwelijks
beïnvloed kunnen worden door externe factoren. In onderzoek buiten het
laboratorium, waaronder veldonderzoek, is het veel lastiger en vaak
zelfs onmogelijk om invloeden van buitenaf onder controle te houden. In
het volgende voorbeeld wordt dit duidelijk.

\begin{center}\rule{0.5\linewidth}{0.5pt}\end{center}

\begin{quote}
\emph{Voorbeeld 5.2}: In een onderzoek worden twee methoden vergeleken om leerlingen een
vreemde taal te leren spreken, i.c. Nieuw-Grieks. De eerste groep moet
Griekse woordjes en grammatica leren in een klaslokaal, gedurende enkele
weken. De tweede groep gaat in diezelfde periode op een studiereis naar
Griekenland, waar leerlingen moeten converseren in de doeltaal. De
totale tijd besteed aan het taalvaardigheidsonderwijs is voor beide
groepen gelijk. Na afloop blijkt de taalvaardigheid van de tweede groep
groter dan die van de eerste groep. Wordt dat verschil in de
afhankelijke variabele inderdaad veroorzaakt door de lesmethode
(onafhankelijke variabele)?
\end{quote}

\begin{center}\rule{0.5\linewidth}{0.5pt}\end{center}

\begin{enumerate}
\def\labelenumi{\arabic{enumi}.}
\setcounter{enumi}{1}
\tightlist
\item
  \textbf{Rijping} is de natuurlijke veroudering of rijping van proefpersonen
  tijdens een onderzoek. Als de proefpersonen gedurende een onderzoek
  ouder worden, zich ontwikkelen, meer ervaren of sterker worden, èn als
  deze rijping niet is opgenomen in de onderzoeksvraag, dan vormt rijping
  een bedreiging van de interne validiteit. In experimenten waarin
  reactietijden worden gemeten, bijvoorbeeld, zien we meestal dat de
  reactietijden van een proefpersoon sneller worden gedurende het
  experiment, als gevolg van training en oefening. We kunnen de interne
  validiteit dan beschermen tegen dit leer-effect, door de stimuli voor
  iedere proefpersoon in een andere willekeurige volgorde aan te bieden.
\end{enumerate}

Meestal is er sprake van rijping doordat de proefpersonen vele malen
achtereen dezelfde taak uitvoeren of dezelfde vragen beantwoorden.
Rijping kan echter ook optreden wanneer proefpersonen hun antwoorden
kenbaar moeten maken op een juist niet gebruikelijke manier, bv. door
een ongewone vraagstelling, of in een ongebruikelijke vorm van
meerkeuze-vragen. Bij de eerste paar keer dat een proefpersoon dan
vragen beantwoordt, kan de wijze van beantwoorden interfereren met het
antwoord zelf. Achteraf kunnen we een vergelijking maken tussen bv. het
eerste kwart en het laatste kwart van de antwoorden, om zo te bekijken
of er een mogelijk effect was van ervaring, d.w.z. van rijping.

\begin{enumerate}
\def\labelenumi{\arabic{enumi}.}
\setcounter{enumi}{2}
\tightlist
\item
  Ook de \textbf{instrumentatie} of instrumenten die voor een onderzoek
  gebruikt worden, kunnen een bedreiging vormen voor de interne
  validiteit. Verschillende instrumenten die worden geacht hetzelfde
  construct te meten, moeten ook gelijke metingen produceren. En hetzelfde
  instrument moet ook gelijke metingen produceren onder verschillende
  omstandigheden. Voor computer-gestuurde experimenten is dat meestal geen
  probleem. Maar bij vragenlijsten, of bij de beoordeling van
  schrijfopdrachten, kan de interne validiteit wel worden bedreigd.
\end{enumerate}

Bij veel onderzoek worden observaties gedaan zowel voorafgaand aan een
behandeling, als na afloop daarvan. Daarbij kan dezelfde toets gebruikt
worden, maar dan kan er een leer-effect optreden (zie hierboven).
Onderzoekers gebruiken daarom vaak verschillende toetsen bij de
voormeting en de nameting, maar daarbij kan er wel een
instrumentatie-effect optreden. De onderzoeker moet de mogelijke voor-
en nadelen tegen elkaar afwegen.

\begin{center}\rule{0.5\linewidth}{0.5pt}\end{center}

\begin{quote}
\emph{Voorbeeld 5.3}: \citep{Rijl86}
onderzocht het effect van `peer evaluation' op de kwaliteit van
schrijfproducten. De opzet van zijn onderzoek was (enigszins
vereenvoudigd) als volgt: eerst schrijven de leerlingen een opstel over
onderwerp A, dan volgt het schrijfonderwijs inclusief `peer evaluation',
waarna nogmaals een opstel geschreven wordt over onderwerp B. De
schrijfproducten van voormeting en nameting worden beoordeeld, waarna
getoetst wordt of de gemiddelde prestaties verschillen tussen voormeting
en nameting.
\end{quote}

\begin{quote}
In dit onderzoek vormt niet alleen de interventie (schrijfonderwijs) een
duidelijk verschil tussen de voormeting en de nameting, maar ook het
onderwerp van de schrijfopdracht (A of B) vormt een verschil. Het is
twijfelachtig of met beide schrijfopdrachten wel precies hetzelfde wordt
gemeten. Dit verschil in instrumentatie bedreigt de interne validiteit,
omdat er op verschillende momenten misschien een (deels) verschillend
aspect van de schrijfvaardigheid is gemeten. De instrumentatie (hier:
het verschil in onderwerpen van de schrijfopdrachten) geeft een
plausibele alternatieve verklaring voor een verschil in
schrijfvaardigheid, naast of in plaats van de onafhankelijke variabele
(hier: het tussentijdse schrijfonderwijs).
\end{quote}

\begin{center}\rule{0.5\linewidth}{0.5pt}\end{center}

\begin{enumerate}
\def\labelenumi{\arabic{enumi}.}
\setcounter{enumi}{3}
\tightlist
\item
  Een volgende bedreiging van de interne validiteit staat bekend als
  het effect van \textbf{regressie naar het gemiddelde}. Regressie naar het
  gemiddelde kan een rol spelen zodra het onderzoek gericht is op speciale
  groepen, bijvoorbeeld slechte lezers, slechte schrijvers, maar evenzo:
  goede lezers, goede schrijvers, etc. We geven eerst een voorbeeld, omdat
  het verschijnsel niet direct intuïtief duidelijk is.
\end{enumerate}

\begin{center}\rule{0.5\linewidth}{0.5pt}\end{center}

\begin{quote}
\emph{Voorbeeld 5.4}:
Er is enige controverse over het gebruik van illustraties in
kinderboeken. Sommigen menen dat in boeken waarmee kinderen leren lezen
geen (of zo min mogelijk) illustraties mogen voorkomen: illustraties
leiden de aandacht af van te leren kenmerken van woorden. Anderen menen
dat in illustraties wezenlijke informatie weergegeven kan worden:
illustraties dienen als extra informatiebron.
\end{quote}

\begin{quote}
\citep{Dona83} onderzocht de invloed van illustraties bij een tekst op het begrip van
die tekst. De onderzoeker selecteerde 120 leerlingen (uit 1868
leerlingen) uit de derde en zesde groep van het basisonderwijs; 60 uit
elk van beide groepen. Volgens de prestaties op een eerder afgenomen
leestoets bleken van de 60 leerlingen per klas er 30 als slechte en 30
als goede lezers geclassificeerd te kunnen worden. Elke leerling kreeg
dezelfde tekst te zien, aangeboden met of zonder illustraties
(onafhankelijke variabele), zie Tabel \ref{tab:designDona83}.
\end{quote}

\begin{quote}
De resultaten bleken goeddeels de tweede hypothese te ondersteunen:
illustraties bevorderen het begrip van de tekst, ook bij onervaren
lezers. De slechtere lezers begrepen de tekst met illustraties beter, en
ook jongere lezers ondervonden voordeel van de illustraties.
\end{quote}

\begin{center}\rule{0.5\linewidth}{0.5pt}\end{center}

\begin{longtable}[]{@{}lccc@{}}
\caption{\label{tab:designDona83} Aanbiedingscondities in het onderzoek van Donald (1983).}\tabularnewline
\toprule
groep & leesvaardigheid & conditie & \(n\)\tabularnewline
\midrule
\endfirsthead
\toprule
groep & leesvaardigheid & conditie & \(n\)\tabularnewline
\midrule
\endhead
3 & slecht & zonder & 15\tabularnewline
3 & slecht & met & 15\tabularnewline
3 & goed & zonder & 15\tabularnewline
3 & goed & met & 15\tabularnewline
6 & slecht & zonder & 15\tabularnewline
6 & slecht & met & 15\tabularnewline
6 & goed & zonder & 15\tabularnewline
6 & goed & met & 15\tabularnewline
\bottomrule
\end{longtable}

Wat is er nu mis met dit onderzoek? Het antwoord is gelegen in de manier
waarop leerlingen zijn geselecteerd. Lezers werden ingedeeld als
`slecht' of `goed' op basis van een leesvaardigheidstoets, maar hun
prestaties op die toets worden altijd beïnvloed door toevallige
factoren, die niets met leesvaardigheid te maken hebben: Tom voelde zich
niet lekker, daarom heeft hij deze toets slecht gemaakt, Sarah was met
haar gedachten elders, Niels had last van zijn knie, Julie was enorm
gemotiveerd en heeft zichzelf overtroffen. Met andere woorden: de
leesvaardigheid is niet geheel betrouwbaar gemeten. Dit betekent (1) dat
de slechte lezers die toevallig boven hun niveau gepresteerd hebben, ten
onrechte niet bij de slechte lezers ingedeeld werden, maar deel
uitmaakten van de groep goede lezers; en omgekeerd (2) dat goede lezers
die bij deze toets toevallig onder hun niveau gepresteerd hebben, ten
onrechte als slechte lezers bestempeld werden. Onder de slechte lezers
zitten dus altijd ook een paar lezers die helemaal zo slecht nog niet
zijn, en onder de goede lezers zitten ook een paar lezers die eigenlijk
niet zo goed zijn.

Wanneer de eigenlijk-goede lezers, die ten onrechte geclassificeerd zijn
als niet-goede lezers, een tweede leestoets maken (nadat zij een tekst
met of zonder illustraties bestudeerd hebben), dan zullen zij meestal
weer op hun gewone hoge niveau presteren. Een hogere score op de tweede
toets (de nameting) kan dus een artefact zijn van de selectiemethode.
Hetzelfde geldt, mutatis mutandis, voor de eigenlijk-slechte lezers die
ten onrechte geselecteerd zijn als niet-slechte lezers. Wanneer deze
leerlingen een tweede leestoets maken, dan zullen ook zij meestal weer
op hun gewone (lage) niveau presteren. De score op de nameting ligt voor
hen dus lager dan de score op de voormeting.

Voor het aangehaalde onderzoek van betekent dit dat het geconstateerde
verschil tussen slechte en goede lezers deels toevallig is. Ook als de
onafhankelijke variabele geen effect heeft, zal de groep `goede' lezers
bij de tweede leestoets gemiddeld slechter presteren, en zal de groep
`slechte' lezers bij de tweede leestoets gemiddeld beter presteren. Met
andere woorden: het verschil tussen de twee groepen is bij de nameting
minder groot dan bij de voormeting, als gevolg van toevallige variatie:
regressie naar het gemiddelde. Het zal duidelijk zijn dat
onderzoeksresultaten getroebleerd kunnen worden door dit verschijnsel.
Zoals we hierboven zagen kan een experimenteel effect afgezwakt worden
of verdwijnen als gevolg van regressie naar het gemiddelde; omgekeerd
kan regressie naar het gemiddelde ten onrechte aangezien worden als een
experimenteel effect \citep{RMP18}.

In het algemeen kan regressie naar het gemiddelde optreden als er een
classificatie gemaakt wordt op basis van een voormeting, waarvan de
scores samenhang vertonen met de scores van de nameting (zie
Hoofdstuk \ref{ch:samenhang}). Als er geen enkele correlatie is tussen
voormeting en nameting, dan speelt regressie naar het gemiddelde zelfs
de hoofdrol: een verschil tussen voormeting en nameting is dan alleen
het gevolg van regressie naar het gemiddelde. Als er perfecte correlatie
is, dan speelt regressie naar het gemiddelde geen enkele rol, maar dan
is ook de voormeting niet informatief, want immers geheel (achteraf) te
voorspellen uit de nameting.

Regressie naar het gemiddelde kan een alternatieve verklaring bieden
voor de vermeende grote toename van scores tussen voormeting en nameting
voor een lage prestatiegroep (bv. slechte lezers), ten opzichte van een
kleinere toename voor een hoge prestatiegroep (bv. goede lezers).
Omgekeerd kan het ook een alternatieve verklaring bieden voor de
vermeende afname van scores tussen voormeting en nameting voor een hoge
prestatiegroep (bv. goede lezers), ten opzichte van een lage
prestatiegroep (bv. slechte lezers).

Het is beter om de groepen \emph{niet} samen te stellen op basis van een van
de uitkomsten van een van de metingen (voormeting of nameting), maar op
basis van een ander, onafhankelijk criterium. De proefpersonen van beide
groepen zullen dan zullen bij de voormeting ongeveer gemiddeld scoren,
en het effect van regressie naar het gemiddelde is dan klein. In alle
groepen zullen dan ongeveer evenveel proefpersonen zitten met een door
het toeval iets te hoge als met een iets te lage uitgevallen score,
zowel bij de voormeting als bij de nameting.

\begin{enumerate}
\def\labelenumi{\arabic{enumi}.}
\setcounter{enumi}{4}
\tightlist
\item
  Een vijfde bedreiging van de interne validiteit is \textbf{selectie}.
  Hiermee doelen we (voornamelijk) op een
  zodanige verdeling van proefpersonen over verschillende condities dat
  deze bij aanvang van het onderzoek niet gelijkwaardig zijn. Wanneer
  bijvoorbeeld in de experimentele conditie alle slimme proefpersonen
  zitten, terwijl in de controleconditie alleen de domme leerlingen
  terecht gekomen zijn, dan kan een effect niet zonder meer aan de
  manipulatie van de onafhankelijke variabele toegeschreven worden. Het
  verschil in aanvangsniveau (hier: in intelligentie) levert dan een
  plausibele alternatieve verklaring die de interne validiteit bedreigt.
\end{enumerate}

\begin{center}\rule{0.5\linewidth}{0.5pt}\end{center}

\begin{quote}
\emph{Voorbeeld 5.5}: Voor een eerlijke vergelijking tussen scholen van hetzelfde schooltype
(VMBO, HAVO, VWO, etc) moeten we rekening houden met verschillen tussen
scholen in hun ingangsniveau van de leerlingen. Stel dat school A
leerlingen heeft met ingangsnivo 50, en eindexamennivo 100 (op een
willekeurige schaal). School B heeft leerlingen met ingangsnivo 30, en
eindexamennivo 90 (op dezelfde schaal). Is school B slechter dan A (want
lager eindnivo), of is school B beter dan A (want kleiner verschil in
eindnivo)?
\end{quote}

\begin{center}\rule{0.5\linewidth}{0.5pt}\end{center}

In veel onderwijskundig onderzoek is het onmogelijk om leerlingen van
verschillende klassen op basis van het toeval aan condities toe te
wijzen --- dit wordt wel aselecte toewijzing genoemd. Dit kan namelijk
onoverkomelijke organisatorische problemen met zich meebrengen. Deze
organisatorische problemen omvatten meer dan alleen het (aselect)
splitsen van de klas, hoewel dit vaak al lastig te realiseren is. Ook
moet de onderzoeker rekenschap afleggen van mogelijke
overdrachtseffecten tussen de condities: de leerlingen praten met
elkaar, leren elkaar misschien zelfs wel de essentialia van de
experimentele conditie(s). Het uitblijven van een effect zou dan op
tenminste één alternatieve manier verklaard kunnen worden. Vanwege de
geschetste problematiek worden vaak complete schoolklassen toegewezen
aan een van de condities. Maar klassen bestaan uit een aantal leerlingen
van dezelfde school. Bij de keuze van leerlingen, en hun ouders, voor
een school treedt zelf-selectie op (in het Nederlandse
onderwijssysteem), waardoor er verschillen zijn in uitgangspositie
tussen condities (d.w.z. tussen klassen binnen condities). Eventuele
gevonden verschillen tussen condities zouden dus ook door zelfselectie
van leerlingen naar scholen veroorzaakt kunnen zijn.

Hierboven is al de meest eenvoudige manier aangegeven om verschillende
condities een gelijk aanvangsniveau te geven: wijs de leerlingen
aselect, volgens toeval, `at random', toe aan de condities. Deze methode
staat bekend als \emph{randomisatie} \citep[p.294 ff]{SCC02}. We kunnen bijvoorbeeld
randomiseren door leerlingen een willekeurig (random) nummer te geven
(zie
Appendix \ref{app:randomgetallen}) en daarna de `even leerlingen' aan de
ene conditie en de `oneven leerlingen' aan de andere conditie toe te
wijzen. Bij aselecte toewijzing van proefpersonen aan condities berusten
alle verschillen tussen de proefpersonen in de verschillende condities
op toeval, en worden die verschillen dus uitgemiddeld. Naar alle
waarschijnlijkheid zijn er dan geen systematische verschillen tussen de
onderscheiden groepen of condities. Dit geldt echter alleen indien de
groepen groot genoeg zijn.

Randomisatie, de aselecte toewijzing van proefpersonen aan condities,
moet onderscheiden worden van de aselecte steekproeftrekking uit een
populatie (zie
§\ref{sec:aselectesteekproef}). Bij aselecte
steekproeftrekking gaat het om de willekeurige selectie van
proefpersonen uit de populatie van mogelijke proefpersonen naar de
steekproef; we streven er dan naar dat de steekproef of steekproeven
lijken op de populatie waaruit die getrokken is/zijn. Bij randomisatie
gaat het om de willekeurige toewijzing van de proefpersonen uit de
steekproef aan de verschillende condities van het onderzoek; we streven
er dan naar dat de steekproeven lijken op elkaar.

Een tweede methode om twee gelijke groepen te creëren is \emph{matching}. Bij
matching worden proefpersonen eerst gemeten op een aantal relevante
variabelen. Daarna worden koppels gevormd die een gelijke score op deze
variabelen hebben. Van deze koppels wordt er één aan de ene conditie en
één aan de andere conditie toegewezen. Matching heeft echter
verschillende bezwaren. Ten eerste kan regressie naar het gemiddelde een
rol gaan spelen. Ten tweede is matching, wanneer de proefpersonen op
meerdere variabelen gematcht moeten worden, zeer bewerkelijk, en is een
grote groep potentiële proefpersonen vereist. Ten derde wordt bij
matching alleen rekening gehouden met variabelen die de onderzoeker
relevant acht, en niet met andere onbekende variabelen. Bij randomisatie
wordt niet alleen gerandomiseerd naar die relevante variabelen, maar ook
naar andere eigenschappen die mogelijk een rol zouden kunnen spelen
zonder dat de onderzoeker zich dat realiseert. Kortom, de relatief
eenvoudige randomisatie is verre te prefereren boven matching.

\begin{enumerate}
\def\labelenumi{\arabic{enumi}.}
\setcounter{enumi}{5}
\tightlist
\item
  \textbf{Uitval} van respondenten is de laatste bedreiging van interne
  validiteit. In sommige gevallen begint een onderzoeker met veel
  proefpersonen. Gedurende het onderzoek vallen echter proefpersonen uit.
  Zolang het percentage uitvallers beperkt blijft, is er geen probleem.
  Maar er ontstaat wel een probleem, als de uitval selectief is voor één
  van de onderscheiden condities. Is dat laatste wel het geval, dan kan er
  over die conditie niet veel meer gezegd worden. Het probleem van uitval
  speelt vooral een rol bij longitudinaal onderzoek. Dit is onderzoek
  waarbij een beperkte groep respondenten gedurende een langere periode
  gevolgd wordt. Men heeft daarbij echter te maken met mensen die
  verhuizen, of overlijden gedurende het experiment, of participanten die
  niet meer willen meewerken, enz. Dit kan een enorme reductie van het
  aantal respondenten teweeg brengen.
\end{enumerate}

Hierboven hebben we een aantal veel voorkomende problemen besproken die
de interne validiteit van een onderzoek kunnen bedreigen. De lijst is
echter niet uitputtend! Elk type onderzoek heeft zo z'n eigen problemen,
en het is de taak van de onderzoeker om alert te zijn op mogelijke
bedreigingen van de interne validiteit. Probeer daartoe plausibele
verklaringen te bedenken die een eventueel effect ook, of zelfs beter
zouden kunnen verklaren dan de te onderzoeken oorzaak. De onderzoeker
moet dus denken als zijn eigen scepticus, die geenszins overtuigd is dat
de onderzochte factor werkelijk de oorzaak is van het gevonden effect.
Welke mogelijke alternatieve verklaringen zijn er volgens die scepticus,
en hoe zou de onderzoeker die bedreigingen voor de validiteit kunnen
wegnemen door de opzet van het onderzoek? Dat vereist goed inzicht in de
logische relaties tussen de onderzoeksvragen, de onderzochte variabelen,
de resultaten, en de conclusie.

\hypertarget{sec:constructvaliditeit}{%
\section{Constructvaliditeit}\label{sec:constructvaliditeit}}

In een experimenteel onderzoek wordt een onafhankelijke variabele
gemanipuleerd. Dit kan, afhankelijk van de vraagstelling, op vele
manieren. Evenzo kan de wijze waarop de afhankelijke variabele(n)
gemeten wordt op verschillende manieren vorm gegeven worden. De manier
waarop de onafhankelijke en de afhankelijke variabelen vorm gegeven
worden noemen we de \emph{operationalisatie} van deze variabelen. De
leesvaardigheid van leerlingen kan bijvoorbeeld geoperationaliseerd
worden als (a) hun score op een tekstbegriptoets met open vragen; (b)
hun score op een tekstbegriptoets met meerkeuzevragen; (c) hun score op
een zgn. cloze-toets (ontbrekend woord invullen); of (d) als de mate
waarin geschreven instructies uitgevoerd kunnen worden. Meestal zijn er
heel veel manieren om een variabele te operationaliseren, en zelden
volgt uit een theorie één dwingende beschrijving voor de wijze van
operationalisatie van de onafhankelijke of de afhankelijke variabelen.
\emph{Constructvaliditeit}, of \emph{begripsvaliditeit}, heeft betrekking op de
mate waarin de operationalisatie van zowel de afhankelijke variabele(n)
als de onafhankelijke variabele(n) een adequate afspiegeling is (zijn)
van de theoretische constructen waar het onderzoek zich op richt. Met
andere woorden: zijn de onafhankelijke en de afhankelijke variabelen
goed gerelateerd aan de theoretische concepten waar het onderzoek op
gericht is?

\begin{center}\rule{0.5\linewidth}{0.5pt}\end{center}

\begin{quote}
\emph{Voorbeeld 5.6}: De \emph{taalontwikkeling} van
babies en peuters is lastig te observeren, en al helemaal als het gaat
om de auditieve en perceptieve ontwikkeling van deze proefpersonen die
nog niet of nauwelijks zelf spreken. Een veel gebruikte methode is het
Head Turn Preference Paradigm \citep{John10}. Bij deze methode kijkt de baby
eerst naar een groen knipperend licht recht voor zich. Als de aandacht
van het kind zo is gevangen, dooft vervolgens het groene licht en begint
een rood licht te knipperen, aan de linker of rechter zijde van de
proefpersoon. Het kind draait dan zijn of haar hoofd om het knipperende
licht te zien. Vervolgens wordt er een spraakgeluidsbestand afgespeeld,
via een luidspreker vlak bij het knipperende licht aan de zijkant. De
afhankelijke variabele is de periode waarin het kind zijwaarts blijft
kijken (met minder dan 2 s onderbreking). Daarna begint een nieuwe
aanbiedingscyclus. De kijktijd wordt opgevat als een indicatie voor de
mate van voorkeur van het kind voor de gesproken stimulus.
\end{quote}

\begin{quote}
De interpretatie van de verkregen kijktijden is echter lastig, omdat
kinderen nu eens voorkeur hebben voor nieuwe geluidsstimuli (bv zinnen
in een onbekende taal), en dan weer juist aan bekende stimuli (bv
grammaticale vs ongrammaticale zinnen). Zelfs als de stimuli nauwkeurig
zijn afgestemd op het ontwikkelingsniveau van de proefpersoon, is het
lastig om de afhankelijke variabele (kijktijd) goed te relateren aan het
bedoelde theoretische construct (voorkeur van kind).
\end{quote}

\begin{center}\rule{0.5\linewidth}{0.5pt}\end{center}

\begin{quote}
\emph{Voorbeeld 5.7}: Zoals hierboven aangegeven kan het begrip \emph{leesvaardigheid} op allerlei
manieren worden geoperationaliseerd. Volgens sommigen kan
leesvaardigheid niet goed gemeten worden met behulp van meerkeuzevragen
(Houtman 1986, Shohamy 1984). Bij meerkeuzevragen worden de antwoorden
zeer sterk beïnvloed door andere zaken zoals algemene ontwikkeling,
gokvaardigheid, ervaring met eerdere toetsen, en door de vraagstelling
zelf, zoals geïllustreerd in deze vraag:
\end{quote}

\begin{quote}
\emph{Wie van de volgende personen heeft de afgelopen jaren een autobiografie
gepubliceerd?\\
\hspace*{0.333em}a. Jeanne d'Arc }(algemene ontwikkeling)\emph{\hfill\break
~b. mijn buurvrouw }(vraagstelling, ervaring)\emph{\hfill\break
~c.~Malala Yousafzai\\
\hspace*{0.333em}d.~Alexander Graham Bell }(algemene ontwikkeling)**\\
\end{quote}

\begin{quote}
Deze vraag is duidelijk niet construct-valide voor het meten van kennis
over autobiografieën.
\end{quote}

\begin{center}\rule{0.5\linewidth}{0.5pt}\end{center}

Uiteraard gelden bovengenoemde problemen met de constructvaliditeit niet
alleen voor schriftelijke vragen of meerkeuzevragen, maar ook voor
mondelinge vragen aan proefpersonen.

\begin{center}\rule{0.5\linewidth}{0.5pt}\end{center}

\begin{quote}
\emph{Voorbeeld 5.8}:
Als we ouders mondeling de vraag stellen \emph{Hoe vaak leest U uw kind
eigenlijk voor?} dan wekken we met die vraag al de suggestie dat
voorlezen wenselijk is. De ouders zouden hun voorleesgedrag wel eens
kunnen overschatten. We meten dus niet alleen het construct
`voorleesgedrag', maar ook het construct `neiging tot sociaal wenselijke
antwoorden' (zie hierna).
\end{quote}

\begin{center}\rule{0.5\linewidth}{0.5pt}\end{center}

Een notoir lastig construct om te operationaliseren is
\emph{schrijfvaardigheid}. Wat is een goed en wat is een slecht
schrijfproduct? En wat is dan eigenlijk schrijfvaardigheid? Kan
schrijfvaardigheid gemeten worden door een telling van relevante
inhoudselementen in een tekst, moeten er zinnen of woorden geteld
worden, of misschien vooral connectieven (\emph{dus, want, omdat, hoewel},
enz), moeten er oordelen van \emph{lezers} verzameld worden over de
geschreven tekst (t.a.v. doelgerichtheid, publiekgerichtheid, stijl), of
moet er één oordeel van lezers verzameld worden over de globale
kwaliteit, moeten er spelfouten geteld worden, etc? De problemen bij de
operationalisatie komen voort uit een gebrek aan theorie over
schrijfvaardigheid, waaruit een definitie voor de kwaliteit van
schrijfproducten afgeleid zou kunnen worden \citep{BM93}. Kritiek op
onderzoek naar schrijfvaardigheid is daarom makkelijk, maar alternatieve
operationalisaties van het construct zijn moeilijk.

Een ander lastig construct is de \emph{verstaanbaarheid} van gesproken
zinnen. Verstaanbaarheid (`intelligibility') kan op diverse manieren
worden geoperationaliseerd. De eerste mogelijkheid is dat de onderzoeker
de woorden of zinnen uitspreekt en dat de proefpersoon die naspreekt,
waarbij fouten in de reproductie geteld worden; een nadeel hierbij is
dat er nauwelijks controle is over de model-uitspraak van de
onderzoeker. Een tweede mogelijkheid is dat de woorden of zinnen vooraf
worden opgenomen en verder dezelfde procedure wordt gevolgd; een nadeel
blijft dat de responsies worden beïnvloed door kennis van de wereld,
grammaticale verwachtingen, bekendheid met de spreker of zijn
taalgebruik, enz. De meest betrouwbare methode is die van de zgn.
`speech reception threshold' \citep{Plomp79} beschreven in het volgende
voorbeeld. Deze methode heeft echter als nadeel dat ze
tijdrovend is, niet goed automatisch afgenomen kan worden, en dat er
veel stimulusmateriaal (spraakopnamen) nodig is (zijn) voor een enkele
meting.

\begin{center}\rule{0.5\linewidth}{0.5pt}\end{center}

\begin{quote}
\emph{Voorbeeld 5.19}: We laten een lijst van 13
gesproken zinnen horen, gemaskeerd met ruisgeluid. De verhouding tussen
spraak en ruis (speech-to-noise ratio, SNR) wordt uitgedrukt in dB. Bij
0 dB SNR zijn spraak en ruis even luid, bij +3 dB SNR is de spraak 3 dB
luider dan de ruis, bij -2 dB SNR is de spraak 2 dB \emph{zachter} dan de
ruis, etc. Na iedere zin moet de luisteraar de aangeboden zin naspreken.
Als dat foutloos gebeurt, dan wordt voor de volgende zin de SNR met 2 dB
verlaagd (minder spraak of meer ruis); als de responsie fout was, dan
wordt voor de volgende zin de SNR met 2 dB verhoogd (meer spraak of
minder ruis). Na een paar zinnen is er weinig variatie meer in SNR, en
schommelt de SNR rond een optimum. De gemiddelde SNR over de laatste 10
aangeboden zinnen vormt de `speech reception threshold' (SRT). Deze SRT
is ook op te vatten als de SNR waarbij de helft van de zinnen goed wordt
verstaan.
\end{quote}

\begin{center}\rule{0.5\linewidth}{0.5pt}\end{center}

Tot nog toe hebben we het gehad over problemen met betrekking tot de
constructvaliditeit van de afhankelijke variabelen. Maar ook de
operationalisatie van de \emph{on}afhankelijke variabele staat vaak ter
discussie. De onderzoeker heeft immers vele keuzes moeten maken tijdens
de operationalisering van zijn onafhankelijke variabele (zie
§\ref{sec:keuzemomenten}), en de gemaakte keuzes zijn vaak wel
aanvechtbaar.

Een onderzoek is niet constructvalide, of niet begripsvalide, als de
operationalisaties van de afhankelijke variabelen de toets der kritiek
niet kunnen doorstaan. Een onderzoek is ook niet constructvalide, als de
onafhankelijke variabele niet een valide operationalisatie is van
het-theoretische-begrip-zoals-bedoeld. Als die operationalisatie niet
valide is, dan wordt er dus eigenlijk iets anders gemanipuleerd dan de
bedoeling was. In dat geval is de relatie tussen de afhankelijke
variabele en de gemanipuleerde onafhankelijke variabele zoals bedoeld
niet eenduidig meer. Eventuele geobserveerde verschillen in de
afhankelijke variabele hoeven niet alleen veroorzaakt te worden door de
onafhankelijke variabele zoals bedoeld, maar kunnen ook beïnvloed zijn
door andere factoren. Een bekend effect in dit opzicht is het zogenaamde
Hawthorne-effect.

\begin{center}\rule{0.5\linewidth}{0.5pt}\end{center}

\begin{quote}
\emph{Voorbeeld 5.10}: De directie
van de Hawthorne Works Factory (Western Electric Company) in Cicero
(Illinois), USA, was gealarmeerd door slechte bedrijfsresultaten. Een
team onderzoekers nam de gang van zaken onder de loep, waarbij ongeveer
alles werd onderzocht: werktijden, beloning, pauzes, verlichting,
verwarming, werkoverleg, management, enz. De resultaten van dat
onderzoek (uit 1927) wezen uit dat de productiviteit enorm was gestegen
-- maar dat er geen verband was met een van de onafhankelijke
variabelen. De toename van productiviteit werd uiteindelijk
toegeschreven aan de grotere aandacht voor de werknemers.
\end{quote}

\begin{center}\rule{0.5\linewidth}{0.5pt}\end{center}

Het Hawthorne-effect houdt dus in dat een verandering in gedrag niet
samenhangt met de manipulatie van enige onafhankelijke variabele, maar
dat die verandering van gedrag het gevolg is van een psychologisch
verschijnsel: proefpersonen die weten dat ze worden geobserveerd, doen
meer hun best om gewenst gedrag te vertonen.

\begin{center}\rule{0.5\linewidth}{0.5pt}\end{center}

\begin{quote}
\emph{Voorbeeld 5.11}: \citep{RDCPW78}
vergeleken de effectiviteit van twee methoden ter verbetering van de
leesvaardigheid van slechte lezers. De leerlingen werden geselecteerd op
basis van hun scores op drie toetsen. De 72 geselecteerde leerlingen
werden aselect toegewezen aan één van de twee methode-condities
(gestructureerd leesonderwijs versus geprogrammeerde instructie). In de
eerste conditie werd het gestructureerde leesonderwijs verzorgd door
vier docenten, die aan een klein groepje (van vier leerlingen) les
gaven. In feite is de leerling-docent-ratio dus \(1:1\). In de tweede
conditie (geprogrammeerde instructie) bemoeiden de docenten zich zo min
mogelijk met de leerlingen. Het experiment nam 75 sessies van 45 minuten
in beslag. Na de tweede observatie bleek dat de leerlingen die volgens
de eerste (gestructureerde) methode les gekregen hadden, beter vooruit
waren gegaan dan de leerlingen die met behulp van de tweede methode
(geprogrammeerde instructie) les gekregen hadden.
\end{quote}

\begin{quote}
Tot zover is er geen probleem met dit onderzoek. Er ontstaat pas een
probleem als we zouden concluderen dat de gestructureerde methode beter
is dan de geprogrammeerde instructie. Een alternatieve verklaring, die
in dit onderzoek niet uitgesloten kan worden, is dat het gevonden effect
niet (alleen) het gevolg is van de methode, maar (ook) een gevolg is van
de grotere individuele aandacht in de eerste conditie (gestructureerd
leesonderwijs).
\end{quote}

\begin{center}\rule{0.5\linewidth}{0.5pt}\end{center}

Net zoals bij de interne validiteit kan ook bij de construct- of
begripsvaliditeit een aantal validiteitbedreigende factoren genoemd
worden.

\begin{enumerate}
\def\labelenumi{\arabic{enumi}.}
\tightlist
\item
  Een eerste bedreiging van de begripsvaliditeit is
  \emph{mono-operationalisatie}. In veel onderzoeken wordt de afhankelijke
  variabele slechts op één manier geoperationaliseerd. De proefpersonen
  hoeven slechts één taak uit te voeren, bv. één auditieve taak met
  reactietijdmetingen (over meerdere aanbiedingen), of één vragenlijst
  (met meerdere vragen). Het onderzoek staat of valt dan met deze
  specifieke operationalisatie van de afhankelijke variabele. Over de
  validiteit van deze specifieke operationalisatie zijn dan geen verdere
  gegevens voorhanden. De onderzoeker laat in zo'n geval ruimte voor
  twijfel. Strikt genomen moeten we de onderzoeker immers op zijn woord
  geloven omtrent de validiteit van zijn operationalisering. Dergelijk
  onderzoek kan veel beter worden uitgevoerd. De onderzoeker moet dan het
  te meten construct op verschillende manieren operationaliseren, bv. door
  meerdere auditieve taken te laten uitvoeren, met niet alleen
  reactietijdmetingen maar ook met tellingen van foutieve responsies. Of
  de onderzoeker laat niet alleen een vragenlijst invullen, maar
  observeert het bedoelde construct ook d.m.v. andere taken en
  observatiemethoden. Wanneer de prestaties op de verschillende typen
  responsies in hoge mate samenhangen, kan daarmee aangetoond worden dat
  al deze toetsen hetzelfde construct vertegenwoordigen. We noemen dit
  \emph{convergente validiteit}. Er is sprake van convergente validiteit als de
  prestaties op instrumenten die hetzelfde theoretische construct
  vertegenwoordigen, in hoge mate samenhangen (convergeren).
\end{enumerate}

Het is echter niet voldoende om te laten zien dat toetsen die hetzelfde
concept of construct beogen te meten, inderdaad convergent valide zijn.
Daarmee is immers nog niet aangetoond wat dit construct is, en evenmin
of het gemeten construct wel het bedoelde construct is. Hebben we wel
echt `vloeiendheid' van de spreker gemeten, met meerdere methoden, of
hebben we eigenlijk steeds het construct `aandacht' of `spreeksnelheid'
gemeten? En hebben we wel echt `mate van tekstbegrip' gemeten, met
verschillende convergente methoden, of hebben we eigenlijk steeds het
construct `faalangst' gemeten? Om de construct-validiteit te waarborgen
moet eigenlijk ook worden aangetoond dat de operationalisaties
\emph{divergent valide} zijn ten opzichte van operationalisaties die een
ánder aspect of een ándere (verwante) vaardigheid beogen te meten.
Kortom de onderzoeker moet kunnen aantonen dat de prestaties op
instrumenten (operationalisaties) die één vaardigheid (construct)
vertegenwoordigen in hoge mate samenhangen (convergeren), terwijl de
prestaties op instrumenten die verschillende vaardigheden
vertegenwoordigen juist lage samenhang vertonen (divergeren). Pas dan
heeft de onderzoeker aannemelijk gemaakt dat de specifieke
operationalisaties inderdaad constructvalide zijn.

\begin{enumerate}
\def\labelenumi{\arabic{enumi}.}
\setcounter{enumi}{1}
\tightlist
\item
  Ook de verwachtingen van de onderzoeker --- die zich uiten in bewust
  èn onbewust gedrag --- kunnen de constructvaliditeit van een onderzoek
  bedreigen. De onderzoeker is ook een mens, en is dus niet immuun voor de
  invloed van zijn of haar eigen verwachtingen op de uitkomsten van het
  onderzoek. Na afloop van het experiment is de invloed van de onderzoeker
  helaas moeilijk te achterhalen.
\end{enumerate}

\begin{center}\rule{0.5\linewidth}{0.5pt}\end{center}

\begin{quote}
\emph{Voorbeeld 5.12}:Kluger Hans was een paard dat kon rekenen.
Als aan Kluger Hans gevraagd werd
\emph{hoeveel is \(4+4\)?}, dan stampte het paard 8 maal met zijn rechter
voorhoef, als gevraagd werd \emph{hoeveel is \(3-1\)?}, dan stampte Hans twee
maal met zijn voorhoef. Kluger Hans baarde veel opzien en werd onderwerp
van verschillende studies. Een commissie stelde in 1904 vast dat Kluger
Hans inderdaad kon rekenen (en communiceren met mensen). Later
constateerde een lid van de onderzoekscommissie, Carl Stumpf, samen met
zijn assistent Oskar Pfungst, echter: ``\ldots het paard laat verstek gaan,
als de oplossing van de gestelde opgave aan geen van de aanwezigen
bekend is'' \citep[ p.185, vert. HQ]{Pfung07}, of als het de persoon die de
oplossing weet niet kan zien. ``Es bedarf also optischer Hilfen'' (idem).
Na nauwkeurige observaties bleek dat de baas van Kluger Hans (en andere
aanwezigen) zich een heel klein beetje ontspande zodra Hans het juiste
aantal malen met zijn rechter voorpoot gestampt had. Dit onopzettelijke
teken was voor Kluger Hans voldoende aanleiding om het stampen te staken
(d.i. om zijn rechter voorhoef op de grond te houden), teneinde daarna
zijn beloning van wortels en brood in ontvangst te nemen
\citep{Pfung07} \citep[p.38--47]{Watz77}.
\end{quote}

\begin{quote}
Een misschien vergelijkbaar, recenter geval is dat van Alex, een
papegaai met bijzondere cognitieve gaven, zie o.a. \citep{BLLB14} en \citep{Alex15}.
\end{quote}

\begin{center}\rule{0.5\linewidth}{0.5pt}\end{center}

Het beroemde voorbeeld van Kluger Hans illustreert hoe subtiel de invloed van een
onderzoeker of proefleider op het te onderzoeken object kan zijn. Deze
invloed bedreigt natuurlijk de constructvaliditeit. Het is daarom beter
als de onderzoeker niet ook zelf fungeert als experimentator\footnote{De experimentator is degene die een experiment afneemt bij een
  proefpersoon. De experimentator kan een andere persoon zijn dan de
  onderzoekers die de onderzoekshypothesen hebben opgesteld en/of
  proefpersonen hebben gerecruteerd.} of
proefleider. Studies waarin de onderzoeker zelf optreedt als behandelaar
of docent of beoordelaar, kunnen worden bekritiseerd omdat de
(verwachtingen van de) onderzoeker de uitkomsten kunnen beïnvloeden,
waardoor de constructvaliditeit van de onafhankelijke variabele wordt
bedreigd. Onderzoekers kunnen zich wel verweren tegen deze `experimenter
bias'. In het Head Turn Preference Paradigm
(voorbeeld 5.6), bijvoorbeeld, is het gebruikelijk dat de
experimentator niet weet uit welke groep een proefpersoon afkomstig is,
en dat de experimentator niet hoort welk geluidsbestand wordt aangeboden
\citep[p.74]{John10}.

\begin{enumerate}
\def\labelenumi{\arabic{enumi}.}
\setcounter{enumi}{2}
\item
  Een derde bedreiging van de constructvaliditeit kan samengevat
  worden onder de term \emph{motivatie}. Aan de bedreiging van de validiteit
  door motivatie zitten tenminste twee kanten. Als (ten minste) één van de
  condities in een onderzoek erg belastend of vervelend is, dan kunnen de
  proefpersonen gedemotiveerd raken en zich minder inspannen bij hun
  taken. Ze presteren dan minder, maar dit is een effect van (gebrek aan)
  motivatie, en niet een direct effect van de onafhankelijke variabele
  (hier: conditie). Het effect hoeft dan niet veroorzaakt te worden door
  de manipulatie van het bedoelde construct, maar door de onbedoelde
  manipulatie van de \emph{motivatie} van de proefpersonen. Ook het omgekeerde
  kan natuurlijk een bedreiging van de constructvaliditeit vormen. Indien
  van één van de condities een extra motiverende werking op de
  proefpersonen heeft, dan kan een eventueel effect toegeschreven worden
  aan motivationele aspecten. Ook dan kan er sprake zijn van een effect
  van een onbedoeld gemanipuleerde variabele.
\item
  Een vierde bedreiging van de validiteit heeft te maken met de keuze
  uit de vele mogelijke waarden van een onafhankelijke variabele, d.w.z.
  de \emph{`dosering'} ervan. Als de onafhankelijke variabele is `het aantal
  keren dat een gedicht ter voorbereiding mag worden doorgelezen', moet de
  onderzoeker bepalen hoeveel keer de proefpersonen het gedicht mogen
  doorlezen: één, twee, drie of meer keren? Als de onafhankelijke
  variabele is `de tijd die de proefpersonen mogen studeren', dan moet de
  onderzoeker kiezen hoe lang de proefpersonen mogen leren: vijf minuten,
  een kwartier, twee uur? De onderzoeker maakt een keuze uit de dosering
  van de onafhankelijke variabele `leertijd'. Op grond van deze dosering
  kan de onderzoeker concluderen dat de afhankelijke variabele niet
  beïnvloed wordt door de onafhankelijke variabele. In feite moet de
  onderzoeker echter concluderen dat er geen verband lijkt tussen de
  \emph{gekozen dosering van} de onafhankelijke variabele, en de afhankelijke
  variabele. Een mogelijk effect wordt verhuld door de keuze van de
  dosering (waarden) van de onafhankelijke variabele.
\end{enumerate}

\begin{center}\rule{0.5\linewidth}{0.5pt}\end{center}

\begin{quote}
\emph{Voorbeeld 5.13}: Als een personenauto en een voetganger botsen,
loopt de voetganger een
risico te overlijden. Dat overlijdensrisico is relatief gering (kleiner
dan 20\%) bij botsingssnelheden tot ca 50~km/u. Als we ons onderzoek naar
het verband tussen botsingssnelheid en overlijdensrisico zouden beperken
tot deze lage `doseringen' van botsingssnelheden, dan zouden we wellicht
concluderen dat de botsingssnelheid géén invloed heeft op het
overlijdensrisico voor de voetganger. Dat zou een foutieve conclusie
zijn (van welk type?), want bij hogere botsingssnelheden neemt het
overlijdensrisico voor de voetganger toe tot bijna 100\%
\citep{Rosen11, SWOV12}.
\end{quote}

\begin{center}\rule{0.5\linewidth}{0.5pt}\end{center}

5. Een vijfde bedreiging van de constructvaliditeit wordt veroorzaakt
door de \emph{sturende werking van de voormeting}. In veel studies wordt de
afhankelijke variabele herhaaldelijk gemeten, zowel voor als na
manipulatie van de afhankelijke variabele: de zgn. voormeting en
nameting. De aard en inhoud van de voormeting kunnen echter sporen
nalaten bij de proefpersoon. Zo kan de proefpersoon zijn onbevangenheid
verliezen, waardoor het effect van de onafhankelijke variabele (bv.
behandeling) wordt verkleind. Een eventueel verschil in scores tussen de
experimentele condities kan dus op meerdere manieren worden verklaard.
De verklaring kan immers liggen in een effect van alleen de
onafhankelijke variabele, maar kan ook liggen in een effect van \emph{de
combinatie van voormeting en onafhankelijke variabele}. Bovendien kan de
afwezigheid van een effect soms worden verklaard door het feit dat een
voormeting is verricht (zie het Solomon vier-groepen-ontwerp, in Hoofdstuk \ref{ch:ontwerp}, voor een onderzoeksontwerp dat
hiermee rekening houdt).

\begin{center}\rule{0.5\linewidth}{0.5pt}\end{center}

\begin{quote}
\emph{Voorbeeld 5.14}: We kunnen de effecten van twee behandelingen vergelijken in een
experiment waarin de deelnemers volgens het toeval in twee groepen
worden ingedeeld. De eerste groep (E) krijgt eerst een voormeting, dan
een behandeling, en dan een nameting. De tweede groep (C) krijgt geen
voormeting, en ook geen behandeling, maar alleen een nameting, die voor
deze groep de enige meting is.
\end{quote}

\begin{quote}
Als we bij de nameting een verschil vinden tussen de twee groepen, dan
is dat niet zonder meer toe te schrijven aan het verschil in
behandeling. Het verschil zou ook, of mede, veroorzaakt kunnen zijn door
de sturende werking van de voormeting, bv als gevolg van de sturende
woordkeuze of zinsbouw van de vragen of opdrachten in de voormeting.
Misschien hebben de deelnemers in groep E iets geleerd in de voormeting,
d.w.z. \emph{niet} in de behandeling, waardoor ze beter of anders presteren
in de nameting dan de deelnemers in groep C.
\end{quote}

\begin{center}\rule{0.5\linewidth}{0.5pt}\end{center}

\begin{enumerate}
\def\labelenumi{\arabic{enumi}.}
\setcounter{enumi}{5}
\tightlist
\item
  Een ander probleem dat van invloed kan zijn op de
  constructvaliditeit is \emph{sociaal wenselijk
  antwoorden}. Dat is niets anders dan dat mensen
  geneigd zijn een antwoord geven, dat in de gegeven sociale situatie
  wenselijk is, en dat hen dus niet in de problemen brengt of tot
  gezichtsverlies leidt. Een voorbeeld kan dit verduidelijken.
\end{enumerate}

\begin{center}\rule{0.5\linewidth}{0.5pt}\end{center}

\begin{quote}
\emph{Voorbeeld 5.15}: Bij peilingen voor verkiezingen zijn respondenten geneigd om sociaal wenselijk te antwoorden, en dat geldt ook voor de vraag of de respondent überhaupt zal gaan stemmen \citep{Karp05}. De neiging tot het sociaal wenselijke antwoord (``ja, ik ga stemmen'') is sterker naarmate respondenten hoger zijn opgeleid, en dus is de overschatting van het opkomst-percentage groter voor hoger-opgeleiden dan voor lager-opgeleiden. Dat heeft weer gevolgen voor de uitslagen van de peilingen van de verschillende partijen, omdat de populariteit van de politieke partijen verschillend is voor kiezers van verschillend opleidingsniveau.
\end{quote}

\begin{center}\rule{0.5\linewidth}{0.5pt}\end{center}

Dit effect heeft mede gezorgd voor de overschatting van het aantal
Clinton-stemmers, en onderschatting van het aantal Trump-stemmers, bij
de peilingen voorafgaand aan de Amerikaanse presidentsverkiezing in
2016.

\begin{enumerate}
\def\labelenumi{\arabic{enumi}.}
\setcounter{enumi}{6}
\tightlist
\item
  Een laatste probleem met betrekking tot de constructvaliditeit kan
  aangeduid worden als: een \emph{beperkte generaliseerbaarheid} over
  constructen. Bij de presentatie van onderzoeksresultaten worden
  regelmatig opmerkingen gemaakt als: `Ja, ik ben het eens met uw
  conclusie dat X van invloed is op Y, maar hoe zit het met\ldots{}'. Op de
  puntjes kan dan van alles ingevuld worden: de toepasbaarheid bij andere
  doelgroepen, of in andere genres, of in andere talen, etc. Deze aspecten
  zijn weliswaar van belang, maar spelen in het onderzoek zelf niet direct
  een rol: we hebben het onderzoek immers uitgevoerd met een bepaalde
  selectie van doelgroep, genre, talen, etc.
\end{enumerate}

Toch bevelen we wel aan om zulke vragen over generaliseerbaarheid onder
ogen te zien. Zijn de conclusies eveneens van toepassing op een andere
doelgroep of taal? Waarom wel of niet? Welke andere factoren zouden de
generalisatie kunnen beïnvloeden? Zou een gunstig effect voor de ene
groep of taal ook kunnen uitpakken als een ongunstig effect voor een
andere groep of taal die buiten het onderzoek is gevallen?

\hypertarget{sec:externevaliditeit}{%
\section{Externe validiteit}\label{sec:externevaliditeit}}

Op basis van de gegevens die zijn verzameld kan een onderzoeker --- als
het goed is --- de conclusie trekken: \emph{in dit onderzoek geldt dat\ldots{}}.
Het is echter zelden de bedoeling van een onderzoeker om conclusies te
trekken die alleen gelden voor één onderzoek. Een onderzoeker wil niet
aantonen dat tweetaligheid een gunstige invloed heeft op de
taalontwikkeling \emph{van de steekproef van onderzochte kinderen}. Een
onderzoeker wil conclusies trekken als: tweetaligheid heeft een gunstige
invloed op de taalontwikkeling \emph{van kinderen}. De onderzoeker wil
generaliseren. In het dagelijks leven doen we hetzelfde: we proeven één
hapje soep uit een hele pan, en op grond daarvan doen we een uitspraak
over die hele pan soep. We gaan er van uit dat onze bevindingen op basis
van dat ene hapje gegeneraliseerd mogen worden naar de hele pan, en dat
het niet nodig is om de hele pan leeg te eten voordat we er een
uitspraak over kunnen doen.

De vraag of een onderzoeker de resultaten kan en mag generaliseren is de
vraag naar de \emph{externe validiteit} van een onderzoek \citep{SCC02}.
Generalisatie heeft betrekking op o.a.

\begin{itemize}
\item
  eenheden: zijn de resultaten ook geldig voor andere elementen (bv.
  scholen, personen, teksten) uit de populatie, die niet aan het
  onderzoek deelnamen?
\item
  behandelingen: zijn de resultaten ook geldig voor andere
  behandelingen die lijken op de specifieke condities in dit
  onderzoek?
\item
  situaties: zijn de resultaten ook geldig buiten de specifieke
  context van dit onderzoek?
\item
  tijden: zijn de resultaten van dit onderzoek ook geldig op andere
  tijdstippen?
\end{itemize}

Bij externe validiteit maken we een onderscheid tussen (1) de
generalisatie \emph{naar} een beoogde specifieke doelgroep, situatie en tijd,
en (2) de generalisatie \emph{over} andere doelgroepen, situaties en tijden.
Het generaliseren \emph{naar} en \emph{over} zijn twee aspecten van de externe
validiteit die goed uit elkaar gehouden moeten worden. Het generaliseren
\emph{naar} een doelgroep of populatie, van personen en vaak ook van
taalmateriaal, heeft te maken met de representativiteit van de gebruikte
steekproef; in hoeverre is de steekproef een goede afspiegeling van de
populatie (van personen, van woorden, van relevante mogelijke zinnen)?
Het generaliseren \emph{naar} is dus direct verbonden met het onderzoeksdoel;
pas als er gegeneraliseerd kan worden naar gedefinieerde populaties kan
een onderzoeksdoel bereikt zijn. Het generaliseren \emph{over} doelgroepen
heeft te maken met de mate waarin de geformuleerde conclusies geldig
zijn voor te onderscheiden deel-populaties. We illustreren dit met een
voorbeeld.

\begin{center}\rule{0.5\linewidth}{0.5pt}\end{center}

\begin{quote}
\emph{Voorbeeld 5.16}: \citep{LevA10} onderzochten of luisteraars minder geloof hechten
aan sprekers met een
vreemd buitenlands accent in de uitspraak van het Engels. Voor de
stimuli lieten ze zinnen uitspreken (bv. \emph{A giraffe can hold more water
than a camel}) door verschillende sprekers zonder enig accent, met licht
accent, of met sterk accent. Luisteraars (moedertaal-sprekers van het
Engels) gaven aan in welke mate ze dachten dat de gesproken zin waar
was. De resultaten lieten zien dat de luisteraars de zinnen beoordeelden
als minder waar, als de zin was gesproken door een spreker met een
vreemd buitenlands accent.
\end{quote}

\begin{center}\rule{0.5\linewidth}{0.5pt}\end{center}

We mogen aannemen dat deze uitkomst gegeneraliseerd kan worden \emph{naar} de
beoogde doelgroep, nl. alle moedertaal-luisteraars van het Amerikaans
Engels. Deze generalisatie kan worden gemaakt ondanks de mogelijkheid
dat verschillende luisteraars misschien in verschillende mate beïnvloed
werden door het buitenlandse accent van de spreker.

Wellicht zou een latere analyse kunnen laten zien dat er verschil is
tussen vrouwelijke en mannelijke luisteraars. Het is denkbaar dat
vrouwen en mannen verschillen in hun gevoeligheid voor het accent van de
spreker. Zo'n (denkbeeldige) uitkomst zou laten zien dat er niet
gegeneraliseerd mag worden \emph{over} deel-populaties binnen de doelgroep,
hoewel er wel gegeneraliseerd kon worden \emph{naar} de doelgroep.

In het (toegepast) taalwetenschappelijk onderzoek proberen onderzoekers
doorgaans om \emph{tegelijkertijd} te generaliseren naar \emph{twee} populaties
van eenheden, nl. van personen (c.q. scholen of families) \emph{en stimuli}
(woorden, zinnen, teksten, enz). We willen aannemelijk maken dat de
resultaten niet alleen geldig zijn voor de onderzochte taalgebruikers,
maar ook voor andere taalgebruikers. Tegelijkertijd willen we ook
aannemelijk maken dat de resultaten niet alleen geldig zijn voor de
onderzochte stimuli, maar ook voor andere vergelijkbaar taalmateriaal
waaruit de steekproef van stimuli is getrokken. Die gelijktijdige
generalisatie vereist een complex onderzoeksontwerp, doordat er
herhaalde observaties zijn binnen proefpersonen (meerdere oordelen van
eenzelfde proefpersoon) en binnen stimuli (meerdere oordelen over
dezelfde stimulus). Stimuli, proefpersonen en condities worden
vervolgens slim gecombineerd om de interne validiteit zo goed mogelijk
te beschermen. Uiteraard vereist de generalisatie naar ander
taalmateriaal wel, dat de stimuli willekeurig zijn geselecteerd uit de
(soms oneindig grote) populatie van al het mogelijke taalmateriaal (zie
Hoofdstuk \ref{ch:steekproeftrekking}).

\hypertarget{ch:ontwerp}{%
\chapter{Ontwerp}\label{ch:ontwerp}}

\hypertarget{sec:ontwerp-inleiding}{%
\section{Inleiding}\label{sec:ontwerp-inleiding}}

Veel van de problemen met validiteit, die we bespraken in
Hoofdstuk \ref{ch:validiteit}, kunnen voorkomen worden door goede gegevens
op een goede manier te verzamelen. Het \emph{ontwerp} (Eng. ``design'') van een
onderzoek geeft aan volgens welk schema of plan de gegevens verzameld
zullen worden. Als we een goed en sterk ontwerp gebruiken, dan kunnen we
daarmee al veel mogelijke bedreigingen voor de validiteit neutraliseren.
Dat maakt ons onderzoek sterker. Het is dus raadzaam om een
onderzoeksontwerp vooraf heel goed te doordenken! Uiteraard moet het
ontwerp nauw aansluiten bij de vraagstelling: de gegevens uit het
onderzoek moeten de onderzoeker immers in staat stellen om een valide
antwoord te geven op de onderzoeksvraag.

De onderzoeksontwerpen die we in dit hoofdstuk bespreken vormen slechts
een beperkte selectie uit de mogelijke ontwerpen. Sommige ontwerpen
bespreken we vooral om aan te geven wat er mis kan gaan bij een ``zwak''
ontwerp; andere ontwerpen zijn juist populair omdat ze relatief ``sterk''
onderzoek mogelijk maken.

Een onderzoeksontwerp is opgebouwd uit verschillende elementen:

\begin{itemize}
\tightlist
\item
  \emph{tijd}, meestal afgebeeld als verstrijkend in de leesrichting. De
  tijdsvolgorde is van belang om een causaal verband vast te stellen:
  eerst de oorzaak, daarna het gevolg
  (§\ref{sec:causaliteit}). De tijdsvolgorde is echter een
  noodzakelijke voorwaarde, maar niet een voldoende voorwaarde om een
  causaal verband vast te stellen. Anders gezegd, ook als het gevolg
  (bv. herstel) inderdaad optreedt na de oorzaak (bv. behandeling),
  dan houdt dat \emph{niet} in dat de behandeling ook inderdaad het herstel
  heeft veroorzaakt. Misschien is het herstel wel spontaan opgetreden,
  of is het herstel het gevolg van een andere oorzaak waar het
  onderzoek niet op gericht was.
\end{itemize}

\begin{center}\rule{0.5\linewidth}{0.5pt}\end{center}

\begin{quote}
\emph{Voorbeeld 6.1}: Stel je Gus voor: als iemand last heeft van brandnetel-uitslag, of een insectenbeet, of eczeem, of een blauwe plek, dan spuit Gus er wat Glassex op --- en na een paar dagen is de aandoening verdwenen. Gus is ervan overtuigd dat zijn Glassex-behandeling de oorzaak is van de genezing. Maar dit is een misvatting die bekend staat als ``post hoc ergo propter hoc'' (na iets dus als gevolg van iets). De aandoening zou hoogstwaarschijnlijk ook goed zijn genezen zonder de Glassex-behandeling. De genezing bewijst dus niet dat de Glassex-behandeling noodzakelijk is. (Dit voorbeeld is ontleend aan de speelfilm \emph{My Big Fat Greek Wedding}, 2004).
\end{quote}

\begin{center}\rule{0.5\linewidth}{0.5pt}\end{center}

\begin{itemize}
\item
  \emph{groepen} van eenheden (bv. proefpersonen), doorgaans correspondeert
  een groep met een regel in het ontwerp.
\item
  \emph{behandeling}, meestal afgebeeld als \texttt{X}. Een behandeling kan ook
  bestaan uit het ontbreken van een behandeling (``control''), of uit
  het aanbieden van de niet-experimentele, gebruikelijke behandeling
  (``usual care'').
\item
  \emph{observatie}, meestal afgebeeld als \texttt{O}.
\item
  de \emph{toewijzing van proefpersonen} aan groepen of behandelcondities
  kan op verschillende manieren gebeuren. Meestal doen we dat aselect
  (willekeurig, at random, hieronder aangegeven met \texttt{R}), omdat
  daarmee de validiteit meestal het beste beschermd wordt.
\end{itemize}

\hypertarget{sec:tussenbinnenproefpersonen}{%
\section{Tussen of binnen ?}\label{sec:tussenbinnenproefpersonen}}

Voor het
onderzoeksontwerp is het van groot belang of een onafhankelijke variabele
gevarieerd wordt \emph{tussen proefpersonen} of \emph{binnen proefpersonen}. Voor
veel taalkundig onderzoek, waarbij meerdere teksten of zinnen of woorden
worden aangeboden als stimuli, geldt hetzelfde voor het onderscheid
\emph{tussen stimuli} of \emph{binnen stimuli}.

Individuele variabelen van de proefpersonen, zoals diens geslacht
(man, vrouw) of meertaligheid, kunnen normaliter alleen variëren
tussen proefpersonen: eenzelfde proefpersoon kan niet meedoen aan beide
geslachts-groepen van een onderzoek, en ééntalige proefpersonen kunnen
niet meedoen in de groep van meertalige proefpersonen. Maar bij andere
variabelen, die betrekking hebben op de wijze waarop stimuli worden
verwerkt, is dat wel mogelijk. Dezelfde proefpersoon kan schrijven met
zijn linkerhand en met zijn rechterhand, of kan geobserveerd worden
voorafgaand aan en volgend op een behandeling. De onderzoeker moet dan
in het onderzoeksontwerp kiezen op welke wijze de behandelingen en
observaties worden gecombineerd. We komen daarop terug in
§\ref{sec:afhankelijkegroepen}.

\hypertarget{het-one-shot-single-case-ontwerp}{%
\section{Het one-shot single-case-ontwerp}\label{het-one-shot-single-case-ontwerp}}

Dit is een zwak ontwerp, waarbij er slechts éénmaal observaties worden
gedaan, na een behandeling. Dit onderzoeksontwerp heeft het volgende
schema:

\begin{verbatim}
  X   O
\end{verbatim}

We zouden bijvoorbeeld kunnen tellen, voor alle eindwerkstukken van
studenten van een bepaalde opleiding van een bepaald cohort, hoeveel
fouten (van een bepaald type) er in die eindwerkstukken voorkomen. Dat
is wel een beetje interessant, maar wetenschappelijk zijn deze gegevens
echter van weinig waarde. Er kan geen enkele vergelijking gemaakt worden
met andere gegevens (van andere studenten, en/of andere werkstukken van
dezelfde studenten). Het is niet mogelijk om een valide conclusie te
trekken over mogelijke effecten van de ``behandeling'' (studie, \texttt{X}) op de
observaties (aantal fouten, \texttt{O}).

Soms worden de gegevens uit een one-shot-single-case-onderzoek
geforceerd vergeleken met andere gegevens, bijvoorbeeld met
norm-resultaten voor een grote controlegroep. Stel je voor dat we willen
onderzoeken of een nieuwe methode van taalonderwijs leidt tot betere
taalvaardigheid in de doeltaal. Na een cursus met de nieuwe lesmethode
meten we de taalvaardigheid, en vergelijken die met de eerder
gepubliceerde resultaten van een controlegroep die de traditionele
lesmethode heeft gebruikt. Deze aanpak wordt veelvuldig toegepast, maar
er zijn desalniettemin diverse factoren die de validiteit bedreigen (zie
§\ref{sec:internevaliditeit}): o.a. geschiedenis (de nieuwe
proefpersonen hebben een andere geschiedenis en levensloop gehad dan de
controlegroep uit het verleden), rijping (de nieuwe proefpersonen zijn
misschien verder of minder ver ontwikkeld dan de controlegroep),
instrumentatie (de toets is mogelijk niet even geschikt voor personen
onderwezen met de nieuwe lesmethode als met de traditionele methode), en
uitval (de uitval van proefpersonen voorafgaand aan de observatie is
niet bekend, noch voor de traditionele methode noch voor de nieuwe
methode).

\begin{center}\rule{0.5\linewidth}{0.5pt}\end{center}

\begin{quote}
\emph{Voorbeeld 6.2}: Een interviewer kan zgn. `gesloten' vragen stellen met slechts enkele
mogelijke antwoorden (\emph{welk van de drie groentesoorten vind je het
lekkerst, doperwtjes of sperziebonen of broccoli?}), of `open' vragen
waarin de mogelijke antwoorden niet worden beperkt door de vraagstelling
(\emph{welke groente vind je het lekkerst?}). Er is ook een derde categorie,
nl. open vragen met voorbeeld-antwoorden (\emph{welke groente vind je het
lekkerst, bijvoorbeeld doperwtjes of sperziebonen of\ldots?}). Het is
echter niet duidelijk of deze voorbeeld-antwoorden wel of niet een
sturend effect hebben, d.w.z. of ze meer vergelijkbaar zijn met gesloten
of met open vragen. \citep{Houtk91} bestudeerde opgenomen gesprekken tussen
artsen en hun patiënten. De artsen stelden regelmatig open vragen met
voorbeeldantwoorden. Meestal bleken de patiënten zo'n vraag \emph{niet} als
sturend op te vatten; zij vatten de vraag vooral op als een verzoek om
te vertellen.
\end{quote}

\begin{quote}
Dit onderzoek is te beschouwen als een one-shot-single-case-ontwerp,
zonder vergelijking met gegevens uit andere condities. De conclusies
zijn weliswaar gebaseerd op empirische observaties, maar we weten niet
wat de geïnterviewde geantwoord zou hebben als de vraag anders gesteld
zou zijn.
\end{quote}

\begin{center}\rule{0.5\linewidth}{0.5pt}\end{center}

Ondanks al deze bezwaren kan een one-shot-case onderzoek wel van nut
zijn in de observatiefase van de empirische cyclus, wanneer het gaat om
het opdoen van ideeën en het formuleren van (globale) hypothesen, die
later goed getoetst kunnen worden.

\hypertarget{het-uxe9uxe9n-groep-voormeting-nameting-ontwerp}{%
\section{Het één-groep-voormeting-nameting-ontwerp}\label{het-uxe9uxe9n-groep-voormeting-nameting-ontwerp}}

Bij het één-groep-voormeting-nameting-ontwerp worden gegevens verzameld
van één groep. Op het eerste tijdstip (meestal aangeduid als \texttt{T1}, maar
soms als \texttt{T0}) wordt een eerste meting uitgevoerd (voormeting, \texttt{O1}),
vervolgens wordt de groep aan de experimentele behandeling blootgesteld,
en tenslotte wordt op een later tijdstip (\texttt{T2}) een tweede meting
uitgevoerd (nameting, \texttt{O2}). In schema ziet een
één-groep-voormeting-nameting-ontwerp er als volgt uit:

\begin{verbatim}
  O1   X   O2
\end{verbatim}

De behandeling \texttt{X} varieert dus niet: iedereen krijgt dezelfde
behandeling, want er is slechts één groep. Het tijdstip van de meting,
meestal aangeduid als voormeting \texttt{T0} (\texttt{O1}) en nameting \texttt{T1} (\texttt{O}),
varieert binnen proefpersonen.

Dit ontwerp is over het algemeen beter beter dan het vorige
one-shot-case-ontwerp én beter dan helemaal geen gegevens. Toch
beschouwen wij het als een zwak onderzoeksontwerp, omdat diverse
bedreigingen van de validiteit niet goed ondervangen worden (zie
§\ref{sec:internevaliditeit}). Een eventueel verschil tussen \texttt{O2}
en \texttt{O1} kan niet uitsluitend toegeschreven worden aan de tussenliggende
behandeling \texttt{X}: dit effect kan ook het gevolg zijn van rijping (de
verbetering is het gevolg van rijping van de proefpersonen) of van
geschiedenis (de verbetering is het gevolg van een of meerdere
gebeurtenissen anders dan \texttt{X} die zijn opgetreden tussen de tijdstippen
van \texttt{O1} en \texttt{O2}). Als de behandeling \texttt{X} of de nameting \texttt{O2}
afhankelijk is van de score op de voormeting \texttt{O1}, dan kan ook de
regressie naar het gemiddelde de validiteit bedreigen. Kortom, aan dit
onderzoeksontwerp kleven diverse bezwaren, omdat de hypothese over het
effect van de onafhankelijke variabele niet zonder meer op valide wijze
beantwoord kan worden.

\hypertarget{sec:voormeting-nameting-controlegroep-ontwerp}{%
\section{Het voormeting-nameting-controlegroep-ontwerp}\label{sec:voormeting-nameting-controlegroep-ontwerp}}

De bovengenoemde problemen kunnen voor een deel ondervangen worden door
een controlegroep toe te voegen aan het ontwerp; we krijgen dan een
voormeting-nameting-controlegroep-ontwerp. Er zijn dan dus twee groepen
van elementen (proefpersonen). In schema wordt dat weergegeven door twee
regels. Dit onderwerp wordt zeer vaak gebruikt. Waar mogelijk proberen
onderzoekers de twee groepen zo vergelijkbaar mogelijk te maken, door de
proefpersonen aselect (at random, willekeurig, volgens toeval) toe te
wijzen aan de twee groepen. In schema ziet dit model er als volgt uit
(de \texttt{R} staat voor random toewijzing aan de twee groepen):

\begin{verbatim}
  R   O1   X   O2
  R   O3       O4
\end{verbatim}

Dit onderzoeksontwerp is populair, omdat het veel mogelijke bedreigingen
van de interne validiteit kan ondervangen (zie
§\ref{sec:internevaliditeit}). Het effect van de manipulatie of
behandeling (\texttt{X}) wordt geëvalueerd door een vergelijking van de twee
verschillen \texttt{(O2-O1)} en \texttt{(O4-O3)}. Dit onderzoeksontwerp heeft
eigenlijk niet één maar twee onafhankelijke variabelen, die van invloed
kunnen zijn op de metingen: (1) de manipulatie of behandeling, \texttt{X} of
niet-\texttt{X}, variërend tussen proefpersonen, en (2) het tijdstip van de
meting, meestal aangeduid als voormeting \texttt{T0} en nameting \texttt{T1},
variërend binnen proefpersonen.

In dit ontwerp wordt wel rekening gehouden met de effecten van
geschiedenis, althans voor zover die effecten voor beide groepen in
gelijke mate zijn opgetreden. Er wordt geen rekening gehouden met
gebeurtenissen die slechts één van de groepen (condities) hebben
beïnvloed. Als er wel zo'n gebeurtenis is geweest voor de ene groep en
niet voor de andere groep, dan kan dat verschil in geschiedenis dus ook
verantwoordelijk zijn voor een ongelijk verschil tussen voormeting en
nameting in de ene groep ten opzichte van de andere groep.

De bedreiging van de interne validiteit door rijping kan in dit
onderzoeksontwerp makkelijk opgevangen worden. Een effect van rijping
komt immers naar verwachting in beide groepen in gelijke mate tot
uiting, en kan daarom niet van invloed zijn op het verschil tussen
\texttt{(O2-O1)} en \texttt{(O4-O3)}. Natuurlijk gaan we er hierbij vanuit dat de
voormetingen voor de twee groepen resp. de nametingen voor de twee
groepen op hetzelfde tijdstip zijn afgenomen.

Ook een storend effect van instrumentatie wordt geneutraliseerd, als aan
de eisen voor vergelijkbare afnamecondities voldaan wordt, en als
gemeten wordt met hetzelfde instrument, zoals hetzelfde apparaat of
computerprogramma of gedrukte toets. Wanneer echter observatoren of
beoordelaars ingeschakeld moeten worden, zoals bij onderzoek naar de
schrijfvaardigheid, dan wordt de instrumentatie een moeilijker factor.
Het is dan van groot belang dat deze beoordelaars \emph{niet} weten door
welke proefpersonen of onder welke conditie de te beoordelen producten
of responsies tot stand zijn gekomen. Anders zouden hun verwachtingen
(onbewust en onbedoeld) een rol kunnen spelen bij het tot stand komen
van hun oordeel. In dat geval zou niet een effect van de onafhankelijke
variabele aangetoond worden, maar een effect van de vooringenomenheid
van beoordelaars.

Ook het probleem van regressie naar het gemiddelde speelt in dit ontwerp
een kleinere rol. Indien de proefpersonen aselect zijn toegewezen aan de
twee groepen, én de gegevens van alle proefpersonen gelijktijdig in een
analyse betrokken worden, dan speelt regressie naar het gemiddelde geen
enkele rol. In beide groepen treedt immers regressie naar het gemiddelde
op, en naar verwachting in gelijke mate, waardoor dat niet van invloed
is op de analyse van het verschil tussen \texttt{(O2-O1)} en \texttt{(O4-O3)}.

Selectie van proefpersonen wordt in dit onderzoeksontwerp uitgesloten
door de steekproef van proefpersonen aselect te kiezen uit de populatie,
en door daarna de proefpersonen wederom aselect toe te wijzen aan de
twee groepen of condities. Natuurlijk geldt hier de wet van de grote
getallen: als een grotere steekproef aselect wordt gesplitst in twee
groepen, dan is ook de kans groter dat de twee groepen gelijkwaardig
zijn, ten opzichte van een kleinere steekproef.

Uitval kan wel degelijk een oorzaak zijn voor een verschil tussen
\texttt{(O2-O1)} en \texttt{(O4-O3)}. Deze voor de validiteit bedreigende factor
is moeilijk te beheersen. We kunnen proefpersonen immers niet dwingen om
aan een onderzoek te blijven meewerken, of om niet te verhuizen, of niet
te overlijden. Uitval kan dus een probleem zijn, zeker wanneer er een
verschil in uitval is tussen de twee groepen of condities. Het is goed
gebruik om de uitval te melden in het onderzoeksverslag, en de mogelijke
gevolgen ervan te bespreken.

Met dit voormeting-nameting-controlegroep-ontwerp kunnen de
verschillende factoren die de interne validiteit bedreigen dus redelijk
goed beheerst worden. Maar hoe zit het met de constructvaliditeit (zie
§\ref{sec:constructvaliditeit})? Deze bedreigingen hebben we niet
eerder aangeroerd bij het one-shot-case-ontwerp en het
één-groep-voormeting-nameting-ontwerp, omdat bij deze
onderzoeksontwerpen de interne validiteit al twijfelachtig was.

Niet alle aspecten van de constructvaliditeit hebben echter repercussies
op het onderzoeksontwerp. Sommige aspecten met betrekking tot de wijze
van operationalisatie, zoals convergente en divergente validiteit, zijn
niet relevant voor de keuze van het onderzoeksontwerp. Maar andere
aspecten zijn wel relevant: de verwachtingen van de onderzoeker,
aandacht, motivatie, en de sturende werking van de voormeting.

Het voormeting-nameting-controlegroep-onderzoeksontwerp biedt voor geen
van deze vier bedreigingen van de constructvaliditeit adequate
waarborgen. De \emph{verwachtingen van de onderzoeker} kunnen in zowel de
experimentele als de controle-conditie een (verschillende) rol spelen,
omdat in beide condities op twee tijdstippen gemeten wordt. Bovendien
kan een eventueel verschil tussen \texttt{(O2-O1}) en \texttt{(O4-O3)} ook te
wijten zijn aan de (extra) \emph{aandacht} die aan de experimentele conditie
besteed is: het zgn. Hawthorne-effect (zie Voorbeeld 5.10 in Hoofdstuk \ref{ch:validiteit}). Dit effect speelt vooral een rol als de
proefpersonen in de ene conditie (groep) meer aandacht krijgen dan in de
andere, zoals in het onderstaande voorbeeld.

Een derde bedreiging van de constructvaliditeit van een onderzoek is de
\emph{motivatie}. Soms kan één van de condities zo demotiverend werken dat de
proefpersonen in deze conditie niet meer serieus aan het onderzoek
meewerken. Net als bij aandacht gaat het niet zozeer om de
aantrekkelijkheid van één van de condities, maar om de verschillen in
aantrekkelijkheid tussen de onderzoekscondities.

In het voormeting-nameting-controlegroep-ontwerp kan de
constructvaliditeit tevens bedreigd worden door de \emph{sturende werking van
de voormeting}. Door een voormeting (\texttt{O1} en \texttt{O3}) kunnen de
proefpersonen zich van bepaalde aspecten van het onderzoek (meer) bewust
worden, waardoor zij zich daarna niet meer als naïeve proefpersonen
gedragen. De voormeting kan dan als een soort manipulatie beschouwd
worden (zie Voorbeeld 6.3 hieronder).

\hypertarget{sec:solomon-vier-groepen-ontwerp}{%
\section{Het Solomon-vier-groepen-ontwerp}\label{sec:solomon-vier-groepen-ontwerp}}

Het Solomon-vier-groepen-ontwerp wordt veel minder vaak gebruikt dan het
voormeting-nameting-controlegroep-ontwerp. Toch verdient dit ontwerp
duidelijk de voorkeur boven het
voormeting-nameting-controlegroep-ontwerp. Met name bedreigingen van de
constructvaliditeit worden in dit ontwerp beter onder controle gehouden.

In het Solomon-vier-groepen-ontwerp worden vier condities onderscheiden.
De proefpersonen worden aselect toegewezen aan een van deze vier
condities. In de eerste twee condities wordt eerst een voormeting
gehouden, waarna één van de groepen aan de experimentele behandeling
blootgesteld wordt. Daarna volgt voor beide groepen een nameting. Tot
zover is het Solomon-vier-groepen-ontwerp dus exact gelijk aan het
voormeting-nameting-controlegroep-ontwerp. In de derde en vierde
conditie wordt echter géén voormeting gehouden. In de ene conditie
worden de proefpersonen wel aan de experimentele behandeling
blootgesteld, maar in de andere conditie niet. Tot slot maken deze beide
groepen weer een nameting. Schematisch kan het
Solomon-vier-groepen-ontwerp als volgt weergegeven worden:

\begin{verbatim}
  O1   X   O2
  O3       O4
       X   O5
           O6
\end{verbatim}

Het Solomon-vier-groepen-ontwerp is dus een uitbreiding van het
voormeting-nameting-controlegroep-ontwerp met twee groepen, die niet
meedoen met de voormeting. Door de twee extra condities zonder
voormeting kan rekening gehouden worden met de sturende werking van de
voormeting: die sturende werking is immers niet aanwezig in de
derde en vierde groep. Bovendien wordt het effect van de manipulatie van
de onafhankelijke variabele \texttt{X} verschillende malen getoetst, in de vier
vergelijkingen van \texttt{O2} en \texttt{O1}, \texttt{O2} en \texttt{O4}, \texttt{O5} en \texttt{O6}, en
\texttt{(O2-O1)} en \texttt{(O4-O3)}. Het effect van de mogelijk sturende
voormeting wordt getoetst in de twee vergelijkingen van \texttt{O2} en \texttt{O5},
en \texttt{O4} en \texttt{O6}. We kunnen dus in één onderzoek zowel een effect van
de behandeling als van de voormeting aantonen. Daarvoor moeten we echter
wel twee extra groepen proefpersonen inzetten (ten opzichte van het
voormeting-nameting-controlegroep-ontwerp).

\begin{center}\rule{0.5\linewidth}{0.5pt}\end{center}

\begin{quote}
\emph{Voorbeeld 6.3}: In een onderzoek \citep{AHW00} werd het effect
onderzocht van een gewenningstraining (\texttt{X}) op de angst om in het
openbaar te spreken. Spreekangst werd gemeten door de proefpersoon eerst
een toespraak te laten houden, en daarna twee vragenlijsten over
spreekangst te laten invullen. Bij elkaar vormt dat één meting. De
gewenningstraining voor de ene groep bestond uit het bekijken van een
trainingsvideo van ca 20 minuten; de tweede groep kreeg in plaats
daarvan een pauze van dezelfde duur. Het onderzoek gebruikte een
Solomon-vier-groepen-ontwerp om een mogelijk sturende werking van de
voormeting te kunnen onderzoeken. Het is immers mogelijk dat de
voormeting (waarvan een spreekbeurt deel uitmaakt) zelf al een training
vormt voor de proefpersonen, zodat de gunstige effecten na de
``behandeling'' \texttt{X} (gewenningstraining) niet toegeschreven kunnen worden
aan die behandeling, maar (mede) aan die voormeting. De resultaten
lieten echter zien dat de gewenningstraining inderdaad een sterk
(gunstig) effect had op de spreekangst, en dat de voormeting alleen (dus
zonder behandeling) geen enkel effect had op de mate van spreekangst bij
de proefpersonen.
\end{quote}

\begin{center}\rule{0.5\linewidth}{0.5pt}\end{center}

\hypertarget{het-nameting-controlegroep-ontwerp}{%
\section{Het nameting-controlegroep-ontwerp}\label{het-nameting-controlegroep-ontwerp}}

In heel veel onderzoekingen wordt een voormeting uitgevoerd, omdat de
onderzoekers willen onderbouwen dat de twee (of meer) onderzoeksgroepen
niet van elkaar verschillen bij de aanvang van het onderzoek. Toch is
die voormeting niet een essentieel onderdeel van een adequaat
onderzoeksontwerp. Als de groepen voldoende groot zijn, en als de
proefpersonen (of andere onderzoekseenheden) geheel volgens het toeval
zijn toegewezen aan de groepen, dan zijn de twee groepen op statistische
gronden al goed vergelijkbaar. Als we bijvoorbeeld 100 proefpersonen
geheel volgens het toeval verdelen over 2 groepen, dan is de kans
buitengewoon klein dat de twee groepen zouden verschillen in de
voormeting. In veel van dit soort gevallen kan dan ook volstaan worden
met een nameting-controlegroep-ontwerp, schematisch weergegeven als
volgt:

\begin{verbatim}
      X   O5
          O6
\end{verbatim}

Dit ontwerp is echter alleen adequaat indien de groepen voldoende groot
zijn, en indien de proefpersonen aselect zijn toegewezen aan de
condities. Als dat niet mogelijk is, dan voldoet dit ontwerp ook niet.

\begin{center}\rule{0.5\linewidth}{0.5pt}\end{center}

\begin{quote}
\emph{Voorbeeld 6.4}: In vervolg op het onderzoek van \citep{Houtk91}
(Voorbeeld 6.2)
onderzochten \citep{Wijff92} in hoeverre vragen met
danwel zonder voorbeeldantwoorden in een mondeling (telefonisch)
vraaggesprek als sturend werden opgevat. Hiertoe werden vijf vragen
geconstrueerd over criminaliteit. Van elke vraag werden twee versies
geconstrueerd: één met en één zonder voorbeeldantwoorden. Aan elke
respondent (van een steekproef van 50) werden twee of drie vragen
gesteld met voorbeeldantwoorden, en twee of drie vragen zonder
voorbeeldantwoorden. De toedeling van vragen met en zonder
voorbeeldantwoorden was gerandomiseerd, en daardoor mogen we aannemen
dat de groep respondenten die een bepaalde vraag met voorbeeldantwoorden
kreeg niet verschilt van de groep respondenten die dezelfde vraag zonder
voorbeeldantwoorden kreeg. Als beide groepen dezelfde (soort) antwoorden
geven, dan hebben de voorbeeldantwoorden blijkbaar geen sturend effect,
maar indien de respondenten vaak antwoorden met de gegeven
voorbeeldantwoorden dan hebben de voorbeelden blijkbaar wel een sturend
effect. Bij analyse bleek dat voor vier van de vijf vragen een dergelijk
sturend effect inderdaad optrad.
\end{quote}

\begin{center}\rule{0.5\linewidth}{0.5pt}\end{center}

De twee onderzoeken van \citep{Houtk91} en \citep{Wijff92} illustreren hoe wetenschappelijke kennis zich
ontwikkelt. \citep{Houtk91} constateert dat de vakliteratuur zich voornamelijk bezig
heeft gehouden met schriftelijke interviews, en vraagt zich af of de
effecten bij mondelinge `face-to-face'-interviews hetzelfde zijn. Zij
concludeert dat voorbeeldantwoorden niet sturend werken in deze
`face-to-face'-gesprekken. \citep{Wijff92} onderzoeken dezelfde hypothese in een
experiment, met mondelinge interviews per telefoon, en zij concluderen
dat voorbeeldantwoorden wel degelijk sturend werken in deze telefonische
gesprekken.

\hypertarget{sec:ontwerp.factorieel}{%
\section{Factoriële ontwerpen}\label{sec:ontwerp.factorieel}}

Tot nog toe hebben we het gehad over experimentele onderzoeksontwerpen
waarbij één onafhankelijke variabele gemanipuleerd wordt. Veel
onderzoekers zijn echter (ook) geïnteresseerd in de effecten van de
gelijktijdige manipulatie van meerdere onafhankelijke variabelen.
Ontwerpen waarin meerdere factoren gelijktijdig variëren noemen we
factoriële ontwerpen. We kwamen deze al tegen bij het
voormeting-nameting-controlegroep-ontwerp
(§\ref{sec:voormeting-nameting-controlegroep-ontwerp}
waar zowel de tijd als de behandeling varieert.

\begin{center}\rule{0.5\linewidth}{0.5pt}\end{center}

\begin{quote}
\emph{Voorbeeld 6.5}: \citep{Drake03}
onderzochten de waarneming van muzikale structuur. Die waarneming kan
indirect gemeten worden door de luisteraar te vragen om mee te tikken
met de muziek. Als een luisteraar de muzikale structuur niet begrijpt of
herkent, is hij geneigd mee te tikken met iedere tel (analytisch
luisteren). Naarmate een luisteraar de muzikale structuur beter begrijpt
en herkent, is hij meer geneigd om mee te tikken met hogere nivo's
(synthetisch of voorspellend luisteren): hij tikt dan niet bij iedere
tel, maar bv \(1\times\) per maat of per muzikale frase. De tijdsafstand
(`inter-onset-interval', IOI) tussen de tikken vormt aldus een indicatie
van de waargenomen muzikale structuur. Aan het onderzoek van deden twee
groepen luisteraars mee, in Frankrijk en in Tunesië\footnote{Merk op dat proefpersonen niet aselect zijn toegewezen aan een van
  deze groepen, en dat dit dus strikt genomen een quasi-experimenteel onderzoek is (zie
  Hoofdstuk 1).}. Alle
proefpersonen luisterden naar 12 muziekstukken, waarvan 6 afkomstig uit
de Franse en 6 uit de Tunesische muzikale culturen (de muziekstukken
waren verschillend qua maatsoorten, tempi, en mate van bekendheid). De
resultaten zijn grafisch samengevat in Figuur 6.1.
\end{quote}

\begin{figure}
\hypertarget{fig:drakebenelheni2003fig2}{%
\centering
\includegraphics{figures/DrakeBenElHeni2003Fig2.png}
\caption{Figuur 6.1: Gemiddelde tijdsafstand tussen tikken (IOI, in ms) voor twee groepen
luisteraars en twee muzieksoorten (ontleend aan Drake \& Ben El Heni,
2003, Fig.2).}\label{fig:drakebenelheni2003fig2}
}
\end{figure}

\begin{quote}
Uit deze resultaten blijkt dat er géén verschil optreedt tussen de
groepen (Franse vs Tunesische luisteraars; de twee groepen hebben
gemiddeld dezelfde IOI), en dat er ook géén verschil optreedt tussen de
muzieksoorten (Franse vs Tunesische muziekstukken; de twee muzieksoorten
resulteren in gemiddeld dezelfde IOI). Hebben de twee onafhankelijke
variabelen dan geen enkel effect? Toch wel! De Franse luisteraars bleken
langere IOI's tussen tikken te produceren als ze naar Franse muziek
luisterden, terwijl de Tunesische luisteraars daarentegen langere IOI's
produceerden als ze naar Tunesische muziek luisterden. Alle luisteraars
blijken dus langere IOI's te produceren als ze luisteren naar een
muzieksoort die voor hen bekend is, en kortere IOI's als ze luisteren
naar een onbekende muzieksoort. concluderen dat de luisteraars beter in
staat zijn om muzikale structuur te herkennen en te begrijpen in muziek
van hun eigen muzikale cultuur dan in die van een andere cultuur. Dit
patroon van resultaten is een klassiek kruislings interactie-effect,
waarbij het effect van de ene onafhankelijke variabele precies
tegengesteld is in de verschillende condities van de andere
onafhankelijke variabele.
\end{quote}

\begin{center}\rule{0.5\linewidth}{0.5pt}\end{center}

Als er een interactie-effect blijkt op te treden, dan is het zinloos om
een eventueel hoofdeffect te interpreteren. Dat werd al geïllustreerd in
Voorbeeld 6.5 hierboven: we kunnen niet concluderen dat er géén
verschil is tussen de muzieksoorten. Maar de grootte (en richting) van
het verschil is afhankelijk van de andere onafhankelijke variabele(n),
nl. van de groep/nationaliteit van de luisteraars. Veel onderzoek is er
juist op gericht om interactie-effecten aan te tonen; niet de
hoofdeffecten maar hun interactie vormt het onderwerp van onderzoek, net
als in het bovenstaande voorbeeld 6.5.

Een factorieel onderzoeksontwerp is lastig om schematisch weer te geven,
omdat er meerdere onafhankelijke variabelen (met elk weer meerdere
niveaus) in voorkomen. We zouden deze schematisch kunnen representeren
door de manipulatie, die we voorheen aangeduid hebben met een \texttt{X}, te
indiceren. De eerste index (subscript) geeft dan het niveau aan voor de
eerste onafhankelijke variabele of factor, en de tweede index geeft het
niveau aan van de tweede factor. Het ontwerp van voorbeeld 6.5 wordt dan als volgt schematisch weergegeven:

\begin{verbatim}
  R    X_{1,1}   O1
  R    X_{1,2}   O2
  R    X_{2,1}   O3
  R    X_{2,2}   O4
\end{verbatim}

Het is vaak verleidelijk om meerdere factoren te combineren in één groot
factorieel onderzoeksontwerp, zodat we kunnen onderzoeken hoe al die
factoren op elkaar inwerken (interageren). Toch is het verstandig om dat
\emph{niet} te doen, en om het aantal factoren beperkt te houden. Ten eerste,
zoals we later zullen zien, moet het aantal observaties ongeveer gelijke
tred houden met het aantal verschillende combinaties van factoren. Als
je meer combinaties van factoren toevoegt, dan zijn er daardoor ook veel
meer proefpersonen nodig (of andere eenheden). Ten tweede is het
moeilijker te garanderen dat alle combinaties van factoren perfect
vergelijkbaar zijn \citep[p.266]{SCC02}: zijn Tunesische deelnemers die in
Tunesië luisteren naar Tunesische muziek wel goed vergelijkbaar met
Franse deelnemers die in Frankrijk luisteren naar Franse muziek? De
vergelijkbaarheid van combinaties wordt lastiger, naarmate er meer
combinaties van factoren in het onderzoek voorkomen. Ten derde zijn
interacties notoir moeilijk te interpreteren, en dat wordt eveneens
lastiger naarmate de interacties complexer zijn, en meer factoren
omvatten. Om al deze redenen is het beter om effecten van meerdere
factoren te bestuderen in verschillende afzonderlijke onderzoeken
\citep{Quene10}.

We zullen later terugkomen op de analyse en interpretatie van gegevens
uit factoriële onderzoeksontwerpen
(Hoofdstuk \ref{ch:variantieanalyse}). Voorlopig concentreren we ons op
ontwerpen met slechts één onafhankelijke variabele.

\hypertarget{sec:afhankelijkegroepen}{%
\section{Afhankelijke- en onafhankelijke-groepen-ontwerp}\label{sec:afhankelijkegroepen}}

In het begin van dit hoofdstuk hebben we gesproken over de manipulatie
van een onafhankelijke variabele tussen danwel binnen proefpersonen ().
In de meeste van de voorafgaande onderzoeksontwerpen werd voor elke
waarde van de onafhankelijke variabele(n) een afzonderlijke groep
geformeerd; we noemen dat een onafhankelijke-groepen-ontwerp. De
onafhankelijke variabele varieert tussen proefpersonen.

Sommige onafhankelijke variabelen kunnen echter ook gevarieerd worden
binnen proefpersonen. We meten dan herhaaldelijk bij (binnen) dezelfde
proefpersonen uit dezelfde groep, onder verschillende condities van de
onafhankelijke variabele. In het onderstaande voorbeeld wordt de
onafhankelijke variabele `taal' (moedertaal of vreemde taal) gevarieerd
binnen proefpersonen. We noemen dat een afhankelijke-groepen-ontwerp.

\begin{center}\rule{0.5\linewidth}{0.5pt}\end{center}

\begin{quote}
\emph{Voorbeeld 6.6}:
\citep{JGSH13} onderzochten de vloeiendheid van de spraak van proefpersonen in hun
moedertaal (Turks) en in een vreemde taal (Nederlands). De proefpersonen
voerden eerst een aantal spreektaken uit in hun moedertaal, en enkele
weken later in het Nederlands. Eén van de afhankelijke variabelen was
het aantal gevulde pauzes (bijv. \emph{eh, ehm}) per seconde spraak: hoe meer
pauzes, hoe minder vloeiend. De sprekers bleken meer pauzes te
produceren (d.w.z. minder vloeiend te spreken) in de vreemde taal dan in
hun moedertaal, zoals ook te verwachten is. Eén van de doelen van het
onderzoek was overigens om te onderzoeken in hoeverre individuele
verschillen in vloeiendheid in de vreemde taal te herleiden zijn naar
individuele verschillen in vloeiendheid in de moedertaal. De samenhang
tussen de twee metingen bleek hoog (correlatie \(r=0.73\), meer hierover
in Hoofdstuk \ref{ch:samenhang}. Sprekers die veel pauzeren in de vreemde
taal, doen dat vaak ook in hun moedertaal. De onderzoekers bepleiten dat
met deze correlatie rekening gehouden moet worden bij het aanleren en
toetsen van spreekvaardigheid in een vreemde taal.
\end{quote}

Het hier beschreven onderzoeksontwerp ziet er in schema als volgt uit:

\begin{verbatim}
   X1   O1   X2   O2
\end{verbatim}

Ondanks de vele mogelijke bedreigingen van de interne validiteit (o.a.
geschiedenis, rijping, sturende werking van de voormeting) is zo'n
ontwerp in veel gevallen nuttig. In het bovenstaande voorbeeld is het
essentieel dat \emph{dezelfde} proefpersonen spreektaken uitvoeren in beide
talen (condities) --- de onderzoeksvragen zijn niet met een andere
methode te beantwoorden.

\hypertarget{onderzoek-ontwerpen}{%
\section{Onderzoek ontwerpen}\label{onderzoek-ontwerpen}}

Een onderzoeker die een onderzoek wil uitvoeren, moet bepalen op welke
wijze hij zijn gegevens gaat verzamelen: hij moet een keuze maken voor
een bepaald onderzoeksontwerp. Soms kan een standaard-ontwerp gekozen
worden, zoals een van de hierboven behandelde ontwerpen. In andere
gevallen zal de onderzoeker zelf het ontwerp moeten opstellen. Het
ontwerp moet uiteraard goed aansluiten bij de onderzoeksvraag
\citep{Levin99}, en het moet zoveel mogelijk storende variabelen uitsluiten
die de validiteit zouden kunnen bedreigen. Het ontwerpen van onderzoek
is een vak dat onderzoekers al doende leren. In het onderstaande
voorbeeld proberen we weer te geven welke redenering en argumenten een
rol spelen bij de ontwikkeling van een onderzoeksontwerp.

Stel je voor dat we willen onderzoeken of de vorm waarin toetsvragen
gesteld worden, als open vs.~gesloten vragen, van invloed is op de
scores op die toets. In een eenvoudig ontwerp nemen we eerst een toets
af met open vragen, en daarna een vergelijkbare toets met gesloten
vragen, bij dezelfde respondenten. Als de samenhang tussen de scores
hoog genoeg is, dan wordt geconcludeerd dat beide toetsen hetzelfde
meten, en dat de prestatie niet wezenlijk beïnvloed wordt door de vorm
waarin de vragen worden gesteld. Schematisch is dit ontwerp als volgt
weer te geven:

\begin{verbatim}
   Open   O1   Gesloten   O2
\end{verbatim}

Dit onderzoeksontwerp heeft echter diverse zwakke punten. Ten eerste is
het onverstandig om eerst alle toetsen met open vragen op het eerste
tijdstip af te nemen, en vervolgens alle toetsen met gesloten vragen op
het tweede tijdstip. De prestaties op de tweede toets worden immers
altijd beïnvloed door volgorde-effecten (transfer): de respondenten
hebben iets onthouden en dus geleerd van de eerste meting. Deze transfer
werkt nu altijd dezelfde kant op, met daardoor (vermoedelijk) relatief
hogere prestaties bij de toets met gesloten vragen (op het tweede
tijdstip). Het is dus beter om de toetsen met open en gesloten vragen at
random te verdelen over het eerste en tweede tijdstip van afname.

Ten tweede kunnen alle respondenten beïnvloed zijn door eventuele
gebeurtenissen tussen de twee tijdstippen (geschiedenis), bijvoorbeeld
door een relevante instructie over het onderwerp van de toets. Omdat er
geen controle-groep is, kan met zo'n effect geen rekening worden
gehouden.

Een derde probleem is gelegen in de wijze waarop van bevindingen naar
conclusie wordt geredeneerd. Zoals gezegd, houdt die redenering in dat,
als de samenhang tussen de scores hoog genoeg is, dan beide toetsen
hetzelfde meten. Als je daarover nadenkt, dan ben je het misschien met
ons eens dat dat een vreemde redenering is. De onderzoeksvraag is
eigenlijk, of de samenhang in prestaties op verschillende toetsen met
verschillende vraagvormen even hoog is als de samenhang in prestaties op
verschillende toetsen met dezelfde vraagvormen, waarvan we immers
aannemen dat die hetzelfde meten. Daarmee hebben we in feite een
controle-groep gedefinieerd, nl. respondenten die op beide tijdstippen
toetsen maken met dezelfde vraagvorm. Voor alle zekerheid voegen we niet
één maar twee controle-groepen toe, met open resp. gesloten toetsvragen
op beide tijdstippen.

We hebben het ontwerp zo in tenminste twee opzichten verbeterd: (1) de
toetsen zijn gerandomiseerd over de opname-tijdstippen, en (2) er zijn
relevante controle-groepen toegevoegd. Schematisch ziet het
onderzoeksontwerp er nu als volgt uit:

\begin{verbatim}
       Exp. groep 1       Open     O1   Gesloten   O2
       Exp. groep 2     Gesloten   O3     Open     O4
      Controlegroep 1     Open     O5     Open     O5
      Controlegroep 2   Gesloten   O7   Gesloten   O6
\end{verbatim}

Voor alle vier de groepen kan nu de samenhang tussen de prestaties op
het eerste en tweede tijdstip bepaald worden. Vervolgens kunnen we deze
samenhang-resultaten uit de vier groepen vergelijken, en daarmee de
onderzoeksvraag beantwoorden. Dit voorbeeld laat goed zien dat de
conclusies die je uit de onderzoeksresultaten kunt trekken, direct
afhankelijk zijn van het gekozen ontwerp \citep{Levin99}. In het eerste
ontwerp leidt een lage gevonden samenhang tot de conclusie dat de twee
onderzochte toetsvormen \emph{niet} een beroep doen op dezelfde intellectuele
vaardigheden bij de respondenten. In het tweede ontwerp hoeft dezelfde
lage samenhang in de derde groep (experimentele groep 1) echter niet tot
dezelfde uitkomst te leiden! De conclusie hangt immers mede af van de
mate van samenhang gevonden in de andere groepen.

\hypertarget{tenslotte}{%
\section{Tenslotte}\label{tenslotte}}

Ondanks alle beschikbare boeken, handleidingen, websites, en ander
instructiemateriaal komen wij nog te vaak onderzoek tegen waar
methodologisch iets mis is in de onderzoeksvragen, operationalisatie,
onderzoeksopzet, steekproeftrekking, en/of dataverwerking. Die problemen
veroorzaken niet alleen een verspilling van tijd, geld en energie, maar
ze resulteren ook in kennis die minder betrouwbaar, valide en robuust is
dan mogelijk. De onderstaande `checklist' voor goed onderzoek (deels
ontleend aan
\url{https://www.linkedin.com/groups/4292855/4292855-6093149378770464768})
kan veel ellende in latere stadia van een onderzoek voorkomen.

\begin{enumerate}
\def\labelenumi{\arabic{enumi}.}
\item
  Denk goed na over je onderzoeksvragen, en formuleer ze helemaal uit.
  Als de vragen niet helder geformuleerd zijn, of als er veel
  deelvragen zijn, denk dan verder na.
\item
  Prioriteer de onderzoeksvragen. Dit helpt bij het maken van keuzes
  in onderzoeksontwerp, steekproeftrekking, operationalisatie, e.d.
\item
  Denk goed na over het ontwerp van het onderzoek. Volgens de
  overlevering levert ieder uur nadenken over je onderzoeksontwerp een
  toekomstige besparing van ongeveer 10 uren tijdens de data-analyse
  en interpretatie. Anders gezegd: een uur minder nadenken over je
  ontwerp kost je later 10 uur extra werk.
\item
  Bedenk ook alternatieve onderzoeksontwerpen, en denk na over de
  voordelen en nadelen van de diverse mogelijke ontwerpen.
\item
  Stel je de toekomst voor: je hebt het onderzoek uitgevoerd, de
  gegevens zijn geanalyseerd, en je hebt het verslag of de scriptie of
  het artikel geschreven. Welke boodschap wil je overbrengen op de
  lezers van dat verslag? Hoe draagt het onderzoeksontwerp bij aan die
  boodschap? Wat zou je kunnen veranderen in je ontwerp om die
  boodschap nog duidelijker te maken? Bedenk waar je naar toe wilt,
  niet alleen waar je nu staat.
\item
  Schrijf een onderzoeksplan, waarin je de verschillende
  methodologische aspecten beschrijft. Beargumenteer en expliciteer je
  onderzoeksvragen, onderzoeksontwerp, steekproef, meetmethode,
  data-verzameling, meetinstrumenten (bv. vragenlijst, software),
  andere benodigdheden (bv. laboratorium, vervoer), en statistische
  verwerking. Onderdelen van dit onderzoeksplan zijn later
  herbruikbaar in het onderzoeksverslag. Maak daarbij ook een
  tijdsplanning: wanneer zullen welke mijlpalen zijn bereikt?
\item
  Schrijf uit hoe je de verzamelde gegevens statistisch zult
  analyseren, nog voordat je begint met de eigenlijke
  data-verzameling. Wees daarbij weer zo expliciet mogelijk (in een
  script, stappenplan, o.i.d.). Maak een mini-dataverzameling van een
  redelijk aantal fictieve observaties of werkelijke observaties uit
  de pilot-fase van het onderzoek, en analyseer deze gegevens alsof
  het de definitieve data-verzameling betreft. Maak eventueel
  aanpassingen in je onderzoeksplan.
\item
  Als je eenmaal doende bent gegevens te verzamelen, maak dan \emph{geen}
  wijzigingen meer in het onderzoeksplan. Houd je aan dat plan en aan
  de bijbehorende tijdsplanning. Analyseer de gegevens op de wijze
  zoals vastgelegd in het (aangepaste) onderzoeksplan. Bespreek
  eventuele problemen die tijdens het onderzoek optraden wel in het
  onderzoeksverslag. Als er grote problemen optreden, breek dan het
  onderzoek af, en overweeg een verbeterde versie van je onderzoek.
\end{enumerate}

\hypertarget{ch:steekproeftrekking}{%
\chapter{Steekproeven}\label{ch:steekproeftrekking}}

Voor de generalisatie van de uitkomsten van een onderzoek naar de
doelgroep of de steekproef, is de kwaliteit van de steekproef bepalend.
Is de steekproef een adequate afspiegeling van de populatie? Om een
extreem voorbeeld te geven: als een steekproef bestaat uit meisjes in de
groep 8 van het basisonderwijs, dan kunnen de resultaten niet goed
gegeneraliseerd worden naar de populatie van alle basisschoolleerlingen,
want deze steekproef vormt geen goede afspiegeling van de populatie
basisschoolleerlingen (die immers bestaat uit jongens en meisjes van
alle groepen).

Afhankelijk van de methode die de onderzoekers gebruiken om de
proefpersonen te selecteren, kunnen er vele soorten steekproeven
onderscheiden worden. In dit hoofdstuk maken we een grove indeling in:
(1) gelegenheidssteekproeven, (2) systematisch getrokken steekproeven,
en (3) aselect of willekeurig (`at random') getrokken steekproeven. Voor
een verdere verdieping in de wijze waarop steekproeven getrokken kunnen
worden en de problemen die daarbij een rol spelen verwijzen we naar
standaardwerken hierover \citep{Coch77, Thom12}.

\hypertarget{sec:gelegenheidssteekproef}{%
\section{Gelegenheidssteekproeven}\label{sec:gelegenheidssteekproef}}

In veel sociaalwetenschappelijk onderzoek wordt gewerkt met steekproeven
die zich nu eenmaal aandienen, zogenaamde \emph{gelegenheidssteekproeven}. De
onderzoeker voert het experiment uit met personen die hem min of meer
toevallig ter beschikking staan. Voor sommige onderzoeken wordt gebruik
gemaakt van al dan niet betaalde vrijwilligers. In andere onderzoeken
worden studenten ingezet, die in het kader van hun studie verplicht zijn
een aantal uren als proefpersoon aan onderzoek mee te werken, of soms
moeten de studenten van een collega van de onderzoeker deelnemen aan het
onderzoek. Een dergelijke steekproef is niet zonder gevaren. De
onderzoeker heeft de mate van generaliseerbaarheid naar de populatie op
geen enkele manier meer in de hand. Natuurlijk heeft de onderzoeker wel
een populatie op het oog, en zal hij proefpersonen uit het onderzoek
weren die geen deel uit maken van de beoogde populatie (zoals
niet-moedertaal-sprekers), maar de onderzoeker kan geen uitspraken doen
over de representativiteit van de steekproef.

Met name in de psychologie heeft deze wijze van
gelegenheidssteekproeftrekking (`convenience sampling') aanleiding
gegeven tot verhitte discussies. Uit een telling bleek bijvoorbeeld dat
67\% van de steekproeven uit gepubliceerde Amerikaanse psychologische
studies uitsluitend bestond uit bachelor-studenten uit cursussen
Psychologie aan Amerikaanse universiteiten \citep{Henr10}. Dergelijke
steekproeven zijn natuurlijk verre van representatief. Gevolg daarvan is
dat de op deze gegevens gebaseerde theorieën slechts een beperkte
geldigheid hebben: de theorieën zouden vooral gelden voor het type
personen (westers, jong, hoog opgeleid, blank) dat ook in de
steekproeven sterk vertegenwoordigd is \citep{Henr10}. Ook in
taalwetenschappelijk onderzoek is de steekproef van proefpersonen
meestal een gelegenheidssteekproef. Kinderen die deelnemen als
proefpersoon hebben vaak hoogopgeleide ouders (niet zelden zelf
taalkundig geschoold, dus vermoedelijk bovengemiddeld verbaal begaafd),
en volwassen proefpersonen zijn vaak studenten uit de omgeving van de
onderzoekers, en dus ook bovengemiddeld hoog opgeleid en verbaal
begaafd.

Ondanks de steekhoudende bezwaren die tegen dit type steekproef naar
voren gebracht worden, dwingen de praktische omstandigheden vaak tot het
gebruik van een zich aandienende gelegenheidssteekproef. Wij bevelen dan
aan om na te gaan in hoeverre deze gelegenheidssteekproef zich
onderscheidt van de populatie waarover de onderzoeker wil generaliseren.
Tot slot van deze bespreking van zich aandienende steekproeven een
voorbeeld over de gevaren van dit type steekproef.

\begin{center}\rule{0.5\linewidth}{0.5pt}\end{center}

\begin{quote}
\emph{Voorbeeld 7.1}: Enige tijd geleden was er op televisie een wedstrijd te zien
over wie
van een negental kandidaten het beste kon zingen. De kijkers mochten hun
voorkeur telefonisch kenbaar maken. Voor alle negen kandidaten was een
aparte telefoonlijn geopend. Voor elke beller kreeg een kandidaat één
punt. Degene die de meeste punten binnen een bepaalde tijdlimiet
verzameld had was de winnaar. De reactie van het publiek was
overweldigend: in grote delen van Nederland was het telefoonnet volledig
overbezet. Al snel bleek één van de kandidaten een flinke voorsprong te
hebben. In de loop van de avond werd deze voorsprong echter steeds
kleiner. Uiteindelijk scheelde het nog maar enkele bellers met nummer
twee. Opvallend was overigens dat naarmate de avond vorderde de
verschillen tussen de deelnemers (relatief) steeds kleiner werden.
\end{quote}

\begin{center}\rule{0.5\linewidth}{0.5pt}\end{center}

We kunnen deze stemprocedure beschouwen als een trekking van een
steekproef van bellers c.q. stemmers. Deze steekproef is echter verre
van representatief. Als veel kiezers willen stemmen op één kandidaat,
dan zal de telefoonlijn voor die kandidaat overbezet raken. Dus: de
zangers die veel bellers trekken, zullen relatief minder stemmen krijgen
dan zangers die weinig bellers trekken, omdat de telefoonlijnen van de
laatsten niet overbezet zullen zijn. Juist bij de populairste kandidaten
is de kans het grootst dat een kiezer zijn stem \emph{niet} kan laten gelden.
In werkelijkheid zal er dus een veel groter verschil zijn in aantal
stemmen per kandidaat, dan de organisator gemeten heeft. De organisator
heeft deze systematische vertekening (bias) van de resultaten helaas
zelf veroorzaakt, door voor elk van de negen kandidaten een eigen
telefoonlijn te openen. De gegevens hadden veel representatiever kunnen
zijn, als de organisator negen telefoonlijnen had geopend, met één
gemeenschappelijk toegangsnummer. De steekproef van bellers die hun stem
kunnen uitbrengen is dan representatief voor de populatie van alle
bellers, en dat was nu niet het geval.

\hypertarget{sec:systematischesteekproef}{%
\section{Systematische steekproeven}\label{sec:systematischesteekproef}}

Wanneer de elementen in de \emph{steekproefruimte} (d.i. de verzameling van
mogelijke elementen in een steekproef) op de een of andere manier
systematisch geordend zijn, dan kan met behulp van een \emph{systematische
trekkingsprocedure} van steekproefelementen een redelijk representatieve
steekproef verkregen worden. Een ordening kan zijn bijvoorbeeld een
namenlijst.

\begin{center}\rule{0.5\linewidth}{0.5pt}\end{center}

\begin{quote}
\emph{Voorbeeld 7.2}:Laten we even aannemen dat we een onderzoek willen doen naar de
taalvaardigheid van derdeklassers in het voortgezet onderwijs. De gehele
populatie van derdeklassers is echter veel te groot om van alle
derdeklassers de taalvaardigheid te meten (lezen, schrijven, spreken, en
luisteren). In de derde klas zitten namelijk ongeveer 200.000
leerlingen. Er moet dus een steekproef genomen worden. Op het Ministerie
van Onderwijs, Cultuur en Wetenschappen is een registratiesysteem
beschikbaar waarin een lijst met de namen van alle scholen met derde
klassen is opgenomen. Een voor de hand liggende werkwijze is nu deze
lijst te nemen en elke 100ste school van die lijst in de steekproef op
te nemen. Deze werkwijze resulteert vermoedelijk in een tamelijk
representatieve steekproef.
\end{quote}

\begin{center}\rule{0.5\linewidth}{0.5pt}\end{center}

Twee factoren kunnen echter roet in het eten gooien bij zo'n
systematische steekproef: ten eerste de \emph{responsiegraad}. Als een
aanzienlijk deel van de aangeschreven scholen geen medewerking verleent,
dan hebben we in feite te maken met zelf-selectie (zie
§\ref{sec:internevaliditeit} punt 5)
en dus met een zichzelf aandienende
gelegenheidssteekproef (zie
§\ref{sec:gelegenheidssteekproef}). Dat is een ongewenste
situatie, want de scholen die wel meewerken hebben vermoedelijk een
grotere `plichtsgetrouwheid' dan de weigerende scholen of dan de
gemiddelde school. Bovendien kunnen de leerlingen op de responderende en
niet-responderende scholen van elkaar verschillen
(zie §\ref{sec:internevaliditeit} punt 5).
De uiteindelijke steekproef is dan
misschien niet meer representatief voor de populatie van alle
derdeklassers. Het gevolg daarvan is weer dat de gemeten resultaten
slecht generaliseerbaar zijn naar andere derdeklassers van andere
scholen.

De tweede factor die de representativiteit van een systematische
steekproef kan beïnvloeden is de \emph{storende trendwerking}. Er is sprake
van een storende trendwerking wanneer populatie-elementen met een
bepaald relevant kenmerk meer kans hebben in de steekproef terecht te
komen dan populatie-elementen die dit kenmerk niet hebben. In ons
voorbeeld van de meting van de taalvaardigheid van derdeklassers hebben
we met de storende trendwerking te maken. Niet alle leerlingen hebben
namelijk een gelijke kans om in de steekproef te komen. Immers, elke
individuele \emph{school} (niet: leerling) heeft dezelfde kans als elke
andere school om in de steekproef terecht te komen. Het gevolg is dat er
relatief meer derdeklassers in de steekproef zullen komen van kleine
scholen met relatief weinig leerlingen, en omgekeerd relatief minder
derdeklassers van grote scholen met relatief veel leerlingen.
Derdeklassers van grote scholen zijn ondervertegenwoordigd. Is dat erg?
Misschien wel, want de taalvaardigheid (afhankelijke variabele) wordt
deels beïnvoed door de vorm van onderwijs, en die onderwijsvorm wordt
weer beïnvloed door de grootte van de school. De hierboven beschreven
steekproef is dus niet representatief voor de populatie van
derdeklassers. Wederom is het gevolg dat de gemeten resultaten slecht
generaliseerbaar zijn naar andere derdeklassers van andere scholen.

\hypertarget{sec:aselectesteekproef}{%
\section{Aselecte steekproeven}\label{sec:aselectesteekproef}}

De hierboven beschreven storende trendwerking kunnen we voorkomen door
random of \emph{aselecte steekproeftrekking}. Aselecte steekproeftrekking kan
op diverse manieren gebeuren, waarvan we er hier drie bespreken.

De eerste vorm is \emph{simple random sampling}: hierbij krijgen alle
elementen van de populatie een gelijke kans om getrokken te worden. Dit
kan bijvoorbeeld gerealiseerd worden door alle elementen van een
\emph{random} nummer te voorzien en dan, afhankelijk van de gewenste
steekproefgrootte, steeds het \(n\)-de element te selecteren. Voor de
selectie van getallen staan de onderzoeker tabellen met toevalsgetallen
ter beschikking (zie Appendix \ref{app:randomgetallen}).
Ook rekenmachines, computers,
spreadsheet-programma's e.d. kunnen random getallen genereren. (Het
verdient aanbeveling om zulke random getallen te gebruiken, want een
door mensen geconstrueerde ``random'' volgorde is niet werkelijk
``random''.) Een voorwaarde voor de toepassing van deze methode is echter
wel dat de elementen van de populatie (steekproefruimte) vooraf
geregistreerd zijn (of worden), zodat ze op enigerlei wijze van een
nummer voorzien kunnen worden.

\begin{center}\rule{0.5\linewidth}{0.5pt}\end{center}

\begin{quote}
\emph{Voorbeeld 7.3}: We willen een steekproef trekken
van \(n=400\) basisscholen. Dit is ongeveer 4\% van de populatie van
basisscholen. We vragen daarom bij het Ministerie van Onderwijs, Cultuur
en Wetenschappen een lijst met alle 9000 basisscholen op; deze lijst
vormt de steekproefruimte. Vervolgens voorzien we alle basisscholen van
een volgnummer \((1, 2, 3, \ldots, 9000)\). Tenslotte selecteren we alle
basisscholen waarvan de laatste twee cijfers \emph{toevallig} 36 of 43 of 59
of 70 zijn (zie Appendix \ref{app:randomgetallen}, eerste kolom, laatste twee cijfers).
Met deze procedure selecteren we volgens het toeval 4 van de 100
mogelijke laatste-twee-cijfer-combinaties, ofwel 4\% van de scholen.
\end{quote}

\begin{center}\rule{0.5\linewidth}{0.5pt}\end{center}

De tweede vorm van aselecte steekproeftrekking is
\emph{stratified random sampling}.
Daarvan is sprake als we van elk populatie-element de waarde
van een kenmerk weten (bv. religieuze denominatie), en als we zorgen dat
in de steekproef de elementen evenredig verdeeld zijn volgens dit
kenmerk. We verdelen de steekproef daarvoor in zogenaamde `strata' of
lagen (Lat. \emph{stratum}, `bedekking, laag', verwant aan Ned. \emph{straat},
`verharde weg'). Terug naar de basisschool om het een en ander te
verhelderen. Om welke reden dan ook zijn we er nu in geïnteresseerd de
steekproef (nog steeds 4\% van de populatie van basisscholen) zo te maken
dat openbare, katholieke en protestante scholen in gelijke mate
vertegenwoordigd zijn. We stellen daarom drie lijsten op: voor alle drie
de schooltype een aparte lijst. Binnen iedere lijst gaan we net zo te
werk als bij simple random sampling. Uiteindelijk worden de drie
deel-steekproeven van de drie strata gecombineerd.

Met \emph{quota sampling} gaan we nog een stapje verder dan bij `stratified
random sampling': we verdisconteren nu ook het feit dat we weten wat de
verdeling is van een bepaald kenmerk (bv denominatie) in de populatie.
Uit de lijst met basisscholen zou hebben kunnen blijken dat 35\% van de
scholen openbaar is, 31\% katholiek, 31\% protestant en dat 3\% een andere
signatuur heeft. We trekken uit de steekproefruimte nu meerdere aselecte
`stratified' steekproeven, en wel zo dat de verhouding van scholen in de
strata een juiste afspiegeling vormt van de verhoudingen van dit kenmerk
in de steekproefruimte \((35:31:31:3)\).

\hypertarget{sec:steekproefgrootte}{%
\section{Steekproefgrootte}\label{sec:steekproefgrootte}}

Als je verschillende onderzoeksartikelen leest, dan is één van de eerste
zaken die opvalt de enorme variatie in aantallen respondenten. In
sommige onderzoeken worden enkele duizenden proefpersonen betrokken en
in andere slechts enkele tientallen of soms nog minder. We zullen hier
twee aspecten bespreken die van invloed zijn op de vereiste grootte van
de steekproef: de homogeniteit van de populatie, en de aard van de
steekproeftrekking. In volgende hoofdstukken zullen we nog twee andere
aspecten bespreken die eveneens van invloed zijn op de gewenste
steekproefgrootte, nl. de gewenste precisie (effectgrootte,
§\ref{sec:ttoets-effectgrootte}) en de gewenste kans om een effect
aan te tonen als dat in de populatie ook daadwerkelijk aanwezig is
(power,
§\ref{sec:effectgrootte-power}).

\begin{center}\rule{0.5\linewidth}{0.5pt}\end{center}

\begin{quote}
\emph{Voorbeeld 7.4}: Wanneer auto's getest worden (in een
tijdschrift of op televisie), dan wordt van een type auto slechts één
exemplaar getest. De resultaten van dit testexemplaar worden zonder
voorbehoud gegeneraliseerd naar alle auto's van hetzelfde type en merk.
Dit is mogelijk omdat de populatie auto's waarnaar gegeneraliseerd wordt
bijzonder homogeen is: de fabrikant streeft er immers naar om de
verschillende exemplaren zo gelijk mogelijk op de markt te brengen.
\end{quote}

\begin{center}\rule{0.5\linewidth}{0.5pt}\end{center}

De vereiste steekproefgrootte hangt ten eerste af van de homogeniteit
van de populatie. Als een populatie \emph{homogeen} is, zoals de auto's in
voorbeeld 7.4 hierboven, dan kunnen we met een kleine
steekproef volstaan. Anders is het wanneer we bijvoorbeeld de
conversatiepatronen van kleuters willen analyseren. In de
conversatiepatronen van kleuters treffen we grote verschillen aan; er is
een zeer grote variatie in conversatiepatronen. (Sommige kinderen praten
voluit, en andere zwijgen vooral. Bovendien zijn er grote individuele
verschillen in taalontwikkeling tussen kinderen.) Om een goed beeld te
krijgen van de taalontwikkeling van kleuters, hebben we daarom een veel
grotere steekproef nodig. De grootte van de benodigde steekproef neemt
dus toe naarmate de populatie waarna gegeneraliseerd moet worden minder
homogeen (heterogener) is.

Ten tweede hangt de vereiste steekproefgrootte ook af van de aard van de
steekproef. Als er in een populatie duidelijke strata aanwezig zijn,
maar we passen -- om welke reden dan ook -- geen `stratified' of `quota
sampling' toe, dan hebben we een grotere steekproef nodig dan wanneer we
dit wel zouden doen. Immers, bij deze laatste twee methoden zorgt de
onderzoeker zelf voor een gelijke dan wel evenredige vertegenwoordiging
van strata in de steekproef, maar bij `simple random sampling' wordt dat
aan het toeval overgelaten. We doen dan dus een beroep op de ``wet van de
grote getallen'' om te zorgen dat er voldoende elementen uit de
verschillende strata in de steekproef terecht komen, om generalisatie
van de resultaten naar die verschillende strata te rechtvaardigen.
Uiteraard werkt die wet alleen bij een voldoende grote steekproef! Bij
een kleine steekproef weten we allerminst zeker dat de verschillende
strata in voldoende mate in de steekproef vertegenwoordigd zijn.

Als we, om naar het basisschool-voorbeeld terug te keren, drie
basisscholen zouden selecteren volgens `simple random sampling', dan
bestaat natuurlijk een kans dat dit precies één openbare, één katholieke
en één protestante school oplevert in deze steekproef. Maar ook andere
uitkomsten zijn zeer reëel, en zelfs meer waarschijnlijk. Bij
`stratified' en `quota sampling' hebben we gegarandeerd van elke
denominatie één element (school) in onze steekproef. Onze basis voor
generalisatie is beter, en de externe validiteit is dus sterker.

Na al deze behartenswaardige aanbevelingen wordt het tijd om te
bespreken hoe we onderzoeksgegevens goed kunnen beschrijven en
analyseren om onze onderzoeksvragen te beantwoorden. Dat gebeurt in het
volgende deel van dit boek.

\hypertarget{appendix-appendices}{%
\appendix}


\hypertarget{app:randomgetallen}{%
\chapter{Willekeurige getallen}\label{app:randomgetallen}}

\begin{table}[t]

\caption{\label{tab:A1}Onderstaande tabel bevat 200 willekeurige getallen tussen 0 en 9999.}
\centering
\begin{tabular}{rrrrrrrrrr}
\toprule
 &  &  &  &  &  &  &  &  & \\
\midrule
2836 & 264 & 6789 & 1483 & 3459 & 9200 & 4996 & 3761 & 699 & 5622\\
1943 & 6034 & 8838 & 1349 & 8750 & 3181 & 8799 & 4525 & 6536 & 5111\\
7259 & 8030 & 5709 & 8334 & 3526 & 2768 & 6296 & 8335 & 6350 & 6192\\
570 & 8266 & 9050 & 7771 & 3 & 7983 & 1871 & 3927 & 5549 & 1487\\
1241 & 2273 & 505 & 8816 & 4786 & 533 & 9347 & 888 & 3728 & 4135\\
\addlinespace
6688 & 9456 & 2880 & 4616 & 7698 & 2955 & 9597 & 9188 & 8932 & 5605\\
1325 & 1294 & 8001 & 1814 & 5020 & 9470 & 8702 & 4083 & 6452 & 2863\\
6196 & 5085 & 9961 & 5306 & 1660 & 1809 & 8405 & 2019 & 2710 & 1368\\
1577 & 5112 & 874 & 6909 & 4126 & 8473 & 2065 & 1511 & 4778 & 4440\\
5778 & 1207 & 3337 & 1888 & 1420 & 6917 & 4160 & 2682 & 5263 & 5926\\
\addlinespace
6635 & 1887 & 8836 & 2940 & 2404 & 7017 & 3119 & 3699 & 2529 & 8663\\
6813 & 5759 & 3314 & 6929 & 5238 & 6008 & 5900 & 8485 & 5938 & 5642\\
5208 & 2391 & 8324 & 6888 & 9449 & 2577 & 7859 & 176 & 1650 & 8389\\
5446 & 4412 & 9857 & 9535 & 2794 & 7883 & 4119 & 6439 & 8082 & 7918\\
2984 & 2126 & 9506 & 2188 & 9762 & 9775 & 4213 & 7624 & 4520 & 1086\\
\addlinespace
371 & 4559 & 12 & 718 & 8403 & 8150 & 6533 & 3741 & 6279 & 8546\\
4669 & 1053 & 3343 & 4889 & 9088 & 9188 & 8093 & 9496 & 8806 & 923\\
4070 & 3408 & 8102 & 3012 & 9706 & 771 & 8296 & 3094 & 148 & 7244\\
4867 & 6267 & 1225 & 6539 & 7958 & 7217 & 7833 & 728 & 1610 & 5284\\
4665 & 1912 & 5320 & 8563 & 1365 & 3834 & 1818 & 7791 & 7704 & 2460\\
\bottomrule
\end{tabular}
\end{table}

  \bibliography{book.bib,packages.bib,hhmhto.bib}

\end{document}
