% Options for packages loaded elsewhere
\PassOptionsToPackage{unicode}{hyperref}
\PassOptionsToPackage{hyphens}{url}
%
\documentclass[
]{book}
\usepackage{lmodern}
\usepackage{amssymb,amsmath}
\usepackage{ifxetex,ifluatex}
\ifnum 0\ifxetex 1\fi\ifluatex 1\fi=0 % if pdftex
  \usepackage[T1]{fontenc}
  \usepackage[utf8]{inputenc}
  \usepackage{textcomp} % provide euro and other symbols
\else % if luatex or xetex
  \usepackage{unicode-math}
  \defaultfontfeatures{Scale=MatchLowercase}
  \defaultfontfeatures[\rmfamily]{Ligatures=TeX,Scale=1}
\fi
% Use upquote if available, for straight quotes in verbatim environments
\IfFileExists{upquote.sty}{\usepackage{upquote}}{}
\IfFileExists{microtype.sty}{% use microtype if available
  \usepackage[]{microtype}
  \UseMicrotypeSet[protrusion]{basicmath} % disable protrusion for tt fonts
}{}
\makeatletter
\@ifundefined{KOMAClassName}{% if non-KOMA class
  \IfFileExists{parskip.sty}{%
    \usepackage{parskip}
  }{% else
    \setlength{\parindent}{0pt}
    \setlength{\parskip}{6pt plus 2pt minus 1pt}}
}{% if KOMA class
  \KOMAoptions{parskip=half}}
\makeatother
\usepackage{xcolor}
\IfFileExists{xurl.sty}{\usepackage{xurl}}{} % add URL line breaks if available
\IfFileExists{bookmark.sty}{\usepackage{bookmark}}{\usepackage{hyperref}}
\hypersetup{
  pdftitle={Methoden en Statistiek 1},
  hidelinks,
  pdfcreator={LaTeX via pandoc}}
\urlstyle{same} % disable monospaced font for URLs
\usepackage{longtable,booktabs}
% Correct order of tables after \paragraph or \subparagraph
\usepackage{etoolbox}
\makeatletter
\patchcmd\longtable{\par}{\if@noskipsec\mbox{}\fi\par}{}{}
\makeatother
% Allow footnotes in longtable head/foot
\IfFileExists{footnotehyper.sty}{\usepackage{footnotehyper}}{\usepackage{footnote}}
\makesavenoteenv{longtable}
\usepackage{graphicx}
\makeatletter
\def\maxwidth{\ifdim\Gin@nat@width>\linewidth\linewidth\else\Gin@nat@width\fi}
\def\maxheight{\ifdim\Gin@nat@height>\textheight\textheight\else\Gin@nat@height\fi}
\makeatother
% Scale images if necessary, so that they will not overflow the page
% margins by default, and it is still possible to overwrite the defaults
% using explicit options in \includegraphics[width, height, ...]{}
\setkeys{Gin}{width=\maxwidth,height=\maxheight,keepaspectratio}
% Set default figure placement to htbp
\makeatletter
\def\fps@figure{htbp}
\makeatother
\setlength{\emergencystretch}{3em} % prevent overfull lines
\providecommand{\tightlist}{%
  \setlength{\itemsep}{0pt}\setlength{\parskip}{0pt}}
\setcounter{secnumdepth}{5}
\usepackage{booktabs}
\usepackage{amsthm}
\makeatletter
\def\thm@space@setup{%
  \thm@preskip=8pt plus 2pt minus 4pt
  \thm@postskip=\thm@preskip
}
\makeatother
\usepackage[]{natbib}
\bibliographystyle{apalike}

\title{Methoden en Statistiek 1}
\author{true}
\date{1 juli 2020}

\begin{document}
\maketitle

{
\setcounter{tocdepth}{1}
\tableofcontents
}
\hypertarget{voorwoord}{%
\chapter*{Voorwoord}\label{voorwoord}}
\addcontentsline{toc}{chapter}{Voorwoord}

In dit boek hebben we gepoogd om de belangrijkste kwantitatieve methoden en statistische technieken die relevant zijn voor de Geesteswetenschappen uit te leggen. De tekst is waar mogelijk gevrijwaard van wiskundige afleidingen en formules, omdat we die voor studenten Geesteswetenschappen minder bruikbaar achten. Indien nodig geven we de belangrijkste formules voor statistische beschrijvingen en analyses in een aparte paragraaf.

Het tekstboek bevat ook aanwijzingen over hoe de besproken statistische analyses en visualisaties uitgevoerd kunnen worden in twee veelgebruikte programma's, nl. SPSS (versie 22 en later) en R (versie 3.0 en later). Ook deze aanwijzingen staan los van de hoofdtekst, in afzonderlijke paragrafen.

In aansluiting op het internationale gebruik en op de conventies van Engelstalige tijdschriften gebruiken we de punt als decimaalteken; we schrijven dus \(\frac{3}{2}=1.5\). Hierbij is een waarschuwing op zijn plaats: het decimale symbool kan verschillen tussen computers, en zelfs tussen programma's op dezelfde computer. Controleer dus welk decimaal symbool gebruikt wordt door (elk programma op) jouw computer.

Graag willen we onze mede-docenten danken voor de vele discussies en voorbeelden die op enige wijze verwerkt zijn in dit tekstboek. Onze studenten danken we voor hun nieuwsgierigheid en nauwkeurigheid die geleid heeft tot deze versie van dit tekstboek.

Ook betonen wij grote dank aan
Willemijn Heeren, Gerrit Bloothooft, Marijn Struiksma,
Margot van den Berg,
Els Rose,
Tobias Quené
en Kirsten Schutter
voor hun adviezen, data, en/of commentaar bij het manuscript van dit digitale tekstboek.

Utrecht, december 2016 - juli 2020

Hugo Quené, \url{https://www.hugoquene.nl}

Huub van den Bergh

\begin{center}\rule{0.5\linewidth}{0.5pt}\end{center}

This document is licensed under the \emph{GNU GPL 3} license (for details see
\url{https://www.gnu.org/licenses/gpl-3.0.en.html}). It was created with the \texttt{bookdown} package \citep{R-bookdown} in \href{https://www.rstudio.com}{Rstudio}.

\hypertarget{part-deel-1-methodologie}{%
\part{Deel 1: Methodologie}\label{part-deel-1-methodologie}}

\hypertarget{ch:inleiding}{%
\chapter{Inleiding}\label{ch:inleiding}}

In dit tekstboek worden de grondbeginselen, methoden en technieken van
empirisch wetenschappelijk onderzoek besproken, zowel in algemene zin
als toegespitst op het brede domein van taal en communicatie. We zullen ons bezighouden met vragen als: Wat is een goede onderzoeksvraag? Welke methode is de beste om de onderzoeksvraag te beantwoorden? Hoe kunnen onderzoekers zinnige en valide conclusies trekken uit (statistische analyses van) hun gegevens? In dit tekstboek beperken we ons tot de belangrijkste grondbeginselen, en tot de belangrijkste methoden en technieken. In dit eerste hoofdstuk zullen we een overzicht geven van verschillende typen en vormen van wetenschappelijk onderzoek. In het vervolg van dit tekstboek geven we de meeste aandacht aan methoden van wetenschappelijk onderzoek waarbij empirische observaties uitgedrukt worden in de vorm van getallen (kwantitatief), die geanalyseerd worden met behulp van statistische technieken.

\hypertarget{sec:wetenschappelijk-onderzoek}{%
\section{Wetenschappelijk onderzoek}\label{sec:wetenschappelijk-onderzoek}}

Om te beginnen moeten we een vraag stellen die terugslaat op de
allereerste zin hierboven: wat is eigenlijk wetenschappelijk onderzoek?
Wat is het verschil tussen wetenschappelijk en niet-wetenschappelijk
onderzoek (bijv. door onderzoeksjournalisten)? Onderzoek dat een
wetenschapper uitvoert, hoeft nog geen wetenschappelijk onderzoek te
zijn. Evenmin is journalistiek onderzoek per definitie
onwetenschappelijk omdat het door een journalist wordt uitgevoerd. In
dit tekstboek hanteren we de volgende definitie: ``(s)cientific
research is systematic, controlled, empirical, amoral, public, and
critical investigation of natural phenomena. It is guided by theory and hypotheses about the presumed relations among such phenomena'' \citep[p.14]{KL00}.

Wetenschappelijk onderzoek is systematisch en gecontroleerd.
Wetenschappelijk onderzoek is zodanig ontworpen dat we geloof kunnen
hechten aan de conclusies, omdat die conclusies goed onderbouwd zijn.
Het onderzoek kan door anderen herhaald worden, met (hopelijk) dezelfde
resultaten. Deze eis van repliceerbaarheid maakt ook dat
wetenschappelijk onderzoek zeer nauwgezet wordt ontworpen en uitgevoerd
(zie Hoofdstukken \ref{ch:integriteit} en \ref{ch:ontwerp}). De sterkste vorm van controle is die van een
wetenschappelijk experiment; we besteden daarom in dit tekstboek veel
aandacht aan experimenteel onderzoek (§\ref{sec:experimenteel-onderzoek}). Mogelijke alternatieve
verklaringen voor het onderzochte verschijnsel worden één voor één
onderzocht en zo mogelijk uitgesloten, zodat tenslotte slechts één
verklaring overblijft \citep{KL00}. Die verklaring vormt dan onze
wetenschappelijk onderbouwde conclusie of theorie over het onderzochte
verschijnsel.

Ook wordt in de definitie gesteld dat wetenschappelijk onderzoek
\emph{empirisch} van aard is. De conclusies die de onderzoeker trekt moeten
uiteindelijk gebaseerd zijn op (systematische en gecontroleerde)
waarnemingen of observaties van een verschijnsel in de werkelijkheid --- bijvoorbeeld op de waargenomen inhoud van een tekst, of op het
waargenomen gedrag van een proefpersoon. Als die waarneming ontbreekt,
dan kunnen de eventuele conclusies niet logisch verbonden worden met de werkelijkheid, waardoor ze geen wetenschappelijke waarde hebben.
Vertrouwelijke gegevens uit een onbekende bron, of inzichten verkregen
in een droom of in een mystieke beleving, zijn niet empirisch
onderbouwd, en kunnen dus niet de basis vormen van een wetenschappelijke theorie.

\hypertarget{sec:theorie}{%
\subsection{Theorie}\label{sec:theorie}}

Het doel van wetenschappelijk
onderzoek is te komen tot een theorie over een deel van de
werkelijkheid. Die theorie is te zien als een coherente en consistente
verzameling van ``justified true beliefs'' \citep{Mort03}. In deze
overtuigingen, en in de theorie, wordt geabstraheerd van de complexe
werkelijkheid van de natuurlijke verschijnselen, naar een abstract
mentaal \emph{construct}, dat uit zijn aard niet rechtstreeks waarneembaar
is. Voorbeelden van dergelijke constructen zijn: leesvaardigheid,
intelligentie, activatie-niveau, verstaanbaarheid, omvang van iemands
actieve woordenschat, schoenmaat, woon-werk-afstand, introvertheid, etc.

Een onderzoeker definieert in een theorie niet alleen verschillende
constructen, maar ook specificeert hij de \emph{verbanden} of relaties tussen
deze constructen. Pas wanneer zowel de constructen gedefinieerd zijn als
de relaties tussen de constructen gespecificeerd zijn, kan een
onderzoeker komen tot een systematische verklaring van het onderzochte
verschijnsel. Deze verklaring of theorie kan weer de basis zijn van een
\emph{voorspelling} over het onderzochte verschijnsel: het aantal gesproken
talen op de wereld zal verminderen in de 21e eeuw; teksten zonder
voegwoorden zullen moeilijker te begrijpen zijn dan teksten met
voegwoorden; kinderen die tweetalig opgroeien zullen niet slechter
presteren op school dan eentalige kinderen.

Wetenschappelijk onderzoek is er in vele verschillende typen en vormen,
die op verschillende manieren ingedeeld kunnen worden. In de volgende sectie \ref{sec:paradigmata} bespreken we een indeling op basis van
paradigma, de manier waarop de onderzoeker tegen de werkelijkheid
aankijkt. Onderzoek kan ook ingedeeld worden op een continuüm van
`zuiver theoretisch' naar `toegepast'. Een derde manier om onderzoek in
te delen is gericht op het type onderzoek, bijvoorbeeld
instrumentatieonderzoek (§\ref{sec:instrumentatie-onderzoek}), beschrijvend onderzoek (§\ref{sec:beschrijvend-onderzoek}),
en experimenteel onderzoek (§\ref{sec:experimenteel-onderzoek}).

\hypertarget{sec:paradigmata}{%
\section{Paradigmata}\label{sec:paradigmata}}

Eén criterium om typen onderzoek te onderscheiden is op basis van het
gebruikte paradigma, de manier waarop de onderzoeker tegen de
werkelijkheid aankijkt. In dit tekstboek besteden we nagenoeg alleen
aandacht aan het empirisch-analytisch paradigma, omdat dit het meest
uitgewerkte en meest invloedrijke paradigma is. Heden ten dage kan deze
benadering opgevat worden als `de' standaardopvatting, waar andere
paradigma's zich min of meer tegen afzetten.

Binnen het \emph{empirisch-analytische paradigma} onderscheiden we twee
varianten: het positivisme en het kritisch-rationalisme. Beide
stromingen hebben gemeen dat er aangenomen wordt dat er wetmatigheden
zijn die `ontdekt' kunnen worden: verschijnselen kunnen beschreven en
verklaard worden in abstracte termen (constructen). Het verschil tussen
beide stromingen binnen de empirisch-analytische traditie is gelegen in
de pretentie van de uitspraken die gedaan worden. Volgens de
positivisten is het mogelijk om uitspraken te doen vanuit feitelijke
waarnemingen naar een theorie. Op basis van de observaties kunnen we
generaliseren naar een algemeen geldende regel, door middel van
inductie. (De vogels die ik zie, die hoor ik ook fluiten, dus alle
vogels fluiten.)

De tweede stroming is het kritisch-rationalisme. De aanhangers van deze
stroming keren zich tegen bovengenoemde inducties: al hoor ik talloze
vogels ook fluiten, dan nog kan ik geen zekerheid verkrijgen over de
veronderstelde algemene regel. Maar we kunnen het wel omkeren, en
proberen aan te tonen dat de veronderstelde algemene regel of hypothese \emph{niet} juist is. Hoe werkt dat? Op basis van de algemeen geldende regel
kunnen we voorspellingen afleiden voor specifieke observaties, door
middel van deductie. (Als alle vogels fluiten, dan moet het zo zijn dat
alle vogels in mijn steekproef fluiten.) Als niet alle vogels in mijn
steekproef fluiten, dan is de algemene regel blijkbaar onjuist. Dit
wordt het falsificatie-principe genoemd; we bespreken dat uitgebreider
in sectie \ref{sec:falsificatie}.

Ook aan het kritisch-rationalisme kleven echter tenminste twee bezwaren.
Met het falsificatieprincipe kunnen waarnemingen (empirische feiten,
observaties, onderzoeksresultaten) gebruikt worden om theoretische
uitspraken te doen (met betrekking tot hypothesen). Strikt genomen moet
een veronderstelde algemene regel meteen verworpen worden na één
geslaagde falsificatie (een van de vogels in mijn steekproef fluit
niet): als theorie en observatie niet overeenstemmen, dan faalt de
theorie, volgens de kritisch-rationalisten. Maar om te komen tot een
observatie moet een onderzoeker vele keuzes maken (bijv.: hoe maak ik
een goede steekproef, wat is een vogel, hoe bepaal ik of een vogel
fluit?), die de geldigheid van de observaties onzeker kunnen maken. Er
kan dus ook iets mis zijn met de waarnemingen zelf (horen), of met de
operationalisaties van de gebruikte constructen (vogels, fluiten).

Een tweede probleem is dat er in de praktijk eigenlijk zeer weinig
theorieën zijn die werkelijk iets uitsluiten. Wanneer er discrepanties
waargenomen worden tussen theorie en observaties, dan wordt de theorie
bijgeschaafd, zodat de nieuwe observaties toch weer binnen de theorie
passen. Theorieën worden dan ook zelden volledig verworpen.

Een tweede paradigma is de kritische benadering. Het \emph{kritische paradigma}
onderscheidt zich van andere paradigmata in de nadruk op
maatschappelijke bepaaldheden; `de' werkelijkheid bestaat niet, ons
beeld ervan is een voorlopige, door maatschappelijke oorzaken bepaalde
werkelijkheid. Inzicht in de maatschappelijke verhoudingen heeft zelf
dus ook invloed op die werkelijkheid. Onze wetenschapsopvatting zoals
verwoord in bovengenoemde definities van onderzoek en theorie wordt in
het kritische paradigma dan ook afgewezen. Kritische onderzoekers menen
dat onderzoeksprocessen niet los gezien kunnen worden van de
maatschappelijke context waarin het onderzoek is verricht. Deze laatste
visie wordt overigens overgenomen door steeds meer onderzoekers, ook
door hen die andere paradigmata aanhangen.

\hypertarget{sec:instrumentatie-onderzoek}{%
\section{Instrumentatie-onderzoek}\label{sec:instrumentatie-onderzoek}}

Onderzoek is, zoals gezegd, een gesystematiseerde en gecontroleerde
wijze om empirische gegevens te verzamelen en te interpreteren.
Onderzoekers streven naar inzicht in natuurlijke verschijnselen, en in
de wijze waarop (de constructen van) die verschijnselen met elkaar
samenhangen. Een voorwaarde hiervoor is dat de onderzoeker deze
verschijnselen daadwerkelijk kan meten, d.i. uitdrukken in een
observatie (bij voorkeur in de vorm van een getal).
Instrumentatieonderzoek is voornamelijk gericht op de constructie van
instrumenten of methoden om verschijnselen, gedrag, vaardigheden,
attitudes, etc. meetbaar te maken. De ontwikkeling van goede
meetinstrumenten is bepaald geen sinecure: het is ambachtelijk handwerk,
waarbij de constructeur vele valkuilen moet zien te vermijden. Het
meetbaar maken van verschijnselen, van gedrag of van constructen noemen
we de \emph{operationalisatie}. Een concrete leestoets is bijvoorbeeld op te
vatten als een operationalisatie van het abstracte construct
`leesvaardigheid'.

We kunnen een nuttig onderscheid maken tussen het abstracte theoretische
construct en het gemeten construct, ofwel een onderscheid tussen: het
begrip-zoals-bedoeld en het begrip-zoals-bepaald. Het is uiteraard de
bedoeling dat het begrip-zoals-bepaald (de toets, de vragenlijst, de
observatie) het begrip-zoals-bedoeld (het theoretische construct) zo
goed mogelijk benadert. Indien het theoretische construct goed wordt
benaderd, dan spreken we van een adequate of valide meting.

Bij operationalisatie van een begrip-zoals-bedoeld moeten talloze keuzen
ge-maakt worden. Zo moet het CITO (Centraal instituut voor
toetsontwikkeling) elk jaar tekstbegripstoetsen construeren om de
leesvaardigheid van eindexamenkandidaten te meten. Daarvoor moet
allereerst een tekst gekozen of geredigeerd worden. Deze tekst mag niet
te moeilijk, maar ook niet te makkelijk zijn voor de doelgroep. Voorts
mag het onderwerp van de tekst niet al te bekend zijn, omdat anders de
bij sommige leerlingen aanwezige algemene kennis kan interfereren met de
meningen en standpunten die in de tekst naar voren gebracht worden.
Vervolgens moeten de vragen zó ontworpen worden dat de verschillende
passages in de tekst aan bod komen. Ook moeten de vragen zó samengesteld
zijn dat het theoretische construct `leesvaardigheid' adequaat
geoperationaliseerd wordt. Tot slot moet ook nog rekening gehouden
worden met de examens uit voorafgaande jaren; het nieuwe examen mag
immers niet al te veel afwijken van oude examens.

Een construct moet dus op de juiste wijze geoperationaliseerd zijn, om
observaties te verkrijgen die niet alleen valide zijn (een goede
benadering van het abstracte construct, zie Hoofdstuk \ref{ch:validiteit})
maar die ook betrouwbaar zijn (ongeveer
gelijke observaties bij herhaalde meting, zie
Hoofdstuk \ref{ch:betrouwbaarheid}).
In ieder onderzoek zijn de validiteit
en de betrouwbaarheid van een meting cruciaal; we besteden dan ook twee hoofdstukken aan deze begrippen.
Maar in instrumentatieonderzoek zijn
deze begrippen zelfs essentieel, omdat dit type onderzoek juist beoogt
om valide en betrouwbare instrumenten te leveren, die een goede
operationatisatie zijn van het abstracte construct-zoals-bedoeld.

\hypertarget{sec:beschrijvend-onderzoek}{%
\section{Beschrijvend onderzoek}\label{sec:beschrijvend-onderzoek}}

Met beschrijvend onderzoek bedoelen we onderzoek dat voornamelijk
gericht is op de beschrijving van een bepaald natuurlijk verschijnsel in
de werkelijkheid. De onderzoeker richt zich dus vooral op een
\emph{beschrijving} van het verschijnsel: het huidige vaardigheidsniveau, het
verloop van een proces of een discussie, de wijze waarop de lessen
Nederlands in het voortgezet onderwijs vorm worden gegeven, de politieke
voorkeur van stemmers vlak voor verkiezingen, de samenhang tussen het
aantal uren zelfstudie en het eindcijfer dat een student behaalt, etc.
Kortom, ook de onderwerpen voor beschrijvend onderzoek kunnen zeer
divers zijn.

\begin{center}\rule{0.5\linewidth}{0.5pt}\end{center}

\begin{quote}
\emph{Voorbeeld 1.1}: \citep{DTE13} hebben opnames van conversaties gekozen of gemaakt in 10 talen. Uit die opgenomen conversaties zijn woorden genomen waarmee een luisteraar om
``open verduidelijking'' vraagt: woordjes als \emph{hè} (Nederlands), \emph{huh}
(Engels), \emph{ã?} (Siwu). Van deze woorden is de klankvorm en het
toonhoogteverloop vastgesteld, met akoestische metingen en met
fonetische transcripties door experts. Een conclusie van dit
beschrijvende onderzoek luidt dat deze tussenvoegsels in de
verschillende talen veel meer op elkaar lijken (in klankvorm en
toonhoogteverloop) dan op grond van toeval te verwachten is.
\end{quote}

\begin{center}\rule{0.5\linewidth}{0.5pt}\end{center}

Dit voorbeeld illustreert dat beschrijvend onderzoek niet ophoudt als de
gegevens (klankvormen, toonhoogteverloop) beschreven zijn. Vaak zijn
\emph{verbanden} tussen de verzamelde gegevens ook zeer interessant (zie §@ref(sec:wetenschappelijk.onderzoek)).
Zo wordt in opiniepeilingen naar het stemgedrag bij verkiezingen vaak
een verband gelegd tussen het gepeilde stemgedrag enerzijds, en
leeftijd, geslacht en opleidingsniveau van de respondent anderzijds. En
evenzo wordt in onderwijskundig onderzoek een verband gelegd tussen
aantal uren studietijd enerzijds, en studiesucces van de respondent
anderzijds. Dit type van beschrijvend onderzoek, waarbij een correlatie
wordt vastgesteld tussen mogelijke oorzaken en mogelijke gevolgen, wordt
ook aangeduid als \emph{correlationeel onderzoek}.

Het essentiële verschil tussen beschrijvend en experimenteel onderzoek
is gelegen in de vraag naar oorzaak en gevolg. Op basis van beschrijvend
onderzoek kan een causaal verband tussen oorzaak en gevolg \emph{niet} goed
vastgesteld worden. Uit beschrijvend onderzoek zou kunnen blijken dat er
een samenhang is tussen een bepaald soort voeding en een langere
levensduur. Is het voedingspatroon dan ook de oorzaak van de langere
levensduur? Dat hoeft bepaald niet het geval te zijn: het is ook
mogelijk dat dat soort voeding vooral genuttigd wordt door mensen die
relatief hoog opgeleid en welvarend zijn, en door deze \emph{andere} factoren
ook relatief langer in leven zijn\footnote{Het is zelfs mogelijk dat het onderzochte voedingspatroon de
  oorzaak is van een relatief \emph{kortere} levensduur, maar dat dit
  negatieve effect gemaskeerd wordt door de sterkere, positieve
  effecten van opleidingsniveau en welvaartsniveau op de levensduur.}. Om vast te kunnen stellen of er
een causaal verband is, moeten we experimenteel onderzoek opzetten en
uitvoeren.

\hypertarget{sec:experimenteel-onderzoek}{%
\section{Experimenteel onderzoek}\label{sec:experimenteel-onderzoek}}

Experimenteel onderzoek wordt gekenmerkt doordat de onderzoeker een
bepaald aspect van de onderzoeksomstandigheden systematisch varieert
\citep{SCC02}. Het effect van deze manipulatie staat dan centraal in het
onderzoek. Een onderzoeker vermoedt bijvoorbeeld dat een bepaalde nieuwe
lesmethode zal resulteren in betere prestatie van de leerlingen dan de
huidige lesmethode. De onderzoeker wil deze hypothese toetsen door
middel van een experimenteel onderzoek. Hij manipuleert het type
onderwijs: sommige klassen of groepen krijgen les volgens de nieuwe
experimentele lesmethode en andere klassen of groepen krijgen les op de
traditionele wijze. Het effect van de nieuwe lesmethode wordt
geëvalueerd door de prestaties van de twee soorten schoolklassen te
vergelijken, na de `behandeling' met de oude vs.~nieuwe lesmethode.

Experimenteel onderzoek heeft als voordeel dat we de
onderzoeksresultaten doorgaans mogen interpreteren als het gevolg van de
experimentele manipulatie. Omdat de onderzoeker het onderzoek
systematisch controleert en slechts één aspect (i.c. de lesmethode)
varieert, kunnen eventuele verschillen tussen de prestaties van de twee
categorieën alleen toegeschreven worden aan het veranderde kenmerk, i.c.
aan de lesmethode. Dit veranderde kenmerk moet dan logischerwijs wel de
oorzaak zijn van de geobserveerde verschillen. Experimenteel onderzoek
is dus gericht op de evaluatie van causale verbanden.

Deze redenering vereist wel dat proefpersonen (of schoolklassen, in
bovenstaand voorbeeld) volgens het toeval, aselect (Eng. `at random'),
worden toegewezen aan de experimentele condities (i.c. de oude of nieuwe
lesmethode). Deze aselecte toewijzing (Eng. `random assignment') is de
beste methode om eventuele niet-relevante verschillen tussen de
behandelcondities uit te sluiten. Een dergelijk experiment met aselecte
toewijzing van proefpersonen aan condities wordt een \emph{gerandomiseerd
experiment} genoemd (Eng. `randomized experiment', `true experiment', ).
Om bij ons voorbeeld te blijven: als de onderzoeker de oude lesmethode
zou inzetten bij jongens, en de nieuwe lesmethode bij meisjes, dan is
een eventueel verschil in prestaties niet meer uitsluitend toe te
schrijven aan het gemanipuleerde kenmerk (de lesmethode), maar ook aan
een niet-gemanipuleerd maar wel relevant kenmerk, hier het geslacht van
de leerlingen. Zo'n mogelijk verstorend kenmerk wordt een \emph{storende
variabele} (Eng. `confound') genoemd. In
Hoofdstuk \ref{ch:ontwerp} bespreken we hoe we deze storende variabelen
kunnen neutraliseren, door random toewijzing van proefpersonen (of
schoolklassen) aan de experimentele condities, in combinatie met andere maatregelen.

Er is ook experimenteel onderzoek waarbij een bepaald aspect (zoals
lesmethode) wel systematisch varieert, maar waarbij de proefpersonen of
schoolklassen \emph{niet aselect} zijn toegewezen aan de experimentele
condities; dit wordt \emph{quasi-experimenteel} onderzoek genoemd \citep{SCC02}.
In het bovenstaande voorbeeld is daarvan sprake als de gebruikte
lesmethode onderzocht wordt, met gegevens van schoolklassen waarvan niet
de onderzoeker maar de docent bepaald heeft of de oude of nieuwe
lesmethode gebruikt wordt. Als de nieuwe lesmethode betere prestaties
zou opleveren, dan weten we \emph{niet} met zekerheid dat het verschil in
prestaties toe te schrijven is aan de lesmethode. Ook het enthousiasme
of de werkstijl van de docent kan een storende variabele zijn geweest in dit quasi-experiment. In dit tekstboek zullen we verschillende
voorbeelden van quasi-experimenteel onderzoek tegenkomen.

Binnen het type van experimenteel onderzoek kunnen we een verdere
verdeling aanbrengen, tussen laboratoriumonderzoek en veldonderzoek. In
beide typen experimenteel onderzoek wordt een aspect van de
werkelijkheid gemanipuleerd. Het verschil tussen beide typen onderzoek
is gelegen in de mate waarin de onderzoeker in staat is om allerlei
storende aspecten van de werkelijkheid onder controle te houden. In
laboratoriumonderzoek kan de onderzoeker zeer exact bepalen onder welke
omgevingscondities de observaties worden gedaan, en kan de onderzoeker
dus ook vele mogelijke storende variabelen onder controle houden (denk
aan verlichting, temperatuur, omgevingslawaai, etc.). In veldonderzoek
is dit niet het geval. De onderzoeker is `in het vrije veld' niet in
staat om alle (mogelijk relevante) aspecten van de werkelijkheid
volledig onder controle te houden.

\begin{center}\rule{0.5\linewidth}{0.5pt}\end{center}

\begin{quote}
\emph{Voorbeeld 1.2}: Margot van den Berg
onderzocht samen met collega's van de Universiteit van Ghana en de
Universiteit van Lomé hoe meertalige sprekers hun talen gebruiken als
zij eigenschappen zoals kleur, grootte en waarde moeten benoemen door
middel van een zogenaamde \emph{Director-Matcher task} \citep{BAEYT2017}. In deze
taak gaf de ene onderzoeksdeelnemer (de directeur) aanwijzingen aan een
ander (de uitvoerder) om een reeks voorwerpen in een bepaalde volgorde
neer te zetten. Zo konden in een kort tijdsbestek veel voorkomens van
eigenschapswoorden worden verzameld (`Zet de gele auto naast de rode
auto maar boven de kleine slipper'). De gesprekken werden opgenomen,
uitgeschreven en vervolgens onderzocht op taalkeuze, moment van
taalwisseling en type grammaticale constructie. Bij dergelijk veldwerk
kunnen echter allerlei niet-gecontroleerde aspecten in de omgeving van
invloed zijn op de geluidsopnames, en daarmee op de gegevens: ``kakelende kippen, een buurman die z'n motor aan het repareren is en 'm om de haverklap moet starten terwijl je een gesprek aan het opnemen bent,
keiharde regen op het aluminium dak van het gebouw waar de interviews
plaats vinden.'' (Margot van den Berg, pers.comm.)
\end{quote}

\begin{center}\rule{0.5\linewidth}{0.5pt}\end{center}

\begin{quote}
\emph{Voorbeeld 1.3}: Bij het luisteren naar gesproken zinnen kunnen we uit de oogbewegingen van een proefpersoon afleiden, hoe die gesproken zinnen worden verwerkt. In een zgn. `visual world'-taak krijgen luisteraars een zin te horen (bijv. ``Bert zegt dat het konijn is gegroeid''), terwijl ze kijken naar meerdere afbeeldingen op het scherm (meestal 4, bijv. een schelp, pauw, zaag, en wortel). Luisteraars blijken vooral te kijken naar de afbeelding die geassocieerd is aan het woord dat ze op dat moment mentaal verwerken: als ze het woord \emph{konijn} verwerken, dan kijken ze naar de wortel (exacter gezegd: ze kijken vaker en langer naar de wortel dan naar de andere afbeeldingen). Met een zgn. `eye tracker' kan worden vastgesteld naar welke positie van het scherm de proefpersoon kijkt (door observatie van de pupillen). De onderzoeker kan zo dus observeren welk woord op welk moment mentaal verwerkt wordt \citep{KMR12}. Dergelijk onderzoek kan het beste uitgevoerd worden in een laboratorium, met controle over achtergrondgeluiden, verlichting, en positie van de ogen t.o.v. computerscherm.
\end{quote}

\begin{center}\rule{0.5\linewidth}{0.5pt}\end{center}

Laboratoriumonderzoek en veldonderzoek hebben beide voordelen en
nadelen. Het grote voordeel van laboratoriumonderzoek is natuurlijk de
mate waarin de onderzoeker allerlei externe zaken onder controle kan
houden. In een laboratorium zal het experiment niet vaak verstoord
worden door een startende motor of door een regenbui. Dit voordeel van
laboratoriumonderzoek is echter ook een belangrijk nadeel, nl. dat het
onderzoek plaatsvindt in een min of meer kunstmatige omgeving. Het is
dan nog maar de vraag in hoeverre resultaten die onder kunstmatige
omstandigheden verkregen zijn, ook zullen gelden in het leven van
alledag buiten het laboratorium. Dit laatste is dan ook een punt in het
voordeel van veldonderzoek: het onderzoek wordt verricht onder
natuurlijke omstandigheden. Het nadeel van veldonderzoek is dan weer dat
er in het veld kan van alles gebeuren wat de onderzoeksresultaten
beïnvloedt, maar waar de onderzoeker geen controle over kan houden (zie het bovenstaande
voorbeeld). De keuze die een onderzoeker maakt tussen
beide typen experimenteel onderzoek wordt uiteraard sterk bepaald door
de vraagstelling van het onderzoek. Sommige vraagstellingen laten zich
beter in laboratoriumsituaties onderzoeken, terwijl andere beter in
veldsituaties onderzocht kunnen worden (zoals bovenstaande voorbeelden
illustreren).

\hypertarget{vooruitblik}{%
\section{Vooruitblik}\label{vooruitblik}}

Dit tekstboek bestaat uit drie delen. Deel I (hoofdstukken 1 tot en met 7) van dit tekstboek behandelt
methoden van onderzoek, en geeft een toelichting bij allerlei termen en
begrippen die van belang zijn bij het ontwerpen en opzetten van goed
wetenschappelijk onderzoek.

In deel II (hoofdstukken 8 tot en met 12) van het tekstboek behandelen we de beschrijvende statistiek
(Eng. `descriptive statistics') en in deel III (hoofdstukken 13 tot en met 17) behandelen we de
elementaire technieken uit de toetsende statistiek (Eng. `inferential
statistics'). Met deze laatste twee delen streven we drie doelen na.

Allereerst
willen we dat je in staat bent om artikelen en andere verslagen waarin
statistische verwerkings- en toetsingstechnieken zijn gebruikt, kritisch
te beoordelen.
Ten tweede willen we dat je dat je de noodzakelijke
kennis en inzicht hebt in de belangrijkste statistische procedures.
Ten derde willen we met deze statistische delen bereiken dat je in staat
bent om zelfstandig statistische bewerkingen uit te voeren voor je eigen
onderzoek, bijvoorbeeld voor je stage of eindwerkstuk.

Deze drie doelen zijn geordend in volgorde van belangrijkheid. Wij menen
dat een adequate en kritische interpretatie van statistische resultaten
en de conclusies die daaraan verbonden kunnen worden van groot belang is
voor alle studenten. Om die reden besteden we in dit tekstboek dan ook
relatief veel aandacht (in deel I) aan de `filosofie' of methodologie achter de besproken statistische technieken en analyses. Ook geven we
aan hoe je de besproken statistische analyses zelf kunt uitvoeren in
SPSS (een populair pakket voor statistische analyses) en in R (een wat
moeilijker, maar ook krachtiger en veelzijdiger pakket, met stijgende
populariteit). Beide statistische pakketten zijn geïnstalleerd in de computerleerzalen van de Faculteit Geesteswetenschappen. SPSS is beschikbaar via SurfSpot.nl voor een sterk gereduceerde prijs. R is vrijelijk beschikbaar via www.R-project.org. Meer achtergrond over het gebruik van R is te vinden via \url{https://hugoquene.github.io/emlar2020/}.

  \bibliography{book.bib,packages.bib,hhmhto.bib}

\end{document}
